
\section{Thesis Context}
\label{sec:thesis_context}

This thesis has been developed in the context of the research centre Deusto
Institute of Technology, DeustoTech, University of Deusto\footnote{\url{http://www.deustotech.deusto.es/}}.
The work that has made possible the development of this thesis is in the context 
of the following research projects:

\begin{itemize}
  \item \acs{piramide}: Funded by the Spanish Ministry of Industry, Tourism and 
  Trade, the \ac{piramide} project proposes to use user mobile device as a 
  catalizer of the interaction of users with their environment, acting as a 
  sixth sense which aids and assists us facilitating and improving our daily 
  interactions with the objects that surround us in our workplace, home or 
  public administrations.
  
%   \item UCADAMI: The User and Context-aware Dynamically Adaptable Mobile 
%   Interfaces project, funded by the Industry, Innovation, Commerce and Tourism 
%   Department of the Basque Government aimed to
  
  \item \acs{dynui}: The aim of \ac{dynui} is to define an intelligent 
  platform that facilitates the development and deployment of user-environment 
  interfaces adaptable to the users, their interaction devices and their 
  context. These interfaces have to be adapted both at the beginning and 
  during the execution of services taking into account the users' capacities, 
  their interaction devices and the users' and their environment's current 
  context.
 
\end{itemize}





