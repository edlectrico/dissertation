\section{Hypothesis, Goals and Limitations}
\label{sec:hypothesis}
Based on the current state of Adaptive User Interfaces, the following hypothesis
is developed:
 
\begin{framed}
\textit{User interaction limitations with user interfaces in mobile devices due 
to users' context disabilities are reduced by dynamically adapting the 
corresponding applications' user interfaces through a semantic reasoning process 
which includes: their capabilities as users, the set of characteristics which 
defines the current environment where they actually are, and the devices they 
use. }
\end{framed}

This hypothesis is validated undertaking the following main goal:

\begin{framed}
 To design and implement an adaptive user interface system which runs fully in
 the user's device, includes current context situation, and considers several
 temporary or enduring user disabilities, supported by different sets of rules
 which makes the adaptation transparent for the user.
\end{framed}


This objective is achieved through the attainment of the following more specific
steps:

\begin{enumerate}
  \item To study the current state of the art on user interface adaptation 
  systems; user, context and device modelling; and mobile reasoning engines.

  \item To design an ontology which models user capabilities through an abstract
  perspective, context several situations and several device static and non 
  static characteristics. The ontology must consider possible interactions 
  between each entity.

  \item To design a set of rules which allow the interaction between the cited
  entities and the final adaptation of the user interface.

  \item To design and implement a reasoning mobile engine which allows reasoning
  in Android based devices.

  \item To provide an \ac{api} for developers to make available the design of 
  adaptive user interfaces and the edition of existing rules sets, as the 
  knowledge represented through the ontology.

  \item To validate the obtained results both qualitatively and quantitatively.
  
%   \item To model this evolution through a process which will be able
% to dynamically adapt the best and most suitable user interface for each 
% precise 
% context situation.
  
%   \item To demonstrate that it is possible to develop dynamic 
% adaptive applications which are able to reduce the impact of users' 
% disabilities taking into account their own capabilities, the devices' 
% characteristics and those which belong to the current context.
\end{enumerate}


% These general objectives are achieved by fulfilling the following more 
% specific
% steps:
% 
% \begin{itemize}
%   \item To study the current state of the art on adaptive user interfaces and 
% on
%   users, context and device modelling techniques and solutions.
%   
%   \item To design a model, focused on adaptive user interfaces domain, which 
% will allow developers and researchers to model several user capabilities, 
% context situation and device characteristics, their evolution and their 
% influences within the current environment.
%   
%   \item To design an interaction model which will gather several aspects of 
% the user current status and satisfaction with the presented and adapted user 
% interface.
% \end{itemize}


The resulting methodology should also satisfy the following requirements:

\begin{itemize}
  \item The designed model should be descriptive, complete and robust enough to
  be able to represent any possible context-aware situation with the users and
  their devices.
  
  \item The model should represent several user capabilities through an abstract 
  perspective, considering that designers, developers and users might lack of 
  physiological background of capabilities.
  
  \item The designed model should be fully domain independent. Consequently, it
  should be exportable and reusable to external user interface adaptation 
  environments.
  
  \item The model should represent physiological capabilities transparently for
  the user and the developer of the adaptive user interface.
  
  \item The designed and implemented reasoning engine should be compatible with
  Android based devices, allowing the use of rules and reasoning features.
  
  \item The designed \ac{api} should allow developers to design automatically
  adaptive user interfaces and also the modification of the knowledge 
  represented by the ontology.
\end{itemize}

The following features will be considered beyond the scope of this research:

\begin{itemize}
  \item Only capabilities related to visual and hearing are taken into account
  for the dynamic adaptation and the user context disability approach. 
  
  \item No cognitive problems have been faced.
  
  \item In the adaptation process the device battery level is not taken under
  consideration.
\end{itemize}
