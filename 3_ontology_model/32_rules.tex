\section{The adaptation rules set}
\label{sec:adaptui_rules}

As said in the introduction of this chapter, the AdaptUI platform is supported
by two bases. The first one, the AdaptUIOnt ontology, described in the previous
sections. The second one, a set of rules that uses the knowledge represented
by the ontology and triggers several actions aiming the user interface adaptation.


In AdaptUI the adaptation process is understood as a three step procedure: 

\begin{itemize}
 \item First, the knowledge of the main entities needs to be collected. Once it
 is gathered, a normalization process needs to be performed. This process classifies
 the collected information so it can be easily read. For example, a noise sensor
 might sense a 100 \acp{db} sound. Initially, we may not understand this value. Thus,
 using several ranges explained during this chapter (see Table~\ref{tbl:luminance},
 Table~\ref{tbl:sounds} and Table~\ref{tbl:batteries}) a high level classification
 is carried out.
 
 \item Second, and taking into account the transformed knowledge in the previous
 step, several actions need to be performed to adapt the user interface to the
 current situation (represented through the ontology in the previous stage).
 
 \item Finally, as we believe the whole process should be iterative, a refinement
 of the adapted user interface is available to improve the final results.
\end{itemize}

This conceptual division of the adaptation process in AdaptUI results in a three
set of different rules:

\begin{itemize}
 \item Pre-adaptation rules: this group of rules are designed to check those
 values which come from each entity of the Entities Model in order to translate
 them to the auxiliary models (see the primary model in Figure~\ref{fig:entities_characteristics}).
 For example, a context brightness value of 35,000 indicates the amount of \ac{lx}
 of the current environment. This value is then classified to a more verbose
 subgroup (in this case, to \textit{direct\_sunlight}); a device battery
 value of 10 implies the percentage of the remaining battery. This is translated
 as a \textit{non\_sufficient} battery level in the auxiliary device class. Thus,
 the corresponding adaptation rule checks these new parameters and decides whether
 the adaptation should follow one path or the other. Here we consider that 10\%
 of the total battery levels for a device might not be enough to run an
 adaptation process. Of course this is not accurate. The user might be able to
 use a nearby power adaptor. Besides, a 20\% battery level might not be enough 
 if the user aims to browse the web, make a video call, and so forth. This means
 that these values are established by us regarding the just the adaptation, but
 they directly depend on the user context. This set of rules directly affects
 the \textit{UserAux}, \textit{ContextAux} and \textit{DeviceAux} classes, being
 the main reason for these classes to belong to the Dynamic Model introduced in
 Section~\ref{sec:dynamic_model}.
 
 \item Adaptation rules: depending on the checked rules and the auxiliary classes’
 status, different rules are triggered. These rules result in different values
 for the Adaptation class, which is the class queried in the device platform for
 bringing the final adaptation to the device. These rules affect the 
 \textit{Adaptation} class.
 
 \item Post-adaptation rules: Once the user interface is adapted, a concrete
 architecture module (see Section \ref{sec:adaptation_polisher}) monitors the
 user activity. Hence, through a series of usability metrics (see 
 Section~\ref{sec:usability_metrics}) the the adaptation is considered satisfactory 
 by AdaptUI. If it detects that the usability level is insufficient, these rules
 are triggered changing the user model.
\end{itemize}

\footnotesize

% \begin{center}
  \begin{equation} \label{ec:pre_rule1}
  \begin{align*} 
  Device(?device) ∧ deviceIsDefinedBy(?device, ?chars) ∧ Battery(?battery) ∧ \\
  DeviceAux(?daux) ∧ deviceBatteryHasLevel(?battery, ?batterylevel) ∧ \\
  lessThanOrEqual(?batterylevel, 100) ∧ greaterThanOrEqual(?batterylevel, 15) \\
  \Rightarrow \\
  deviceAuxBatteryIsSufficient(?daux, true)
  \end{align*}
  \end{equation}
% \end{center}

\normalsize

Equation~\ref{ec:pre_rule1} shows an example of a pre-adaptation rule used in
AdaptUI. Equation~\ref{ec:rule2} shows an example of an adaptation rule. These
rules have been written using the Protégé rules editor, which is based on the
\ac{swrl} syntax~\citep{swrl}.

\footnotesize
%  \begin{center}
  \begin{equation} \label{ec:rule2}
  \begin{align*} 
  UserAux(?user) ∧ userAuxHasDisplayBrightness(?user, ?brightness) ∧ \\ 
  equal(?brightness, 255) ∧ Adaptation(?adaptation) ∧  DeviceAux(?device) ∧ \\
  deviceAuxBatteryIsSufficient(?device, ?battery) ∧  equal(?battery, true) \\
  \Rightarrow \\
  adaptui:adaptationBrightnessHasValue(?adaptation, ?brightness)
  \end{align*}
  \end{equation}
% \end{center}
\normalsize

\begin{table}[H]
  \caption{AdaptUI adaptation rules set. The asterisk at the end of each rule
  means that there is a corresponding equivalent rule taking into account
  Table~\ref{tbl:sounds}, Table~\ref{tbl:luminance} and Table~\ref{tbl:batteries}.
  Thus, there is a \textit{CheckLightLevelSunrise} or a \textit{CheckNoiseLevelTraffic}.
  The whole rules set is detailed in Appendix~\ref{cha:appendixB}.}
 \label{tbl:pre_adaptui_rules_set}
\footnotesize
\centering
 \begin{tabular}{l l l}
  \hline 
  \textbf{Rule group} 	& \textbf{Rule}					& \textbf{Description}						\\
  \hline
  Pre-adaptation 	& \textit{CheckBatteryLevelIsSufficient} 	& Considering Table~\ref{tbl:batteries} this rule evaluates	\\ 
			& 						& if the current battery level is enough to			\\
			& 						& perform any adaptation.					\\
			& \textit{CheckLightLevel*}			& These rules evaluate the context light input  		\\
			& 						& through sensors by using the classification 			\\
			& 						& shown in Table~\ref{tbl:luminance}. The result of  this	\\
			& 						& evaluation would be a verbose concept (e.g.,			\\
			&						& \textit{sunrise}) which is stored in the ontology		\\
			&						& using the \textit{contextAuxHasLightLevel} datatype 		\\
			&						& property							\\
			& \textit{CheckNoiseLevel*}			& These rules evaluate the context noise input 			\\
			& 						& through sensors by using the classification 			\\
			& 						& shown in Table~\ref{tbl:sounds}. The result of this		\\
			& 						& evaluation would be a verbose concept (e.g.,	 		\\
			&						& \textit{sunrise}) which is stored in the ontology  		\\
			&						& using the \textit{contextAuxHasNoiseLevel} datatype  		\\
			&						& property.							\\
			& \textit{CheckUserHasRestriction}		& If the current activity impedes the user, these 		\\
			& 						& rules would store the corresponding boolean 			\\
			&						& value in the \textit{userAuxHasRestriction} datatype 		\\
			&						& property.							\\
  Adaptation		& \textit{Adapt*}				& The adaptation rules are triggered once the 			\\
			&						& checking rules and the user, context and			\\
			&						& device characteristics are stored.				\\
  Post-adaptation	& \textit{CheckUsabilityMetric*}		& These set of rules check the cited usability metrics 		\\
			& 						& shown in Table~\ref{tbl:effectiveness_metrics}.		\\
			& \textit{CheckProductivityMetric*}		& These set of rules check the cited usability metrics 		\\
			&						& shown in Table~\ref{tbl:productivity_metrics}.		\\
  \hline

\end{tabular}
\end{table}


The given set of rules by AdaptUI have been included as they are understood as
vital for a precise adaptation. Nevertheless, and as it will be explained in
Chapter~\ref{cha:architecture}, these rules are not final. This means that they
are modifiable. To this end, AdaptUI provides a series of tools for developers
that allow them to change the knowledge managed by the platform.


\subsection{Conclusions}
\label{sec:rules_conclusions}

In this second part of the chapter the second main basis of the AdaptUI platform
has been presented: the AdaptUI set of rules. Divided into three different groups
(considering that the adaptation process is understood as a three step procedure)
these rules aim to finally generate a user interface adaptation under the following
steps:

\begin{itemize}
  \item Firstly, a classification and normalization of the collected knowledge of
  the main entities is needed. This might be low level knowledge. Thus, a higher
  level information is obtained during this process. To reach this goal, the
  pre-adaptation rules set is needed. These rules affect the \textit{UserAux},
  \textit{ContextAux} and \textit{DeviceAux} classes.
  
  \item Next, once the knowledge has been normalized, the adaptation rules set
  are triggered, generating the corresponding changes in the \textit{Adaptation}
  class in the AdaptUIOnt ontology.
  
  \item Finally, as the adaptation might be improvable, several rules are executed
  regarding a refinement of the last adapted user interface. This is detailed
  in Section~\ref{sec:adaptation_polisher}.
\end{itemize}