\section{The AdaptUI Rules Set}
\label{sec:adaptui_rules}

As said in the introduction of this chapter, the AdaptUI platform is supported
by two bases. The first one, the AdaptUIOnt ontology, described in the previous
sections. The second one, a set of rules that uses the knowledge represented
by the ontology and triggers several actions aiming the user interface adaptation.


In AdaptUI the adaptation process is understood as a three step procedure: 

\begin{itemize}
 \item First, the knowledge of the main entities needs to be collected. Once it
 is gathered, a normalization process needs to be performed. This process classifies
 the collected information so it can be easily read. For example, a noise sensor
 might sense a $100$~\acp{db} sound. Initially, we may not understand this value. Thus,
 using several ranges explained during this chapter (see Table~\ref{tbl:luminance},
 Table~\ref{tbl:sounds} and Table~\ref{tbl:batteries}) a high level classification
 is carried out.
 
 \item Second, and taking into account the transformed knowledge in the previous
 step, several actions need to be performed to adapt the user interface to the
 current situation (represented through the ontology in the previous stage).
 
 \item Finally, as we believe the whole process should be iterative, a refinement
 of the adapted user interface is available to improve the final results.
\end{itemize}

This conceptual division of the adaptation process in AdaptUI results in a three
set of different rules, which are detailed below.

\subsection{The Pre-adaptation Rules}
This group of rules are designed to check those values which come from each entity 
of the Entities Model in order to translate them to the auxiliary models (see 
the primary model in Figure~\ref{fig:entities_characteristics}). For example, a 
context brightness value of $35,000$ indicates the amount of \ac{lx} of the 
current environment. This value is then classified to a more verbose subgroup 
(in this case, to \textit{direct\_sunlight}); a device battery value of $10$ 
implies the percentage of the remaining battery. This is translated as a 
\textit{non\_sufficient} battery level in the auxiliary device class. Thus,
the corresponding adaptation rule checks these new parameters and decides whether
the adaptation should follow one path or the other. Here we consider that $10$\%
of the total battery levels for a device might not be enough to run an adaptation
process. Of course this is not accurate. The user might be able to use a nearby 
power adaptor. Besides, a $20$\% battery level might not be enough if the user 
aims to browse the web, make a video call, and so forth. This means that these 
values are established by us regarding the just the adaptation, but they directly 
depend on the user context. This set of rules directly affects the 
\textit{UserAux}, \textit{ContextAux} and \textit{DeviceAux} classes, being the 
main reason for these classes to belong to the Dynamic Model introduced in 
Section~\ref{sec:dynamic_model}.

Table~\ref{tbl:pre_adaptation_rules} describe the default pre-adaptation rules
included in AdaptUIOnt.

\begin{table}
  \caption{The AdaptUIOnt pre-adaptation rules.}
 \label{tbl:pre_adaptation_rules}
\footnotesize
\centering
 \begin{tabular}{l l l}
  \hline 
  \textbf{Pre-adaptation rule} 			& \textbf{Involved ontologies} 	& \textbf{Description} 	\\
						& \textbf{and classes} 		& 			\\
  \hline
  \textit{checkBatteryLevelIsSufficient}& \textit{soupa:Battery}	& Considering Table~\ref{tbl:batteries} this rule\\
					& \textit{adaptui:Device}	& evaluates if the current battery 		\\
					& \textit{adaptui:DeviceAux}	& level is enough to perform any 		\\
					&				& adaptation. 					\\
  \hline
  
  \textit{checkLightLevelDarkOvercast}	& \textit{adaptui:Context}	& These rules evaluate the context   		\\
  \textit{checkLightLevelTwilight}	& \textit{adaptui:ContextAux}	& light input through sensors by  		\\
  \textit{checkLightLevelSunrise}	& \textit{gumo:Light}		& using the classification shown in 		\\
  \textit{checkLightLevelOvercast}	& 				& Table~\ref{tbl:luminance}. The result of this \\
  \textit{checkLightLevelOfficeLightning}&				& evaluation would be a verbose  		\\
  \textit{checkLightLevelOfficeHallway}	&				& concept (e.g., \textit{sunrise}) which is   	\\
  \textit{checkLightLevelMoonlessClear}	&				& stored in the ontology using the  		\\
  \textit{checkLightLevelDayLight}	& 				& \textit{contextAuxHasLightLevel} datatype	\\
  \textit{checkLightLevelDirectSunlight}& 				&  property.					\\
  \textit{checkLightLevelFullMoon}	& 				& 						\\
  \textit{checkLightLevelLivingRoom}	& 				& 						\\
  \hline
  
  \textit{checkNoiseLevelBreathing}	& \textit{adaptui:Context}	& These rules evaluate the context 	 	\\
  \textit{checkNoiseLevelTruck}		& \textit{adaptui:ContextAux}	& noise input through sensors by  		\\
  \textit{checkNoiseLevelTrain}		& \textit{gumo:Noise}		& using the classification  shown in 		\\
  \textit{checkNoiseLevelTraffic}	& 				& Table~\ref{tbl:sounds}. The result of this 	\\
  \textit{checkNoiseLevelOffice}	&				& evaluation would be a verbose 		\\
  \textit{checkNoiseLevelLibrary}	&				& concept (e.g., \textit{sunrise}) which is  	\\
  \textit{checkNoiseLevelLeavesMurmuring}&				& stored in the ontology using the 		\\
  \textit{checkNoiseLevelBreathing}	&				& \textit{contextAuxHasNoiseLevel} datatype	\\
  \textit{checkNoiseLevelBuildingWork}	&				& property.					\\
  \textit{checkNoiseLevelConversation}	&				& 						\\
  \textit{checkNoiseLevelFactory}	&				& 						\\
  \textit{checkNoiseLevelGig}		&				& 						\\
  \textit{checkNoiseLevelJackhammer}	&				& 						\\
  \hline
  
  \textit{checkUserHasAttentionRestriction}& \textit{adaptui:Activity}	& If the current activity impedes the 		\\
  \textit{checkUserHasNoRestriction}	& \textit{adaptui:UserAux}	& user, these rules would store the 		\\
  \textit{checkUserHasSightRestriction}	&				& corresponding boolean value in the 		\\
  \textit{checkUserHasMovementRestriction}&				& \textit{userAuxHasRestriction} datatype 	\\
  \textit{checkUserHasHandsRestriction}	&				& property.					\\
  \textit{checkUserHasHearingRestriction}&				& 						\\
  \hline
\end{tabular}
\end{table}

Equation~\ref{ec:pre_adaptation_rule} shows an example of the 
\textit{checkNoiseLevelTraffic} pre-adaptation rule.

\footnotesize
\begin{equation} \label{ec:pre_adaptation_rule}
  \begin{align*} 
  Context(?context) ∧ Noise(?noise) ∧ ContextAux(?caux) ∧ \\
  contextHasNoise(?noise, ?value) ∧ lessThanOrEqual(?value, 70) ∧ \\
  greaterThan(?value, 60) \\
  \Rightarrow \\
  contextAuxHasNoiseLevel(?caux, "traffic")
  \end{align*}
\end{equation}
\normalsize


\subsection{The Adaptation Rules}
Depending on the checked rules and the auxiliary classes’ status, different 
rules are triggered. These rules result in different values for the Adaptation 
class, which is the class queried in the device platform for bringing the final 
adaptation to the device. These rules affect the \textit{Adaptation} class.
Table~\ref{tbl:adaptation_rules} details the default adaptation rules within
the AdaptUIOnt ontology.

\begin{table}
  \caption{The AdaptUIOnt adaptation rules.}
 \label{tbl:adaptation_rules}
\footnotesize
\centering
 \begin{tabular}{l l l}
  \hline 
  \textbf{Adaptation rule} 	& \textbf{Involved ontologies} 	& \textbf{Description} 		\\
				& \textbf{and classes} 		& 				\\
  \hline
  \textit{adaptBrightness}	& \textit{adaptui:Adaptation}	& These rules consider the  	\\
				& \textit{adaptui:DeviceAux}	& values stored in the 		\\
				& \textit{adaptui:UserAux}	& \textit{UserAux} class to adapt the \\
  \textit{adaptVolume}		& \textit{adaptui:Brightness}	& brightness and volume 	\\
				& \textit{adaptui:Volume}	& accordingly. 			\\
				& \textit{adaptui:Display}	& 				\\
  \hline 
  \textit{adaptButtonSize}	& \textit{adaptui:Adaptation}	& These rules take into account	\\
				& \textit{adaptui:DeviceAux}	& the current button configu-	\\
				& \textit{adaptui:UserAux}	& ration and the sensed context	\\
  \textit{adaptButtonColor}	& \textit{adaptui:Button}	& disabilities.			\\
  \textit{adaptButtonBackgroundColor}&				& 				\\
  \textit{adaptButtonTextColor}	& 				& 				\\

  \hline
  \textit{adaptEditTextSize}	& \textit{adaptui:Adaptation}	& These rules take into account	\\
				& \textit{adaptui:DeviceAux}	& the current edit text configu-\\
				& \textit{adaptui:UserAux}	& ration and the sensed	context	\\
  \textit{adaptEditTextBackgroundColor}	& \textit{adaptui:EditText}& disabilities.		\\
  \textit{adaptEditTextTextSize}& 				& 				\\
  \textit{adaptEditTextTextColor}&				& 				\\

  \hline 
  \textit{adaptTextViewtSize}	& \textit{adaptui:Adaptation}	& These rules take into account	\\
				& \textit{adaptui:DeviceAux}	& the current text view configu-\\
				& \textit{adaptui:UserAux}	& ration and the sensed	context	\\
  \textit{adaptTextViewtBackgroundColor}& \textit{adaptui:TextView}& disabilities.		\\
  \textit{adaptTextViewtTextSize}&				& 				\\
  \textit{adaptTextViewtTextColor}&				& 				\\
  \hline 
\end{tabular}
\end{table}

Equation~\ref{ec:adaptation_rule} shows an example of the \textit{adaptBrightness} 
adaptation rule.

\footnotesize
\begin{equation} \label{ec:adaptation_rule}
  \begin{align*} 
  Adaptation(?adaptation) ∧ DeviceAux(?device) ∧ UserAux(?user) ∧ \\
  deviceAuxBatteryIsSufficient(?device, ?battery) ∧ equal(?battery, true) ∧ \\
  userAuxHasDisplayBrightness(?user, ?brightness) ∧ equal(?brightness, 255)\\ 
  \Rightarrow \\
  adaptationBrightnessHasValue(?adaptation, ?brightness)\\
  \end{align*}
\end{equation}
\normalsize


\subsection{The Usability Rules}
In order to check the usability satisfaction of the user with the provided
adapted user interface, the usability rules set is provided. By checking several
usability metrics (detailed in Section~\ref{sec:usability_metrics}) these rules
determine if the interaction with the adapted user interface might be considered
enough. Table~\ref{tbl:usability_rules} introduced the usability rules included
in AdaptUIOnt.

\begin{table}
  \caption{The AdaptUIOnt usability rules. The metrics mentioned in this table
  are detailed in Table~\ref{tbl:effectiveness_metrics} and Table~\ref{tbl:productivity_metrics}.}
 \label{tbl:usability_rules}
\footnotesize
\centering
 \begin{tabular}{l l l}
  \hline 
  \textbf{Usability rule} 	& \textbf{Involved ontologies} 	& \textbf{Description} 		\\
				& \textbf{and classes} 		& 				\\
  \hline
  \textit{checkTaskEffectiveness}&\textit{adaptui:UserAux}	& It measures the proportion of \\
				& \textit{adaptui:Effectiveness}& goals of the task achieved	\\
				& \textit{adaptui:Polisher}	& correctly. 			\\
  \textit{checkTaskCompletion}	& 				& It measures the proportion of	\\
				& 				& the task that is completed.	\\
  \textit{checkErrorFrequency}	& 				& It measures the frequency of 	\\
				& 				& errors.			\\
  \hline
  \textit{checkTaskTime}	& \textit{adaptui:UserAux}	& It measures the required time	\\
				& \textit{adaptui:Productivity}	& to complete the current task.	\\
  \textit{checkTaskEfficiency}	& \textit{adaptui:Polisher}	& It measures the efficiency of	\\
				& 				& the user.			\\
  \textit{checkEconomicProductivity} &				& It measures the cost-effectiveness\\
				&				& of the user.			\\
  \textit{checkProductiveProportion} & 				& It measures the proportion of \\
				& 				& the time the user is performing\\
				&				& productive actions.		\\
  \textit{checkRelativeUserEfficiency}& 			& It compares the efficiency of the\\
				&				& user compared to an expert.	\\
  \hline
\end{tabular}
\end{table}

Equation~\ref{ec:usability_rule} shows an example of the \textit{checkRelativeEfficiency} 
adaptation rule.

\footnotesize
\begin{equation} \label{ec:usability_rule}
  \begin{align*} 
  UserAux(?user) ∧ Productivity(?productivity) ∧ Polisher(?polisher) ∧\\ 
  userAuxHasProductivityMetrics(?user, ?productivity) ∧ \\
  hasRelativeEfficiency(?productivity, ?efficiency) ∧ \\
  lessThanOrEqual(?efficiency, 0.5) ∧ \\
  \Rightarrow \\
  launchPolisherRules(?polisher, true)
  \end{align*}
\end{equation}
\normalsize

\subsection{The Post-adaptation Rules}
Once the user interface is adapted, a concrete architecture module (see 
Section~\ref{sec:adaptation_polisher}) monitors the user activity. Hence, 
through a series of usability metrics (see Section~\ref{sec:usability_metrics}) 
the the adaptation is considered satisfactory by AdaptUI. If it detects that the 
usability level is insufficient, these rules are triggered changing the user model.

\begin{table}
  \caption{The AdaptUIOnt post-adaptation rules. For edit texts and text views
  the same rules that are applied for the buttons are provided.}
 \label{tbl:post_adaptation_rules}
\footnotesize
\centering
 \begin{tabular}{l l l}
  \hline 
  \textbf{Adaptation rule} 	& \textbf{Involved ontologies} 	& \textbf{Description} 		\\
				& \textbf{and classes} 		& 				\\
  \hline
  \textit{incrementBrightness}	& \textit{adaptui:UserAux}	& It increments the brightness of\\
				& \textit{adaptui:Polisher}	& the device in $1$F.		\\
  \textit{decrementBrightness}	& \textit{adaptui:Adaptation}	& It decrements the brightness of\\
				& 				& the device in $1$F.		\\
  \textit{incrementVolume}	& 				& It increments the volume of	\\
				& 				& the device in $1$ unit.	\\
  \textit{decrementVolume}	& 				& It decrements the volume of	\\
				& 				& the device in $1$ unit.	\\
  \hline
  \textit{incrementButtonSize}	& \textit{adaptui:UserAux}	& It increments the size of the \\
				& \textit{adaptui:Polisher}	& button adding $10$ dpis. 	\\
  \textit{decrementButtonSize}	& \textit{adaptui:Adaptation}	& It decrements the size of the \\
				& 				& button in $10$ dpis.		\\
  \textit{darkenButtonBackgroundColor}&				& It darkens the colour.	\\
  \textit{lightenButtonBackgroundColor}&			& It lightens the colour.	\\
  \textit{darkenButtonTextColor}&				& It darkens the text colour.	\\
  \textit{lightenButtonTextColor}&				& It lightens the text colour.	\\
  \hline
\end{tabular}
\end{table}

Equation~\ref{ec:post_adaptation_rule} shows an example of the \textit{incrementButtonSize} 
post-adaptation rule.

\footnotesize
\begin{equation} \label{ec:post_adaptation_rule}
  \begin{align*} 
  Polisher(?polisher) ∧ launchPolisherRules(?polisher, true) ∧\\
  UserAux(?user) ∧ userAuxHasEffectivenessMetrics(?user, ?effectiveness) ∧ \\
  effectivenessMetricHasErrorFreequency(?effectiveness, ?freq?) ∧\\
  greaterThan(?freq, 0.5) ∧ Adaptation(?adaptation) ∧\\
  adaptationHasButtonSize(?adaptation, ?size)
  \Rightarrow \\
  adaptationButtonHasSize(?adaptation, ?size + 10)
  \end{align*}
\end{equation}
\normalsize



The given set of rules by AdaptUI have been included as they are understood as
vital for a precise adaptation. Nevertheless, and as it will be explained in
Chapter~\ref{cha:architecture}, these rules are not final. This means that they
are modifiable. To this end, AdaptUI provides a series of tools for developers
that allow them to change the knowledge managed by the platform.


\subsection{Conclusions}
\label{sec:rules_conclusions}

In this second part of the chapter the second main basis of the AdaptUI platform
has been presented: the AdaptUI set of rules. Divided into three different groups
(considering that the adaptation process is understood as a three step procedure)
these rules aim to finally generate a user interface adaptation under the following
steps:

\begin{itemize}
  \item Firstly, a classification and normalization of the collected knowledge of
  the main entities is needed. This might be low level knowledge. Thus, a higher
  level information is obtained during this process. To reach this goal, the
  pre-adaptation rules set is needed. These rules affect the \textit{UserAux},
  \textit{ContextAux} and \textit{DeviceAux} classes.
  
  \item Next, once the knowledge has been normalized, the adaptation rules set
  are triggered, generating the corresponding changes in the \textit{Adaptation}
  class in the AdaptUIOnt ontology.
  
  \item Finally, as the adaptation might be improvable, several rules are executed
  regarding a refinement of the last adapted user interface. This is detailed
  in Section~\ref{sec:adaptation_polisher}.
\end{itemize}