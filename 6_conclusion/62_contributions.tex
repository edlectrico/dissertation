\section{Contributions}
\label{sec:contributions}

A summary of the different contributions described in this thesis is presented 
in this section: 

\begin{enumerate}[label=\alph*)]
  \item In Chapter~\ref{cha:state_of_the_art} an in-depth analysis of the state 
  of the art is presented:
  \begin{itemize}
    \item Evaluating several approaches to modelling and reasoning over user,
    context and device.
    \item Analysing several adaptive and adaptable solutions in the field of
    user interfaces.
    \item Studying these solutions architectures and possibilities considering
    nowadays technology.
    \item This covers the objective 1 detailed in Section~\ref{sec:hypothesis}.
  \end{itemize}
  
  \item In Chapter~\ref{cha:ontology_model} an ontology for modelling users, 
  context, devices and the corresponding user interface adaptations is presented.
  \begin{itemize}
    \item Allowing the inclusion of adaptation rules to manage the adaptation 
    process.
    \item Avoiding the explicit user physiological capabilities modelling.
    \item Providing a dynamic design to allow the platform to dynamically update
    the corresponding entity.
    \item The results of this chapter accomplishes the goal specified in the
    objectives 2 and 3 in Section~\ref{sec:hypothesis}.
  \end{itemize}

  \item Chapter~\ref{cha:architecture} presents an Android compatible mobile 
  reasoning engine based on Pellet, and the corresponding architecture
  for enabling the dynamic user interface adaptation supporting semantics.
  \begin{itemize}
    \item Providing a full Pellet port available for Android devices.
    \item Allowing the use of semantics and \acp{swrl} compatible reasoning.
    \item Describing the different modules which power the whole system.
    \item Including the decisions consequences and their conclusions.
    \item This covers all the objectives mentioned in Section~\ref{sec:hypothesis},
    as it provides the necessary infrastructure to build the whole adaptation
    platform.
  \end{itemize}

%   \item Chapter~\ref{cha:architecture} introduces the corresponding architecture
%   for enabling the dynamic user interface adaptation supporting semantics.
%   \begin{itemize}
%     \item Describing the different modules which power the whole system.
%     \item Including the decisions consequences and their conclusions.
%     \item This covers all the objectives mentioned in Section~\ref{sec:hypothesis},
%     as it provides the necessary infrastructure to build the whole adaptation
%     platform.
%   \end{itemize}

  \item Finally the previous contributions where combined to offer an 
  implementation of a dynamic user interface adaptation system that provides a 
  more practical and complete vision of the user interaction capabilities.
  
\end{enumerate}