\section{Context Modelling}
\label{sec:context}
%------------------------------------------------------------------------- 
\subsection{What is Context?}
\label{sec:context_definition}

% Context is mostly defined by the definition by Dey as follows:
Context is often defined according to \citeauthor{dey_understanding_2001}
issued definition:

\begin{description}
  \item[\Defi{Context (I), by~\citet{dey_understanding_2001}}] \hfill \\
  \begin{mdframed}[hidealllines=true,backgroundcolor=gray!20]
  \textit{`Context is any information that can be used to characterize the situation
  of an entity. An entity is a person, place, or object that is considered 
  relevant to the interaction between a user and an application, including the 
  user and applications themselves``}.
  \end{mdframed}
\end{description}

In the past decades there were many definitions of context~\citep{adomavicius_context_aware_2011}.
Nevertheless, the one stated by~\citet{dey_understanding_2001} is one of the most
popular and extended definitions. \citeauthor{dey_understanding_2001}'s 
definition enables developers to easily enumerate those elements which take part 
in the context for a certain application domain. \citeauthor{dey_understanding_2001} stated that:

\begin{description}
  \item[\Defi{Context (II), by~\citet{dey_understanding_2001}}] \hfill \\
  \begin{mdframed}[hidealllines=true,backgroundcolor=gray!20]
  \textit{``If a piece of information can be used to characterize the situation 
  of a participant in an interaction, then that information is context''}.
  \end{mdframed}
\end{description}

A proposed example explains this definition: \textit{``Take the canonical 
context-aware application, an indoor mobile tour guide. The obvious entities in 
this example are the user, the application and the tour sites. We will look at 
two pieces of information – weather and the presence of other people – and use 
the definition to determine whether either one is context. The weather does not 
affect the application because it is being used indoors. Therefore, it is not 
context. The presence of other people, however, can be used to characterize the 
user's situation. If a user is travelling with other people, then the sites they 
visit may be of particular interest to her. Hence the presence of other people 
is context because it can be used to characterize the user's situation.''~\citep{dey_understanding_2001}}

By this example, it is seen how different pieces of information are analysed to
determine if they belong to what \citeauthor{dey_understanding_2001} states 
context is. These definitions are based on research experience. 
Section~\ref{sec:context_models} shows how modelling and defining context has 
evolved in the past 20 years. 

% Context management allows us to identify the conditions of the environment. This
% way, developers are able to adapt services or applications for the user taking
% into account these conditions. To do this first there is the need of gathering
% context information. Next this information has to be somehow processed and,
% finally, it will be used to personalize and contextualize the current situation.

% ----------------------------------------------------------------------

\subsection{A Chronological Review of the Evolution of Context Management}
\label{sec:chronological_review}
% \dots
Figure~\ref{fig:context_models} shows the evolution for context modelling by 
chronological order for the last 15 years. 

\vspace{1cm}
\setlength\taskwidth{1.9cm}

\begin{timeline}
  \label{chr:context}
    \Task[2000]{\citet{chen_survey_2000}}
    \Task[2001]{\citet{jameson_modelling_2001}}
    \Task[2002]{\citet{henricksen_modeling_2002}, \citet{held_modeling_2002}}
    \Task[2004]{\citet{gu_toward_2004}}
    \Task[2005]{\citet{chen_using_2005}, \citet{yamabe_citron_2005}}
    \Task[2008]{\citet{wood_context_aware_2008}}
    \Task[2011]{\citet{baltrunas_incarmusic_2011}}
    \Task[2012]{\citet{mcavoy_ontology_based_2012}}
    \Task[2013]{\citet{almeida_assessing_2012}}
\end{timeline}
\captionof{figure}{The chronological view of the evolution of remarkable context
models considered in this dissertation.\label{fig:context_models}}


\subsection{Context Models}
\label{sec:context_models}

The first definition of \textit{context-aware systems} is given by~\citet{schilit_disseminating_1994}.
\citet{dey_understanding_2001} also defines \textit{context-aware systems} as
those systems which, using context data, provide significant information and/or
services to the user where the relevancy of the given information depends on the
user task.~\citet{schmidt_there_1999} consider several issues about context
modelling. Authors emphasize the excess of abstraction about context-aware systems
and environments which causes a lack of models to be compared . Therefore they
present a working model for context-aware systems categorized into human and
physical environment factors. In 2001~\citet{jameson_modelling_2001} studies how
context-aware computing represents a challenging frontier for researchers. In
this work, information about the environment, the user current state, longer
term user properties and the user behaviour are compared in order to take the
correct adaptation decision. Several works focused on user interface adaptation
area base their processes on context changes as triggers. However, they lack of
a common model of context in their platforms~\citep{calvary_plasticity_2002}
~\citep{nilsson_model_based_2006}.
% This fact emphasizes the argument established by Schmidt et al. \cite{schmidt_there_1999}.

In the following subsections a review of the most popular context-aware systems
is presented, mainly focusing on the modelled context parameters, techniques, 
domains and dependencies. Besides, several authors define context and 
context-awareness through their own experience. All these models and definitions 
have been considered for the context model proposed in 
Chapter~\ref{cha:ontology_model}. In Section~\ref{sec:context_model_comparison} 
Table~\ref{tbl:context_comparison} shows several significant features of each 
approach, and a comparing analysis is performed.


\subsubsection{2000: Chen and Kotz}
%\cite{chen_survey_2000}
\label{sec:chen}

\citet{chen_survey_2000} define context as follows: 

\begin{description}
  \item[\Defi{Context, by~\citet{chen_survey_2000}}] \hfill \\
  \begin{mdframed}[hidealllines=true,backgroundcolor=gray!20]
  \textit{``Context is the set of environmental states and settings that either 
  determines an application’s behaviour or in which an application event occurs 
  and is interesting to the user''}~\citep{chen_survey_2000}.
  \end{mdframed}
\end{description}

Besides, the definition of active and passive context-aware computing is also 
given. To \citeauthor{chen_survey_2000}, active context awareness is:

\begin{description}
  \item[\Defi{Active Context Awareness}] \hfill \\
  \begin{mdframed}[hidealllines=true,backgroundcolor=gray!20]
  \textit{``(\dots) an application automatically adapts to discovered context, by the 
  changing in the application's behaviour''}~\citep{chen_survey_2000}.
  \end{mdframed}
\end{description}

On the contrary, they define passive context awareness as:

\begin{description}
  \item[\Defi{Passive Context Awareness}] \hfill \\
  \begin{mdframed}[hidealllines=true,backgroundcolor=gray!20]
  \textit{``(\dots) an application presents the new or updated context to an interested
  user or makes the context persistent for the user to retrieve later''}~\citep{chen_survey_2000}.
  \end{mdframed}
\end{description}

Based on the work by~\citet{schilit_context_aware_1994},~\citet{chen_survey_2000}
consider time as an important and natural context feature for many 
applications. Besides, they introduce the term \textit{context history}, which 
is an extension of the time feature recorded across a time span.

A significant problem remarked in this work is the impossibility to exchange
context information between the studied context-aware systems due to the way
they model the environment. Furthermore, as location is one of the most 
modelled context features, \citeauthor{chen_survey_2000} provide a study of 
several aspects that should be taken into account when researchers face the 
problem of modelling it. 

% \begin{table}[htbp]
% \caption{Data structures \cite{chen_survey_2000}}
% \label{tbl:chen}
% \begin{tabular}{ll}
% Technique & Description  \\
% \hline
% Key-value pairs & Used by Schilit et al. \cite{schilit_context_aware_1994}, \\
%  & an environmental variable acts as the key while the \\
%  & actual context data is the value. \\
% Tagged encoding & Used by P.J. Brown \cite{brown_stick-e_1995}, the \\
%  & contexts are modeled as tags and corresponding \\ 
%  & fields. \\
% Object-oriented model & The systems that use this technique usually \\
%  & consider the contextual information as the states \\ 
%  & of the object and the object provides methods to \\ 
%  & access and modify these states \\
% Logic-based model & Context data can be expressed as facts in a \\
%  & rule-based system. \\
% % Others & Lightning \\
% \end{tabular} 
% \end{table}


% ----------------------------------------------------------------------


\subsubsection{2001: Anthony Jameson}
%\cite{jameson_modelling_2001}
\label{sec:jameson}

\citet{jameson_modelling_2001} analyses in 2001 how context-aware computing 
represents a challenging frontier for researchers in the field of \acp{aui}. 
In this work information about the environment, the user's current 
state, longer term user properties and the user behaviour are compared in order 
to take the correct adaptation decision. The idea is to compare several 
scenarios. In the first one, only information about the environment is 
considered. In the following scenarios the user's current state, behaviour and 
long-term properties are taken into account. Thus, the results conclude 
that considering a wider range of user information can help context-aware 
systems designers.

% Table~\ref{tbl:jameson} shows the context parameters modeled by Jameson:
% 
% \begin{table}[htbp]
% \caption{Modeled context parameters \cite{jameson_modelling_2001}}
% \label{tbl:jameson}
% \begin{tabular}{ll}
% Scenario & Modeled parameters  \\
% \hline
% Using only information about & Location (GPS or similar readings) \\
% the environment & \\
% 
% Adding the user's current state & Emotional arousal \\
% Adding the user's behavior & Cognitive load, \\
%  & Current Goal \\
% Adding long-term user properties & Personal characteristics, \\
%  & Knowledge, Interests, Noncognitive Abilities \\
% 
% \end{tabular} 
% \end{table}

% ----------------------------------------------------------------------


\subsubsection{2002: Henricksen et al.}
\label{sec:henricksen}
% MENCIONAR TAMBIÉN SU TRABAJO EN 2003.

The approach followed by~\citet{henricksen_modeling_2002} makes the following
notes about context in pervasive environments:

\begin{itemize}
  \item Context information exhibits \textit{temporal characteristics}. Context 
  can be  static (e.g., birthday) or dynamic (e.g., user location). Besides, the 
  persistence of the dynamic information can easily change. Thus, it is justified 
  that the static context should be provided by the user, while the dynamic 
  one should be gathered by sensors. Past historic and a possible forecasting 
  of future context are also taken into account as part of the description of 
  the whole context description.
  
  \item A second property of the context information is its \textit{imperfection}. 
  Information can be useless if it cannot reflect a real world state. It also 
  can be inconsistent if it contains contradictions, or incomplete if some 
  context aspect are unknown. There are many causes to these situations. For 
  example, information can change so fast that it may be invalid once it is 
  collected. This is obviously because the dynamic nature of the environment. 
  Besides, there is a strong dependency on software and hardware infrastructures, 
  which can fail any time.
  
  \item Context has \textit{multiple alternative representations}. Context 
  information usually comes from sensors which speak different languages. For 
  example, a location sensor can use latitude and longitude physical magnitudes 
  while the involved application works with a logical representation of location. 
  
  \item \textit{Context information} is highly \textit{disassociated}. There are 
  obvious connections between some context aspects (e.g., users and devices). 
  However, other connections need to be computed with the available information.
\end{itemize}

% \The Figure~\ref{fig:henricksen} shows the annotated context model designed by Henricksen et al.
This work also indicates the dependency between context models, the scenarios 
and use cases of the application domains. Authors extract several context 
parameters to consider:

\begin{itemize}
  \item User activity, distinguishing between the current one and the planned one.
  \item Device that is being used by the user.
  \item Available devices and resources.
  \item Current relationships between people.
  \item Available communication channels.
\end{itemize}

% \InsertFig{henricksen}{fig:henricksen}{Context model annotated with quality parameters and metrics \cite{henricksen_modeling_2002}}{}{0.70}{}

% 
% \subsubsection{2003, Henricksen et al.}
% %\citep{henricksen_generating_2003}
% \label{sec:henricksen}
% 
% Henricksen et al. discuss about the difficulties of constructing context-aware
% applications. Besides, they remark the lack of formality and expressiveness of
% previous context-aware solutions and models. For example, several specific context
% information, as histories, uncertainty, incompleteness, sensor-derived information
% and different kind of dependencies between the information are barely conceptually 
% modeled with ER or UML approaches, which are not well suited for this task \citep{henricksen_generating_2003}.
% Therefore, they present a context modeling approach which allows developers to describe high level context
% information. In addition, a mapping process for transforming high-level context
% models to management systems is also described. This way, they characterize the
% Object-Role Modeling approach to support specific context information based
% on abstraction concepts and quality annotations. 
% 
% Authors classify context into static or dynamic facts\dots
\subsubsection{2002: Held et al.}
% \citep{held_modeling_2002}
\label{sec:held}

In 2002~\citet{held_modeling_2002} discussed about the necessity of content 
adaptation and representation formats for context information in dynamic 
environments. A significant perspective is the justification of modelling not 
only the user but the network status and device context information as well.
The following parameters are considered as relevant context information:

\begin{itemize}
  \item \textit{Device}: basic hardware features (e.g., \acs{cpu} power, memory 
  and so forth), user interface input (e.g., keyboard and voice recognition), 
  output (e.g., display and audio), and other particular specifications of the 
  device (e.g., display resolution or colour capability).
  \item \textit{User}: service selection preferences, content preferences and 
  specific information about the user.
  \item \textit{Network connection}: bandwidth, delay, and bit error rate. 
\end{itemize}

Authors also present several requirements concerning the representation of
context information. Accordingly, they defend that a context profile should be:

\begin{itemize}
  \item Structured, to ease the management of the amount of gathered information
  and remark relevant data about the context.
  \item Interchangeable, for components to interchange context profiles.
  \item Composable/decomposable, to maintain context profiles in a distributed way.
  \item Uniform, to ease the interpretation of the information.
  \item Extensible, for supporting future new attributes.
  \item Standardized, for context profile exchanges between different entities
  in the system.
\end{itemize}
\subsubsection{2004: Gu et al.: The \ac{socam} Ontology}
\label{sec:gu}

In 2004~\citet{gu_toward_2004} design \ac{socam} architecture for designing and 
prototyping applications in an \ac{ie}. Built on top of the 
\ac{osgi}\footnote{www.osgi.org} architecture, such middleware consisted of the 
following components:

\begin{itemize}
  \item Context providers, which abstract context information from heterogeneous
  sources and semantically annotate it according to the defined ontology.
  \item The context interpreter, which provides logic reasoning to process
  information about the context.
  \item The context database, which stores current and past context instance data.
  \item Context-aware applications, which adapt their behaviour according to the
  current context situation.
  \item The service-locating service, which allows context providers and 
  interpreters to advertise their presence for users and applications to locate 
  them.
\end{itemize}

\citeauthor{gu_toward_2004} use \ac{owl} to describe their context ontologies 
in order to support several tasks in \ac{socam}. As the pervasive computing 
domain can be divided into smaller sub-domains, the authors also divided the 
designed ontology into two categories: 

\begin{itemize}
  \item An \textit{upper ontology}, which captures high-level and general 
  context knowledge about the physical environment.
  \item A \textit{low-level} ontology, which is related to each sub-domain and 
  can be plugged and unplugged from the upper ontology when the context changes.
\end{itemize}

As a result, the upper ontology considers person, location, computational entity
and activity as context concepts.

\citet{gu_ontology_based_2004} also present an \ac{owl} based model to represent, 
manipulate and access context information in smart environments. The model 
represents contexts and their classification, dependency and quality of
information using \ac{owl} to support semantic interoperability, contextual 
information sharing, and context reasoning. The ontology allows to associate 
entities' properties with quality restrictions that indicate the contextual 
information quality. 
\subsubsection{2005: Chen et al.: The \ac{cobra} Ontology}
\label{sec:gu}

Another work under a similar approach is the one performed by~\citet{chen_using_2005}.
Authors introduce the \ac{cobra} ontology based system, which provides a set of 
semantic concepts for characterizing entities such as people, places or other 
objects within any context. The system provides support for context-aware 
platforms in runtime, specifically for Intelligent Meeting Rooms. The context 
broker is the central element of the architecture. This broker maintains and 
manages a shared context model between agents (applications, services, web 
services, and so forth) within the community. In intelligent environments participating
agents often have limited resources and capabilities for managing, reasoning and
sharing context. The broker's role is to help these agents to reason about the
context and share its knowledge. The presented ontology relies on:

\begin{itemize}
  \item Concepts that define physical places and their spatial connections.
  \item Concepts that define agents (humans and not humans).
  \item Concepts that define the location of an agent.
  \item Concepts that describe an agent activity.
\end{itemize}
\subsubsection{2005: Yamabe et al.: The Citron Framework}
%\cite{yamabe_citron:_2005}
\label{sec:yamabe}

In 2005~\citet{yamabe_citron_2005} present a framework for personal devices 
which gathers context information about the user and his/her surrounding 
environment. Muffin, a personal sensor-equipped device is designed. Using it, 
several context parameters are gathered.

Sensor information is also considered for evaluating several user high-level 
context information. For example, accelerometer readings might recognize a 
walking or running activity, shaking and rotating, an so forth. The microphone 
is not only used to measure the ambient noise. It is also useful for detecting 
the place where the user is (e.g., meeting room, restaurant, street and so on).

The context acquisition is categorized in two different groups: the user and 
the environment. For the user several issues are analysed. For example, 
activity recognition requires the user to use the device in specific ways (it 
is not the same to use it with hands or waist-mounted). Another problem they 
encounter is about the time consuming process since an event is captured, then 
processed and finally validated. The last contextual issue deals with the 
intrinsic complexity and ambiguity of context information. For example, the 
meaning of what is loud might depend on the current situation (e.g., if the 
user is in a meeting room, if the user is in the street, and so on). For the 
environment, Muffin suffer several heat problems due to the sensors sensitivity
to environment temperature. This way, sometimes the gathered measures are invalid.

% \begin{table}
% \caption{Modeled context parameters \cite{yamabe_citron:_2005}}
% \label{tbl:yamabe}
% \begin{tabular}{ll}
% Sensor category & Modeled parameters  \\
% \hline
% Environmental sensors & Air temperature \\
%  & Relative humidity \\ 
%  & Barometer \\
%  
% Physiological sensors & Alcohol gas \\
%  & Pulse \\
%  & Skin temperature \\
%  & Skin resistance \\
% 
% Motion/location sensors & Compass/tilt \\
%  & 3D accelerometer \\
%  & Grip \\
%  & Ultrasonic range finder \\
%  & GPS \\
% 
% Other sensors & RFID tags \\
%  & Front rear images \\
%  & Sound \\
% \end{tabular} 
% \end{table}
\subsubsection{2008: Wood et al. and the AlarmNet system}
% \cite{wood_context-aware_2008}
\label{sec:wood}

AlarmNet is an \ac{aal} \ac{wsn} for pervasive adaptive healthcare in assisted 
living residences for patients with special needs. In this 
work,~\citet{wood_context_aware_2008} contribute with several novelties:

\begin{itemize}
  \item An extensible heterogeneous network middleware.
  \item Novel context-aware protocols.
  \item SenQ, a query protocol for efficiently streaming sensor data.
\end{itemize}

The context-aware protocol uses a two-way network information flow. On the one 
hand, environmental, system and residents data flow. On the other hand, \ac{car} 
analysis goes back into the system to enable smart power management and dynamic 
alerts.

Several sensors are used for sensing environmental quality: light, temperature,
dust, resident activities (motion sensors) and so on. The devices' queries to the 
system are negotiated by the Query Manager.
\subsubsection{2011: Baltrunas et al.: InCarMusic}
%\cite{baltrunas_incarmusic_2011}
\label{sec:baltrunas}

Assuming that context-aware systems adapt to the specific user situation,
\citet{baltrunas_incarmusic_2011} present a music recommendation system which
takes into account the user's mood and the traffic conditions. To this
end, the authors design a methodology where users can be requested to judge several
contextual factors (e.g., if current traffic conditions are relevant for a decision
making task) and to rate an item when a certain contextual condition is met. 

In order to take into account the user's music preference and the influence that
context might have into them, context is modelled as several independent factors.
% (see Table~\ref{tbl:baltrunas}).
% 
% \begin{table}
% \caption{Modeled context parameters \cite{baltrunas_incarmusic_2011}}
% \label{tbl:baltrunas}
% \begin{tabular}{ll}
% Contextual factor & Contextual conditions  \\
% \hline
% Driving style & Relaxed driving \\
%  & Sport driving \\
% Road type & City \\
%  & Highway \\
%  & Serpentine \\
% Landscape & Coast line \\
%  & Country side \\
%  & Mountains/hills \\
%  & Urban \\
% Sleepiness & Awake \\
%  & Sleepy \\
% Traffic conditions & Free road \\
%  & Many cars \\
%  & Traffic jam \\
% Mood & Active \\
%  & Happy \\
%  & Lazy \\
%  & Sad \\
% Weather & Cloudy \\
%  & Snowing \\
%  & Sunny \\
%  & Rainy \\
% Natural phenomena & Day time \\
%  & Morning \\
%  & Night \\
%  & Afternoon \\
%   
% \end{tabular} 
% \end{table}
\subsubsection{2012: McAvoy et al.}
\label{sec:mcavoy}

\citeauthor{mcavoy_ontology_based_2012} propose an ontology-based system for the 
managing of context within smart environments. One of the most significant 
contributions deals with the high-level information managed through the metadata 
and meaning which are collected by the sensor network~\citep{mcavoy_ontology_based_2012}. 
The sensing devices within the smart environment have to be modelled in order 
to be semantically enriched. Data is formally represented as an ontology by 
using entities and the relationships which link them together. After the data 
is collected from the sensors it is passed to an enrichment module where it is 
made semantically rich. This new enriched data is stored in a semantic 
repository in the form of triples. The meaning of this information and the 
metadata are added to the data within this enrichment component. 

\subsubsection{2013: Almeida and López-de-Ipiña: The AMBI2ONT Ontology}
\label{sec:almeida}

\citet{almeida_assessing_2012} consider two common problems dealing with 
ambiguity in the area of context modelling: the uncertainty and the vagueness. 
Uncertainty models the likeness of a certain fact, while the vagueness 
represents the degree of membership to a fuzzy set. The uncertainty is 
represented by a certainty factor.

Due to the nature of the process of collecting data from the environment, the
proposed ontology has been designed to support two types of uncertainty:

\begin{itemize}
  \item Uncertain data: This kind of uncertainty is generated from the capture
  of data from sensors due to the imperfect nature of the devices.
  \item Uncertain rules: It occurs in the execution of the rules. 
\end{itemize}

To reason over the ambiguous information the JFuzzy Logic Open Source fuzzy
reasoner has been adapted to support uncertainty information.



\subsubsection{Context Models Comparison}
\label{sec:context_model_comparison}

The previous section has reviewed several significant context-aware systems and 
the followed approaches for modelling relevant context parameters of the environment
depending on the application domain. In fact, this is one of the main problems in
context modelling: the lack of model independence from similar domains and, also,
the lack of models to be compared. Despite the fact that sometimes the primary
domains are similar (context-aware computing, pervasive environments and ubiquitous
computing) regarding the necessity of managing context knowledge, the concrete
applications and approaches' domains are different. 
Here,~\citet{henricksen_modeling_2002} realize about the lack of formality and 
expressiveness of previous context models.

However, to avoid this problem~\citet{gu_ontology_based_2004} present an
ontology-based solution in which context information is modelled in two separated
layers:

\begin{itemize}
  \item In the upper layer there is an ontology which describes high-level 
  knowledge about the current context and physical environment.
  \item Under it there is the possibility to add and remove ontologies which
  model low-level information of the current specific domain.
\end{itemize}

On the one hand, this approach allows developers to consider richer information,
as activities, and abstract knowledge about the current global context. On the
other hand, it makes possible to model specific knowledge of the current sub-domain.
Besides, the possibility to plug and unplug these low-level ontologies makes this
solution powerful. The solution provided by~\citet{gu_toward_2004} is significant
because of the following reasons:

\begin{itemize}
  \item It considers high-level information on top of a more specific and domain
  dependent sub-model. 
  \item Activities are modelled as a relevant concept for context in the upper
  ontology.
  \item The modelled entities are related. Persons are associated with locations
  and activities, each location is linked with indoor or outdoor entities, 
  activities can be categorized into scheduled or deduced ones, and so forth. 
\end{itemize}

\citet{chen_survey_2000} survey several context-aware computing solutions. The 
remarked context characteristics (\textit{location} and \textit{time}) are given 
more as an advice, since they do not provide a reference model. Nevertheless, 
they introduce several novelties like the term \textit{context history}, which 
might be useful for future predictions about user's behaviour and trends.

Modelling high-level information allows to perform deeper computations taking
into account behavioural characteristics, trends information, inferred knowledge
from small pieces of information combinations, and so on. As can be seen in
Table~\ref{tbl:context_comparison} many authors work with high-level data in 
their context-aware systems. 

On the other hand, physical context parameters are frequently modelled in the
analysed literature. \textit{Location}, \textit{time} and \textit{environment 
conditions} (e.g., temperature, pressure, light and noise) are usually modelled 
to achieve final system's goal (e.g., adapting the user interface or recommending 
items or services). Besides, several approaches take user related characteristics 
to fulfil their purposes. For example,~\citet{schmidt_there_1999} consider not 
only physical environment as context information but also several human factors 
categorized as follows:

\begin{itemize}
  \item User information, which gathers knowledge about user's habits, emotional
  stated and bio-physiological conditions.
  \item Social environment, made up by other user's locations, social interactions
  of the current user and knowledge about the behaviour of groups of people.
  \item Task, taking into account spontaneous activities, engaged tasks and 
  general goals.
\end{itemize}

A user modelling approach which considers emotional issues of the user is 
presented by~\citet{pereira_triple_2005}. The author distinguishes between 
emotions and perceptions to separate both concepts for modelling processes. 
These approaches are found useful for recommending systems in which user mood 
and psychological state are relevant to filter multimedia content 
\citep{baltrunas_incarmusic_2011}. Following a similar perspective~\citet{evers_achieving_2012}
analyse the user's participation in adaptive applications. Several concepts that
should be taken into account regarding the user behaviour and interaction with
the application are presented. Another similar approach regarding the user's more 
psychological aspects is presented by~\citet{liao_decision_2005}. Here the 
authors present a model for modelling user stress levels. This is related to the 
concept of Considerate Computing, term that it is explained in~\citep{gibbs_considerate_2005}.

\citet{schmidt_there_1999} also remark the social environment as relevant for 
context modelling. More related to recommending environments, non explicit 
information about user's likes need to be computed. Similar approaches in these 
environments tend to avoid recommending systems intrinsic problems, like the 
so-called \textit{cold start problem}~\citep{castillejo_alleviating_2012}\citep{castillejo_social_2012}.

Another relevant point remarked by~\citet{schmidt_there_1999} are the user's tasks.
Several authors consider activities relevant for modelling context 
\citep{henricksen_modeling_2002}\citep{gu_toward_2004}\citep{abowd_towards_1999}.
Activities enrich context information about the user~\citep{razmerita_ontology_based_2003}.
It is common to model user activities in the user model (see Table~\ref{tbl:user_comparison}). 
% However,
% taking into account that these activities are identified by sensors and with the
% interaction with the environment, the proposed model for context considers them
% as a context parameter. In addition, the followed perspective takes into account
% activities as an abstract term. The context model in this dissertation does not
% model individual activities but those situations in which the user can not achieve
% the final goal. This is explained with more detail in Section~\ref{sec:stressful_activities}.

Finally, sometimes the collected data might lead to misunderstandings or 
non-trustworthy data.~\citet{almeida_assessing_2012} consider \textit{ambiguity}
and \textit{uncertainty} in their work and present an ontology-based process 
which allows to model them within a smart environment.

\bigskip

Table~\ref{tbl:context_comparison} shows the differences between each reviewed
solution for modelling context. It is difficult to gather all the approaches in a
unique table. Therefore, Table~\ref{tbl:context_comparison} just emphasizes several
differences, although these solutions have more characteristics modelled. For example,
the work by~\citet{almeida_assessing_2012} is focused on modelling context uncertainty
and vagueness due to the nature of context data from sensors.

Besides and as it is deeply discussed by~\citet{strang_context_2004}, modelling 
the context with ontologies offers the following advantages:
\begin{itemize}
  \item Ontologies are the most expressive approach to model context.
  \item Composition and management of the model can be done in a distributed
  manner.
  \item It is possible to partially validate the contextual knowledge.
  \item One of the main strengths of ontologies is the simplicity to enact the
  normalization and formality of the model.
\end{itemize}

Regarding the context-aware systems, we cannot go further without remarking the 
issues of gathering information from different sources or sensors. This 
information might sometimes be unreliable. Sensors can fail in the collecting 
process, they also can stop working due to several reasons (e.g., power or 
malfunction). What is more, every sensor speaks its language (this issue is 
addressed applying different techniques, like data fusion). Therefore, a process 
to evaluate the quality and trustworthiness of the collected information should 
be included in every context-based system as a method to avoid undesired results.

\begin{table}
 \caption{Related work for context modelling approaches.}
 \label{tbl:context_comparison}
 \footnotesize
 \centering

\begin{tabular}{l c c c c c c c c c c c }
  \hline 
 \textbf{Solution} & \textbf{Ontologies} & \multicolumn{10}{c}{\textbf{Context parameters}} \\
 
 \textbf{(Domain)} & & \textbf{L} & \textbf{T} & \textbf{A} & \textbf{R} & \textbf{P} & \textbf{E} & \textbf{S} & \textbf{I} & \textbf{U} & \textbf{H}\\
    \hline 
        
    2000,~\citet{chen_survey_2000} 		&  & $\checkmark$ & $\checkmark$ &  &  &  &  &  &  &  &  \\
    (Context awareness)\\
    2001,~\citet{jameson_modelling_2001} 	&  & $\checkmark$ &  & $\checkmark$ &  &  &  &  &  & $\checkmark$ & $\checkmark$ \\
    (Context modelling)\\
    2002,~\citet{henricksen_modeling_2002} 	&   &  &  & $\checkmark$ & $\checkmark$ &  &  & $\checkmark$ & $\checkmark$ &  & $\checkmark$ \\
    (Pervasive computing)\\
    2002,~\citet{held_modeling_2002} 		&  &  &  &  &  &  &  &  & $\checkmark$ & $\checkmark$  &  \\
    (Context awareness)\\
    2004,~\citet{gu_toward_2004} 		& $\checkmark$ &  &  &  &  &  &  &  &  &  & $\checkmark$ \\
    (Smart environments)\\
    2005,~\citet{chen_using_2005} 		& $\checkmark$ & $\checkmark$ & $\checkmark$ & $\checkmark$ & $\checkmark$ & $\checkmark$ & $\checkmark$ &  & $\checkmark$ & $\checkmark$ & \\
    (Pervasive computing)\\
    2005,~\citet{yamabe_citron_2005} 		&  & $\checkmark$ &  & $\checkmark$ &  &  & $\checkmark$ &  &  & $\checkmark$ & $\checkmark$ \\
    (Mobile computing)\\
    2008,~\citet{wood_context_aware_2008} 	&  &  &  & $\checkmark$ &  &  & $\checkmark$ &  &  & $\checkmark$ &  \\
    (\ac{aal})\\
    2011,~\citet{baltrunas_incarmusic_2011} 	&  &  &  &  &  &  & $\checkmark$ &  &  & $\checkmark$ &  \\ 
    (Recommender systems)\\
    2012,~\citet{mcavoy_ontology_based_2012} 	& $\checkmark$ &  &  &  &  & $\checkmark$ &  &  &  &  & $\checkmark$  \\
    (Smart environments)\\
    2013,Almeida and	& $\checkmark$ & $\checkmark$ &  &  &  &  & $\checkmark$ &  &  &  &  \\
    López-de-Ipiña~\citep{almeida_assessing_2012}\\
    (Smart environments)\\
\hline

\end{tabular}
\end{table}

\subsubsection{Modelling Techniques}
\label{sec:modelling_techniques}

\citet{strang_context_2004} reviewed the context modelling literature in 2004.
They differentiate among several paradigms. Here we discuss several previously 
used context modelling techniques.


\myparagraph{Key-value}

\citet{maass_location_aware_1998} adopted a X.500 based solution to store location 
data. This approach is also used in distributed searching systems. Although it 
is very easy to maintain and handle its main problem is that it makes difficult 
to build complex structures~\citep{strang_context_2004}. Similar solutions follow 
this key-value approach to identify a context element (key), like location, with 
an environment variable (value)~\citep{schilit_customizing_1993}~\citep{voelker_mobisaic_1996}.

\myparagraph{Markup scheme}

Based on several derivative \ac{sgml}, for example \ac{xml}, marking schemes 
based models are widespread for modelling profiles. Some extensions are defined 
as \ac{ccpp} \footnote{http://www.w3.org/Mobile/CCPP/} standards and 
\ac{uaprof} \footnote{http://www.mobilemultimedia.be/en/uaprof/}. This kind of 
context modelling usually extends and completes the \ac{ccpp} and \ac{uaprof} 
basic vocabularies. In~\citep{held_modeling_2002} authors present an extension 
of this model, \ac{cscp}, which provides hierarchy to such schemes supporting 
the \ac{rdf} flexibility to express natural structures of profile information.

\myparagraph{Graphic models}

While~\citet{bauer_identification_2003} used a \ac{uml} tool to model the context in
a air traffic domain,~\citet{henricksen_generating_2003} presented a graphic model
(as an extension of Object-Role Modelling\footnote{http://www.orm.net/}). 
\ac{uml} is a widespread general purpose modelling tool with a very powerful 
graphic component (graphic models): the \ac{uml} diagram. 

\myparagraph{Object oriented models}

\citet{strang_context_2004} presented an object oriented model in which context
process details are embedded into object level. Data is hidden from other components.
Therefore, the access to this context data is just allowed through several interfaces.
This approach tries to use the object oriented programming benefits, as re-usability
and encapsulation, to cover ubiquitous environment's problems about context. Another
example of this approach is the one given by the GUIDE project by~\citet{cheverst_design_1999},
which is focused on location. In this case the context information is also in the
object as accessible states through those methods defined by the object itself
and by modifying these states.

% \paragraph{Logic models}

\myparagraph{Ontology based models}

As~\citet{strang_context_2004} discussed, modelling the context with
ontologies offers the following advantages:
\begin{itemize}
  \item Ontologies are the most expressive approach to model context.
  \item Composition and management of the model can be done in a distributed 
  manner.
  \item It is possible to partially validate the contextual knowledge.
  \item One of the main strengths of ontologies is the simplicity to enact 
  the normalization and formality of the model.
\end{itemize}

Multiple ontology based context models have been developed in the past. In the 
following section several ontology-based relevant models will be discussed.



% ----------------------------------------------------------------------

