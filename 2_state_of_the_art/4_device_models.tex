
\section{Device Models}
\label{sec:devices}

There are many and different approaches in the literature for devices modelling.
Devices capabilities might determine the boundaries of an adaptation process or
the consumable resources.~\citet{lemlouma_context_aware_2004} considered that
an independent device model would probably reduce adaptation process efforts for
different context situations. In fact, \ac{w3c} reinforces this perspective
by warning about the wide range of device capabilities and 
sizes~\citep{device_independence} which define the boundaries of the content
that each device can handle. Several techniques, like Device Descriptors, content
transformation guides, devices \acp{api} and \ac{ccpp} systems help developers 
to optimize the user experience. These modelling techniques are analysed in the 
following section. 

\subsection{Composite Capabilities/Preference Profiles}
\label{sec:ccpp}
\ac{ccpp}~\citep{ccpp_status} is a \ac{w3c} standard system to express user and 
device capabilities. Using \ac{ccpp} a user will be able to show a specific 
preference or disability. For example, even though a user's device can display 
millions of colours, perhaps the user can just distinguish between a small set 
of colours. The necessity of this system stems from the wide range of web and 
ubiquitous devices available in the market. These devices have more and more 
multimedia and Web capabilities. This makes troublesome to Web content providers 
to service their content to these devices keeping usability and user 
satisfaction~\citep{lemlouma_context_aware_2004}. 

However, managing a large number of devices is not a new problem. Many approaches 
have been proposed in the literature to tackle this situation. Most of them are 
based on content management. This approach considers different presentation 
alternatives. Depending on the client device, a presentation configuration is 
served. Hence, in the content serving process there are two options: the 
server chooses which is the best configuration for the device, and the client 
decides what to do with the content. This approach is very easy to implement, 
since every device identifies itself against the server.

\ac{ccpp} is based on \textit{profiles}. A profile contains components and attributes.
Each component has, at least, one attribute, and each profile has at least one
component. The main components are: the hardware platform, the software platform
and single applications (e.g., a browser). Attributes can contain one or many 
values. For example, in case of the hardware platform component we can find the 
attributes \textit{displayWidth} and \textit{displayHeight}. These attributes 
have a single value. \ac{ccpp} uses \ac{rdf} as formal language to build these 
profiles. Table~\ref{tbl:ccpp} shows several advantages and drawbacks of this 
approach.



\begin{table}
\caption{\ac{ccpp}: several advantages and drawbacks}
\label{tbl:ccpp}
\footnotesize
\centering
\begin{tabular}{l l}
  \hline
  \textbf{Advantages}				& \textbf{Drawbacks}			\\
  \hline
  A good infrastructure for modelling devices 	& Device dependent 			\\
  Content negotiation flexibility 		& It requires a more mature user 	\\
						& preferences definition 		\\
  Using \ac{ccpp}, Web based device developers	& 					\\
  and user agents can define accurate profiles 	&					\\
  for their products. Web servers and server 	&					\\
  proxies can use these profiles to perform the &					\\
  adaptation					&					\\
  Open to new protocol proposals for profile 	&					\\
  exchanging					&					\\
  \hline
\end{tabular}
\end{table}

\subsection{\ac{uaprof}}
\label{sec:uaprof}
\ac{uaprof}~\citep{uaprof} is concerned with collecting wireless devices 
capabilities and preferences. This information is provided to content servers 
to easy the content format selection process. \ac{uaprof} is directly related to 
the \ac{w3c} \ac{ccpp} specification and it is also based on \ac{rdf}. Hence, 
the document schema is extensible~\citep{butler_implementing_2001}\citep{butler_ccpp_2002}. 
These files, usually served as application/\ac{xml} mimetype, describe several mobile 
devices capabilities (e.g., vendor, model, screen size, multimedia capabilities, 
and so forth). Most recent versions have also information about \ac{mms}, video streaming 
and more multimedia features. \ac{uaprof} profiles are voluntarily built by the 
vendors (e.g., Samsung and LG for \acs{gsm} devices), or by several 
telecommunications company for \acs{cdmabrew} devices.

The system works as follows:

\begin{enumerate}
  \item The device sends a header containing a URL and its \ac{uaprof} within an 
  HTTP request.
  \item The server side analyses the received \ac{uaprof} to adapt the content 
  to the device's display size.
  \item Finally, the server takes the decision and serves the corresponding items
  to the device.
\end{enumerate}


However, this approach has several drawbacks:

\begin{itemize}
 \item Not every device has a \ac{uaprof}.
 \item Not every \ac{uaprof} profile is available.
 \item \ac{uaprof} data can contain schema or data errors.
 \item There is no industry-wide data quality standard for the data within each
 field in an \ac{uaprof}.
\end{itemize}

% Figure~\ref{fig:uaprof_profile} shows an example of a \ac{uaprof} profile.

\lstset {
%   language=xml,
  basicstyle={\footnotesize\ttfamily},
  numbers=none,
%   backgroundcolour=\colour{lightgray},
  aboveskip=3mm,
  belowskip=3mm,
  showstringspaces=false,
  columns=flexible,
  keywordstyle={\bfseries\colour{Blue}},
  commentstyle={\colour{Red}\textit},
  stringstyle=\colour{Magenta},
  frame=single,
  breaklines=true,
  breakatwhitespace=true,
  tabsize=4,
  morekeywords={rdf,rdfs,owl},  % <-- adding custom keywords
  caption={\ac{uaprof} example (source: \url{http://www.w3.org/wiki/UAProfIndex})},
  label=fig:uaprof_profile
}
% \lstinputlisting[language=XML, frame=single]{2_state_of_the_art/device/uaprof_example.xml}

\subsection{Device Description Repository}
\label{sec:ddr}
\ac{ddr} is a concept proposed by the \ac{ddwg}, an organization within the 
\ac{w3c}. The \ac{ddr} is supported by a standard interface and an initial 
vocabulary core about devices' properties. Web content authors will use these 
repositories to adapt their content to these devices. Thus, the Web content 
interaction with different devices will be easier. Screen size, input mechanisms, 
supported colour sets, known limitations, special features, and so forth are 
stored in these repositories. In the following lines two of the most popular
\ac{ddr} systems are introduced.

\subsubsection{\ac{wurfl}} 

\ac{wurfl}~\citep{wurfl} is an \ac{xml} based
open source database which contains the characteristics and capabilities of a 
wide range of devices. These capabilities are classified into several groups. 
These groups represent a simple way to understand \ac{wurfl} and its data. Its 
\ac{api} is very easy to use and it provides a hierarchy able to infer several 
capabilities for devices which are still not present in the file.

The following are several capabilities used by~\citet{almeida_imhotep_2011} for
Imhotep, a framework which aims to ease the development of accessible and adaptable
user interfaces:

\begin{itemize}
 \item \textit{display}: It contains information about the device's screen (e.g.,
 the resolution, number of columns, orientation, and so on).
 \item \textit{image\_format}: Supported image formats.
 \item \textit{playback}: Supported video codec.
 \item \textit{streaming}: Available streaming capabilities.
 \item \textit{sound\_format}: Supported audio formats.
\end{itemize}

\ac{wurfl} has become the de-facto standard for mobile capabilities. Nevertheless,
there are several free and open source alternatives that are growing within the
community very fast.

\subsubsection{OpenDDR}

OpenDDR\footnote{https://github.com/OpenDDRdotORG/OpenDDR-Resources} is an open 
\ac{ddr} alternative which also provides an \ac{api} to access \ac{ddr}s about 
devices capabilities. It has two main advantages:

\begin{itemize}
 \item The conviction that the application will work with any \ac{w3c} \ac{ddr} 
 \ac{api} implementation.
 \item Adopting the \ac{w3c} standard the Copyright of the interfaces is 
 protected by the \ac{w3c} against any intellectual property and patent claims.
\end{itemize}

Nevertheless, the OpenDDR \ac{api} is complex. It does not provide an architecture
approach like \ac{wurfl}. Thus, it assumes default values for unknown parameters
(e.g., \textit{displayWidth} $= 800$ pixels).

% \subsubsection{51Degrees}
% 
% It works similar to OpenDDR, with the difference that 51Degrees\footnote{http://51degrees.codeplex.com/}
% has two versions. The first one (Lite) is free to use and it gathers several 
% devices capabilities. The second one (Premium) is not free, but it has more data 
% about devices and automatic updates for the stored information. 


\subsubsection{\ac{ddr} Solution Comparison}
\label{sec:ddr_comparison}

The main problem with these solutions is their inability to provide all the
information developers usually need. For instance, in \ac{wurfl} many significant 
fields are empty (which means that default values are used) or with error data. 
Another disadvantage is that, for the interface adaptation domain, sometimes we 
may need dynamic information about the device. For example, the battery levels, 
or the available memory, can be crucial pieces of information before making any 
adaptation process. Table~\ref{tbl:ddr_comparison} depicts each solution's drawbacks 
and advantages, extracted from our experience with these \ac{ddr} solutions.


\begin{table}
\caption{Analysed \ac{ddr}s comparison~\citep{ddr_comparison}.}
\label{tbl:ddr_comparison}
\footnotesize
\centering
\begin{tabular}{l l l}
\hline
 \textbf{\ac{ddr}} 	& \textbf{Advantages} 			& \textbf{Drawbacks}  	\\
\hline
  \ac{wurfl} 	& Upgradeable to new versions 			& Errors in data 	\\
		& A hierarchy which allows to infer values 	& Many empty values	\\
		& Many capabilities modelled 			& 			\\
		& Very easy to configure 			& 			\\
		& Powerful \ac{api} 				&			\\
  OpenDDR 	& Free to use, even commercially 		& Limited number of 	\\
		&		 				& capabilities		\\
		& Growing community 				& Default values for 	\\
		& 						& unknown data 		\\
%   51Degrees.mobi& A Lite version, free to use 			& Limited number of 	\\
% 		& even commercially 				& capabilities		\\
% 		& Easy to install and use 			&			\\
\hline
\end{tabular}
\end{table}

\subsection{Ontologies}

There are other alternatives for modelling devices. Ontologies has the ability to
give some meaning to the modelled concepts and the collected information. From the 
reviewed ontologies we remark the \ac{soupa} ontology which, in addition to consider 
context information, it models many different aspects of static mobile capabilities 
and characteristics~\citep{chen_soupa_2004}. Another significant ontology is the 
one presented by~\citet{hervas_context_2010} as part of the \ac{pivon} ontology. 
The Device Model Ontology describes not only the device's capabilities but its 
relationships with the service and user ontologies, dependencies and other features.

\subsection{Device Discussion}
\label{sec:device_discussion}

Apart from all the mentioned techniques and solutions for considering devices and 
their capabilities, it is necessary to remark how technology evolves in a way that 
makes difficult to predict how we will work with smart devices in the near future. 
A few years ago context was understood as the result of the mixture of every 
physical sensor deployed in the environment. Temperature, light, noise and so 
forth. All these context variables have been historically collected through several 
sensors located strategically. Now these sensors (and more) are also embedded in 
these smart devices. They have become wearable, and using their capabilities of 
the devices they area embedded in they have improved their performance. Touchable 
screens, network capabilities, parallel computing resources and high storage space 
embedded in small devices have changed the way context and smart devices interact. 
This situation makes us contemplate how this interaction channel would be in the 
near future. Smart watches, glasses and other wearable devices are, undoubtedly, 
changing the way smart environments will sense, behave and react.

% 
% \subsection{Related Standardization Work}
% ANEXO!!!!
% 
% 
% Regarding the standardization, there are several initiatives (dealing with HCI)
% that, considering users and devices, are now presented. 
% 
% \myparagraph{The Web Accessibility Initiative}
% 
% The Web Accessibility Initiative (WAI)~\citep{wai} is focused
% on enabling people with disabilities to equally participate in the Web, e.g.,
% including social inclusion, regarding not only their disabilities but the location
% and available infrastructure. On the other hand, there are also different approaches
% to standardise the content~\citep{wcag} and the presentation of the user interface, i.e.,
% the Pennsylvania State University~\citep{accessibility}
% studies how to deal with \ac{w3c}'s standards, accessibility and usability issues. Besides,
% several workshops are proposed to keep investigating under this issue (see~\citep{wfsmbui}).
% 
% \myparagraph{The Uncertainty Reasoning for the World Wide Web Incubator Group}
% 
% Considering the uncertainty of the collected context information, the \ac{w3c}'s 
% Uncertainty Reasoning for the World Wide Web Incubator Group (URW3-XG)~\citep{wwwig}
% aims to define the challenge of reasoning with and representing uncertain information
% through related WWW technologies. In 2008 the \ac{w3c} Incubator Group released a report
% where there are many recommendations through different analysis to identify and describe
% not only potential uncertain situations but applicable methodologies and the fundamentals
% of a standardized representation to effectively use them.
% 
% Table~\ref{tbl:standards} shows several \ac{w3c} standards and different specifications
% considering users and devices. Besides, there are also many guidelines and discussion
% groups working on these issues.
% 
% The \ac{w3c} remarks the lack of definition that makes the interoperability of user models
% difficult. In this context, they propose the following areas to work on:
% 
% \begin{itemize}
%   \item A standard interoperability model providing \ac{api}'s for differnt purposes and applications
%   \item Common data storage format for user profiles
%   \item Common calibration/validation technique
%   \item Collaboration on ethical issues
%   \item Ensuring sustainability by making them available within a standard
%   \item Mechanisms for exchanging user profile data between sources
%   \item Protection mechanisms for privacy issues
%   \item Control mechanisms for user profile data exposure
% \end{itemize}
% 
% \begin{table}
%   \caption{\ac{w3c} standards and specifications for user and device modelling}
%  \label{tbl:standards}
% \footnotesize
% \centering
%  \begin{tabular}{l l c c }
%   \hline 
%   \textbf{Entity} & \textbf{Contribution} & \multicolumn{2}{c}{\textbf{Type}} \\
%     & & \textbf{Standard} & \textbf{Specification}\\
%   \hline
%   
% User	&	UAProf	& $\checkmark$ & \\
%  &	Multimodal Architecture and Interfaces	& $\checkmark$  & \\
%  &	Extensible MultiModal Annotation markup language & $\checkmark$  &	\\
%  &	Ink Markup Language (InkML) & $\checkmark$ & 	\\
% 
%  & Accessibility (All) &  & $\checkmark$ \\
%  & Web Content Accessibility Guidelines (WCAG) &  & $\checkmark$ \\
%  & Accessible Rich Internet Applications (WAI-ARIA) &  & $\checkmark$ \\
%  & User Agent Accessibility Guidelines (UAAG) &  & $\checkmark$ \\
%  & Authoring Tool Accessibility Guidelines (ATAG) &  & $\checkmark$ \\
%  & Evaluation and Report Language (EARL) &  & $\checkmark$ \\
%  & IndieUI &  & $\checkmark$ \\
%  
%  
% % Context	&			\\
% Device	&	\ac{ccpp}	& $\checkmark$ & 	\\
% % 	&	Device Description Working Group & & $\checkmark$ &   & \\
% 	& Introduction to Model-Based User Interfaces &  & $\checkmark$ \\
% 	& Model-Based User Interfaces Glossary &  & $\checkmark$ \\
% 	& Guidelines for writing device independent tests &  & $\checkmark$ \\
% 	& Delivery Context Overview for Device Independence &  & $\checkmark$ \\
% 	& Authoring Techniques for Device Independence &  & $\checkmark$ \\
% 	& Device Independence Principles &  & $\checkmark$ \\
% 	& Authoring Challenges for Device Independence &  & $\checkmark$ \\
% 	& Delivery Context Ontology & & $\checkmark$\\
% 	& Delivery Context: Client Interfaces (DCCI) &  & $\checkmark$\\
%   \hline
% 
% \end{tabular}
% \end{table}
% 
% \myparagraph{The Video in the Web Activity}
% 
% Promoting the use of the video in the Web, the Video in the Web Activity \citep{wva}
% aims to build a solid architecture to enable users to create, navigate, search,
% link and distribute video, effectively making video part of the Web. This activity
% group is formed by three different working groups:
% 
% \begin{itemize}
%   \item The Timed Text Working Group \citep{ttml}, which mission is to provide a language that represents
%   textual information that is associated with timing information. This group has
%   released two versions of the Timed Text Markup Language (TTML). The first one was
%   released in 2010. The last one during 2013.
%   \item The Media Annotations Working Group \citep{mawg}, which main purpose is to provide an
%   ontology and \ac{api} to facilitate cross-community data integration of multimedia
%   information in the Web. The group has published the following documents:
%     \begin{itemize}
%       \item A second version of Use Cases and Requirements for Ontology and \ac{api}
%       for Media Resource 1.0 on January 2010, as an input for the development of
%       ``the Ontology and the \ac{api} for Media Resource 1.0'' Specification.
%       \item A \ac{w3c} Recommendation version of the Ontology for Media Resource on February 2012.
%       \item The group is now about to publish a Proposed Recommendation of the
%       \ac{api} for Media Resources.
%     \end{itemize}
%   \item The Media Fragments Working Group \citep{mfwg}, which successfully aimed to address temporal and
%   spatial media fragments in the Web using URIs. The group was closed on December 2013
%   having successfully published two versions (basic and advanced) of the Media
%   Fragments URI 1.0 specification as a Recommendation.
% \end{itemize}
% 
% The Video in the Web Activity members are still working over the issue of the
% video codec to be used in the \ac{w3c} specifications (in particular HTML5).
% 
% \myparagraph{The Provenance Working Group}
% 
% Provenance is defined as the information about the involved entities in producing
% data which can be used to form assessments about its trustworthiness.
% In order to allow users to mark up web pages using the terms provided or by
% making available provenance information expressed as linked data the PROV specification \citep{prov}
% provides a vocabulary to interchange this provenance information. This specification
% consists of 11 documents that define various necessary aspects to achieve the
% vision of inter-operable interchange of provenance information in heterogeneous
% environments such as the Web (we do not consider the Overview document). Table~\ref{tbl:prov} shows the cited
% documents and their details. After the publication of the PROV
% Ontology ~\citep{provo} the PROV group was closed on June, 2013. 
% 
% \begin{table}
%   \caption{PROV documents description. Document type (Recommendation or Note) is also shown}
%  \label{tbl:prov}
% \footnotesize
% \centering
%  \begin{tabular}{l l l}
%   \hline 
%   \textbf{Document} 	& \textbf{Type} & \textbf{Details} \\
%   \hline 
% %   PROV-OVERVIEW \citep{PROV-OVERVIEW}	& Note 		& An overview of the PROV family of documents\\
%   PROV-PRIMER \citep{PROV-PRIMER}		& Note		& The entry point to PROV offering an introduction to the provenance data model\\
% %   & & provenance data model\\
%   PROV-XML \citep{PROV-XML}			& Note		&  Defines an XML schema for the provenance data model PROV data model\\ 
% %   & & This is intended for developers who need a native XML serialization of the \\
% %   & & PROV data model\\
%   PROV-O \citep{PROV-O}				& Rec. 		& PROV-O defines a light-weight OWL2 ontology for the provenance data model \\
% %   & & This is intended for the Linked Data and Semantic Web community\\
%   PROV-DM \citep{PROV-DM}			& Rec. 		& Defines a conceptual data model for provenance including UML diagrams. \\
%   & & PROV-O, PROV-XML and PROV-N are serializations of this conceptual model
%   \\
%   PROV-N \citep{PROV-N}				& Rec. 		& Defines a human-readable notation for the provenance model. \\
% %   & & This is used to provide examples within the conceptual model as well as used in \\
% %   & & the definition of PROV-CONSTRAINTS\\
%   PROV-CONSTRAINTS \citep{PROV-CONSTRAINTS}	& Rec. 		& Defines a set of constraints on the PROV data model that specifies a notion of\\
%   & & valid provenance\\
% %   It is specifically aimed at the implementors of validators\\
%   PROV-AQ \citep{PROV-AQ} 			& Note		&  Defines how to use Web-based mechanisms to locate and retrieve provenance \\
%   & & information\\
%   PROV-DC \citep{PROV-DC} 			& Note		&  Defines a mapping between Dublin Core and PROV-O\\
%   PROV-DICTIONARY \citep{PROV-DICTIONARY}	& Note		&  Defines constructs for expressing the provenance of dictionary style data structures\\
%   PROV-SEM \citep{PROV-SEM}			& Note		&  Defines a declarative specification in terms of first-order logic of the PROV data \\
%   & & model\\
%   PROV-LINKS \citep{PROV-LINKS}			& Note		&  Defines extensions to PROV to enable linking provenance information across  \\
%   & & bundles of provenance descriptions\\
%   \hline
% %   \hline
% 
% \end{tabular}
% \end{table}
% 
% \myparagraph{The Moving Picture Experts Group}
% 
% The Moving Picture Experts Group (MPEG) is a working group of ISO/IEC with the
% mission to develop standards, e.g., MPEG-ACC, MPEG-H, MPEG-DASH, MPEG-4\dots for coded
% representation of digital audio and video. The group has produced several
% standards that help the industry offer end users an ever more enjoyable digital
% media experience. As this group keeps working on the cited objectives the 108
% meeting is scheduled March 2014 in Valencia, Spain ~\citep{mpeg_valencia}. 
% 
% \myparagraph{The Internet Engineering Task Force}
% 
% The mission of the Internet Engineering Task Force (IETF) is to produce high
% quality and relevant technical documents that influence people to design, use,
% and manage the Internet. IETF's standards development work is organized into 8
% areas. Within each area there are multiple Working Groups. Each Working Group
% has one or more chairs who manage the work, and a written charter defining what
% the work is and when it is due. Table~\ref{tbl:ietf} shows the current active
% IETF Working Groups. The next meeting is scheduled for March, 2014 in London, England.
% % Tabla con las publicaciones y timelines de cada subgrupo
% 
% 
% \begin{table}
%   \caption{Active IETF Working Groups}
%  \label{tbl:ietf}
% \footnotesize
% \centering
%  \begin{tabular}{l}
%   \hline 
%   \textbf{Working Group} 	 \\
%   \hline 
%     Applications Area Working Group \\
%     Constrained RESTful Environments \\
%     Extensible Provisioning Protocol Extensions \\
%     Hypertext Transfer Protocol Bis \\
%     BiDirectional or Server-Initiated HTTP \\
%     JSON data formats for vCard and iCalendar \\
%     JavaScript Object Notation \\
%     Protocol to Access WS database \\
%     Preparation and Comparison of Internationalized Strings \\
%     IMAP QRESYNC Extension \\
%     System for Cross-domain Identity Management \\
%     SPF Update \\
%     Uniform Resource Names, Revised \\
%     Using TLS in Applications \\
%     Web Security \\
%     Web Extensible Internet Registration Data Service \\
%   \hline
% 
% 
% \end{tabular}
% \end{table}
