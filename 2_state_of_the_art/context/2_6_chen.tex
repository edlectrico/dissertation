\subsubsection{2005: Chen et al.: The \ac{cobra} Ontology}
\label{sec:gu}

Another work under a similar approach is the one performed by~\citet{chen_using_2005}.
Authors introduce the \ac{cobra} ontology based system, which provides a set of 
semantic concepts for characterizing entities such as people, places or other 
objects within any context. The system provides support for context-aware 
platforms in runtime, specifically for Intelligent Meeting Rooms. The context 
broker is the central element of the architecture. This broker maintains and 
manages a shared context model between agents (applications, services, web 
services, and so forth) within the community. In intelligent environments participating
agents often have limited resources and capabilities for managing, reasoning and
sharing context. The broker's role is to help these agents to reason about the
context and share its knowledge. The presented ontology relies on:

\begin{itemize}
  \item Concepts that define physical places and their spatial connections.
  \item Concepts that define agents (humans and not humans).
  \item Concepts that define the location of an agent.
  \item Concepts that describe an agent activity.
\end{itemize}