
\subsubsection{Modelling Techniques}
\label{sec:modelling_techniques}

\citet{strang_context_2004} reviewed the context modelling literature in 2004.
They differentiate among several paradigms. Here a discussion of several 
previously used context modelling techniques is presented:

\begin{enumerate}
  \item Key-value: \citet{maass_location_aware_1998} adopted a X.500 based 
  solution to store location data. This approach is also used in distributed 
  searching systems. Although it is very easy to maintain and handle its main 
  problem is that it makes difficult to build complex structures~\citep{strang_context_2004}. 
  Similar solutions follow this key-value approach to identify a context element 
  (key), like location, with an environment variable (value)~\citep{schilit_customizing_1993}~\citep{voelker_mobisaic_1996}. 
  
  \item Markup scheme: Based on several derivative \ac{sgml}, for example 
  \ac{xml}, marking schemes based models are widespread for modelling profiles. 
  Some extensions are defined as \ac{ccpp} \footnote{http://www.w3.org/Mobile/CCPP/} 
  standards and \ac{uaprof} \footnote{http://www.mobilemultimedia.be/en/uaprof/}. 
  This kind of context modelling usually extends and completes the \ac{ccpp} and 
  \ac{uaprof} basic vocabularies. In~\citep{held_modeling_2002} authors present 
  an extension of this model, \ac{cscp}, which provides hierarchy to such schemes 
  supporting the \ac{rdf} flexibility to express natural structures of profile 
  information.
  
  \item Graphic models: While~\citet{bauer_identification_2003} used a \ac{uml} 
  tool to model the context in a air traffic domain,~\citet{henricksen_generating_2003} 
  presented a graphic model (as an extension of Object-Role Modelling\footnote{http://www.orm.net/}). 
  \ac{uml} is a widespread general purpose modelling tool with a very powerful 
  graphic component (graphic models): the \ac{uml} diagram. 
  
  \item Object oriented models: \citet{strang_context_2004} presented an object 
  oriented model in which context process details are embedded into object level. 
  Data is hidden from other components.Therefore, the access to this context data 
  is just allowed through several interfaces. This approach tries to use the 
  object oriented programming benefits, as re-usability and encapsulation, to 
  cover ubiquitous environment's problems about context. Another example of this 
  approach is the one given by the GUIDE project by~\citet{cheverst_design_1999}, 
  which is focused on location. In this case the context information is also in 
  the object as accessible states through those methods defined by the object 
  itself and by modifying these states.
  
  \item {Ontology based models: As~\citet{strang_context_2004} discussed, modelling 
  the context with ontologies offers the following advantages:
  \begin{itemize}
    \item Ontologies are the most expressive approach to model context.
    \item Composition and management of the model can be done in a distributed 
    manner.
    \item It is possible to partially validate the contextual knowledge.
    \item One of the main strengths of ontologies is the simplicity to enact 
    the normalization and formality of the model.
  \end{itemize}
  Multiple ontology based context models have been developed in the past. In the 
  following section several ontology-based relevant models will be discussed.}
  
\end{enumerate}