
\subsubsection{Context Models Comparison}
\label{sec:context_model_comparison}

The previous section has reviewed several significant context-aware systems and 
the followed approaches for modelling relevant context parameters of the environment
depending on the application domain. In fact, this is one of the main problems in
context modelling: the lack of model independence from similar domains and, also,
the lack of models to be compared. Despite the fact that sometimes the primary
domains are similar (context-aware computing, pervasive environments and ubiquitous
computing) regarding the necessity of managing context knowledge, the concrete
applications and approaches' domains are different. 
Here,~\citet{henricksen_modeling_2002} realize about the lack of formality and 
expressiveness of previous context models.

However, to avoid this problem~\citet{gu_ontology_based_2004} present an
ontology-based solution in which context information is modelled in two separated
layers:

\begin{itemize}
  \item In the upper layer there is an ontology which describes high-level 
  knowledge about the current context and physical environment.
  \item Under it there is the possibility to add and remove ontologies which
  model low-level information of the current specific domain.
\end{itemize}

On the one hand, this approach allows developers to consider richer information,
as activities, and abstract knowledge about the current global context. On the
other hand, it makes possible to model specific knowledge of the current sub-domain.
Besides, the possibility to plug and unplug these low-level ontologies makes this
solution powerful. The solution provided by~\citet{gu_toward_2004} is significant
because of the following reasons:

\begin{itemize}
  \item It considers high-level information on top of a more specific and domain
  dependent sub-model. 
  \item Activities are modelled as a relevant concept for context in the upper
  ontology.
  \item The modelled entities are related. Persons are associated with locations
  and activities, each location is linked with indoor or outdoor entities, 
  activities can be categorized into scheduled or deduced ones, and so forth. 
\end{itemize}

As the work by~\citet{chen_survey_2000} is a survey of context-aware computing
the remarked context characteristics (location and time) are given more as an
advice, since there is no model in their work. Nevertheless, they introduce 
several novelties like the term \textit{context history}, which might be useful 
for future predictions about user's behaviour and trends.

Modelling high-level information allows to perform deeper computations taking
into account behavioural characteristics, trends information, inferred knowledge
from small pieces of information combinations, and so on. As can be seen in
Table~\ref{tbl:context_comparison} many authors work with high-level data in 
their context-aware systems. 

On the other hand, physical context parameters are frequently modelled in the
analysed literature. Location, time and environment conditions (e.g., temperature,
pressure, light and noise) are usually modelled to achieve final system's goal
(e.g., adapting the user interface or recommending items or services). Besides, 
several approaches take user related characteristics to fulfil their purposes. 
For example,~\citet{schmidt_there_1999} consider not only physical environment 
as context information but also several human factors categorized as follows:

\begin{itemize}
  \item User information, which gathers knowledge about user's habits, emotional
  stated and bio-physiological conditions.
  \item Social environment, made up by other user's locations, social interactions
  of the current user and knowledge about the behaviour of groups of people.
  \item Task, taking into account spontaneous activities, engaged tasks and 
  general goals.
\end{itemize}

A user modelling approach which considers emotional issues of the user is presented
by~\citet{pereira_triple_2005}. As is remarked in Section~\ref{sec:pereira}
the author distinguishes between emotions and perceptions to separate both concepts
for modelling processes. These approaches are found useful for recommending systems
in which user mood and psychological state are relevant to filter multimedia
content \citep{baltrunas_incarmusic_2011}. Following a similar perspective~\citet{evers_achieving_2012}
analyse the user's participation in adaptive applications. Several concepts that
should be taken into account regarding the user behaviour and interaction with
the application are presented (see Section~\ref{sec:evers}). Another similar
approach regarding the user's more psychological aspects is presented by~\citet{liao_decision_2005}.
Here the authors present a model for modelling user stress levels. This is related
to the concept of Considerate Computing, term that it is explained in~\citep{gibbs_considerate_2005}.

\citet{schmidt_there_1999} also remark the social environment as relevant for 
context modelling. More related to recommending environments, non explicit 
information about user's likes need to be computed. Similar approaches in these 
environments tend to avoid recommending systems intrinsic problems, like the 
so-called \textit{cold start problem}~\citep{castillejo_alleviating_2012}~\citep{castillejo_social_2012}.

Another relevant point remarked by~\citet{schmidt_there_1999} are the user's tasks.
Several authors consider activities relevant for modelling context 
\citep{henricksen_modeling_2002}~\citep{gu_toward_2004}~\citep{abowd_towards_1999}.
Activities enrich context information about the user~\citep{razmerita_ontology_based_2003}.
It is common to model user activities in the user model (see Table~\ref{tbl:user_comparison}). 
% However,
% taking into account that these activities are identified by sensors and with the
% interaction with the environment, the proposed model for context considers them
% as a context parameter. In addition, the followed perspective takes into account
% activities as an abstract term. The context model in this dissertation does not
% model individual activities but those situations in which the user can not achieve
% the final goal. This is explained with more detail in Section~\ref{sec:stressful_activities}.

Finally, sometimes the collected data might lead to misunderstandings or 
non-trustworthy data.~\citet{almeida_assessing_2012} consider ambiguity and 
uncertainty in their work and present an ontology-based process which allows to 
model them within a smart environment.

Table~\ref{tbl:context_comparison} shows the differences between each reviewed
solution for modelling context. It is difficult to gather all the approaches in a
unique table. Therefore, Table~\ref{tbl:context_comparison} just emphasizes several
differences, although these solutions have more characteristics modelled. For example,
the work by~\citet{almeida_assessing_2012} is focused on modelling context uncertainty
and vagueness due to the nature of context data from sensors.

Besides and as it is deeply discussed by~\citet{strang_context_2004}, modelling 
the context with ontologies offers the following advantages:
\begin{itemize}
  \item Ontologies are the most expressive approach to model context.
  \item Composition and management of the model can be done in a distributed
  manner.
  \item It is possible to partially validate the contextual knowledge.
  \item One of the main strengths of ontologies is the simplicity to enact the
  normalization and formality of the model.
\end{itemize}

Regarding the context-aware systems, we cannot go further without remarking the 
issues of gathering information from different sources or sensors. This 
information might sometimes be unreliable. Sensors can fail in the collecting 
process, they also can stop working due to several reasons (e.g., power or 
malfunction). What is more, every sensor speaks its language (this issue is 
addressed applying different techniques, like data fusion). Therefore, a process 
to evaluate the quality and trustworthiness of the collected information should 
be included in every context-based system as a method to avoid undesired results.

\begin{table}
 \caption{Related work for context modelling approaches.}
 \label{tbl:context_comparison}
 \footnotesize
 \centering

\begin{tabular}{l c c c c c c c c c c c }
  \hline 
 \textbf{Solution} & \textbf{Ontologies} & \multicolumn{10}{c}{\textbf{Context parameters}} \\
 
 \textbf{(Domain)} & & \textbf{L} & \textbf{T} & \textbf{A} & \textbf{R} & \textbf{P} & \textbf{E} & \textbf{S} & \textbf{I} & \textbf{U} & \textbf{H}\\
    \hline 
        
    2000,~\citet{chen_survey_2000} 		&  & $\checkmark$ & $\checkmark$ &  &  &  &  &  &  &  &  \\
    (Context awareness)\\
    2001,~\citet{jameson_modelling_2001} 	&  & $\checkmark$ &  & $\checkmark$ &  &  &  &  &  & $\checkmark$ & $\checkmark$ \\
    (Context modelling)\\
    2002,~\citet{henricksen_modeling_2002} 	&   &  &  & $\checkmark$ & $\checkmark$ &  &  & $\checkmark$ & $\checkmark$ &  & $\checkmark$ \\
    (Pervasive computing)\\
    2002,~\citet{held_modeling_2002} 		&  &  &  &  &  &  &  &  & $\checkmark$ & $\checkmark$  &  \\
    (Context awareness)\\
    2004,~\citet{gu_toward_2004} 		& $\checkmark$ &  &  &  &  &  &  &  &  &  & $\checkmark$ \\
    (Smart environments)\\
    2005,~\citet{chen_using_2005} 		& $\checkmark$ & $\checkmark$ & $\checkmark$ & $\checkmark$ & $\checkmark$ & $\checkmark$ & $\checkmark$ &  & $\checkmark$ & $\checkmark$ & \\
    (Pervasive computing)\\
    2005,~\citet{yamabe_citron_2005} 		&  & $\checkmark$ &  & $\checkmark$ &  &  & $\checkmark$ &  &  & $\checkmark$ & $\checkmark$ \\
    (Mobile computing)\\
    2008,~\citet{wood_context_aware_2008} 	&  &  &  & $\checkmark$ &  &  & $\checkmark$ &  &  & $\checkmark$ &  \\
    (\ac{aal})\\
    2011,~\citet{baltrunas_incarmusic_2011} 	&  &  &  &  &  &  & $\checkmark$ &  &  & $\checkmark$ &  \\ 
    (Recommender systems)\\
    2012,~\citet{mcavoy_ontology_based_2012} 	& $\checkmark$ &  &  &  &  & $\checkmark$ &  &  &  &  & $\checkmark$  \\
    (Smart environments)\\
    2013,Almeida and	& $\checkmark$ & $\checkmark$ &  &  &  &  & $\checkmark$ &  &  &  &  \\
    López-de-Ipiña~\citep{almeida_assessing_2012}\\
    (Smart environments)\\
\hline

\end{tabular}
\end{table}