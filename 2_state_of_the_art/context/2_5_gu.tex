\subsubsection{2004: Gu et al.: The \ac{socam} Ontology}
\label{sec:gu}

In 2004~\citet{gu_toward_2004} design the \acf{socam} architecture for designing 
and prototyping applications in an \ac{ie}. Built on top of the 
\ac{osgi}\footnote{www.osgi.org} architecture, such middleware consisted of the 
following components:

\begin{itemize}
  \item Context providers, which abstract context information from heterogeneous
  sources and semantically annotate it according to the defined ontology.
  \item The context interpreter, which provides logic reasoning to process
  information about the context.
  \item The context database, which stores current and past context instance data.
  \item Context-aware applications, which adapt their behaviour according to the
  current context situation.
  \item The service-locating service, which allows context providers and 
  interpreters to advertise their presence for users and applications to locate 
  them.
\end{itemize}

\citeauthor{gu_toward_2004} use \ac{owl} to describe their context ontologies 
in order to support several tasks in \ac{socam}. As the pervasive computing 
domain can be divided into smaller sub-domains, the authors also divided the 
designed ontology into two categories: 

\begin{itemize}
  \item An \textit{upper ontology}, which captures high-level and general 
  context knowledge about the physical environment.
  \item A \textit{low-level} ontology, which is related to each sub-domain and 
  can be plugged and unplugged from the upper ontology when the context changes.
\end{itemize}

As a result, the upper ontology considers person, location, computational entity
and activity as context concepts.

\citet{gu_ontology_based_2004} also present an \ac{owl} based model to represent, 
manipulate and access context information in smart environments. The model 
represents contexts and their classification, dependency and quality of
information using \ac{owl} to support semantic interoperability, contextual 
information sharing, and context reasoning. The ontology allows to associate 
entities' properties with quality restrictions that indicate the contextual 
information quality. 