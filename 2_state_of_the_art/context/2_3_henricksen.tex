\subsubsection{2002: Henricksen et al.}
\label{sec:henricksen}
% MENCIONAR TAMBIÉN SU TRABAJO EN 2003.

The approach followed by~\citet{henricksen_modeling_2002} makes the following
notes about context in pervasive environments:

\begin{itemize}
  \item Context information exhibits \textit{temporal characteristics}. Context 
  can be  static (e.g., birthday) or dynamic (e.g., user location). Besides, the 
  persistence of the dynamic information can easily change. Thus, it is justified 
  that the static context should be provided by the user, while the dynamic 
  one should be gathered by sensors. Past historic and a possible forecasting 
  of future context are also taken into account as part of the description of 
  the whole context description.
  
  \item A second property of the context information is its \textit{imperfection}. 
  Information can be useless if it cannot reflect a real world state. It also 
  can be inconsistent if it contains contradictions, or incomplete if some 
  context aspect are unknown. There are many causes to these situations. For 
  example, information can change so fast that it may be invalid once it is 
  collected. This is obviously because the dynamic nature of the environment. 
  Besides, there is a strong dependency on software and hardware infrastructures, 
  which can fail any time.
  
  \item Context has \textit{multiple alternative representations}. Context 
  information usually comes from sensors which speak different languages. For 
  example, a location sensor can use latitude and longitude physical magnitudes 
  while the involved application works with a logical representation of location. 
  
  \item \textit{Context information} is highly \textit{disassociated}. There are 
  obvious connections between some context aspects (e.g., users and devices). 
  However, other connections need to be computed with the available information.
\end{itemize}

% \The Figure~\ref{fig:henricksen} shows the annotated context model designed by Henricksen et al.
This work also indicates the dependency between context models, the scenarios 
and use cases of the application domains. Authors extract several context 
parameters to consider:

\begin{itemize}
  \item User activity, distinguishing between the current one and the planned one.
  \item Device that is being used by the user.
  \item Available devices and resources.
  \item Current relationships between people.
  \item Available communication channels.
\end{itemize}

% \InsertFig{henricksen}{fig:henricksen}{Context model annotated with quality parameters and metrics \cite{henricksen_modeling_2002}}{}{0.70}{}

% 
% \subsubsection{2003, Henricksen et al.}
% %\citep{henricksen_generating_2003}
% \label{sec:henricksen}
% 
% Henricksen et al. discuss about the difficulties of constructing context-aware
% applications. Besides, they remark the lack of formality and expressiveness of
% previous context-aware solutions and models. For example, several specific context
% information, as histories, uncertainty, incompleteness, sensor-derived information
% and different kind of dependencies between the information are barely conceptually 
% modeled with ER or UML approaches, which are not well suited for this task \citep{henricksen_generating_2003}.
% Therefore, they present a context modeling approach which allows developers to describe high level context
% information. In addition, a mapping process for transforming high-level context
% models to management systems is also described. This way, they characterize the
% Object-Role Modeling approach to support specific context information based
% on abstraction concepts and quality annotations. 
% 
% Authors classify context into static or dynamic facts\dots