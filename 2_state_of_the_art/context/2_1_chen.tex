\subsubsection{2000: Chen and Kotz}
%\cite{chen_survey_2000}
\label{sec:chen}

\citet{chen_survey_2000} define context as follows: 

\newpage

\begin{description}
  \item[\Defi{Context, by~\citet{chen_survey_2000}}] \hfill \\
  \begin{mdframed}[hidealllines=true,backgroundcolor=gray!20]
  \textit{``Context is the set of environmental states and settings that either 
  determines an application’s behaviour or in which an application event occurs 
  and is interesting to the user''}.
  \end{mdframed}
\end{description}

Besides, the definition of active and passive context-aware computing is also 
given. To \citeauthor{chen_survey_2000}, active context awareness is:

\begin{description}
  \item[\Defi{Active Context Awareness, by~\citet{chen_survey_2000}}] \hfill \\
  \begin{mdframed}[hidealllines=true,backgroundcolor=gray!20]
  \textit{``(\dots) an application automatically adapts to discovered context, by the 
  changing in the application's behaviour''}.
  \end{mdframed}
\end{description}

On the contrary, they define passive context awareness as:

\begin{description}
  \item[\Defi{Passive Context Awareness,~\citep{chen_survey_2000}}] \hfill \\
  \begin{mdframed}[hidealllines=true,backgroundcolor=gray!20]
  \textit{``(\dots) an application presents the new or updated context to an interested
  user or makes the context persistent for the user to retrieve later''}.
  \end{mdframed}
\end{description}

Based on the work by~\citet{schilit_context_aware_1994},~\citet{chen_survey_2000}
consider \textit{time} as an important and natural context feature for many 
applications. Besides, they introduce the term \textit{context history}, which 
is an extension of the time feature recorded across a time span.

A significant problem remarked in this work is the impossibility to exchange
context information between the studied context-aware systems due to the way
they model the environment. Furthermore, as \textit{location} is one of the most 
modelled context features, \citeauthor{chen_survey_2000} provide a study of 
several aspects that should be taken into account when researchers face the 
problem of modelling it. 

% \begin{table}[htbp]
% \caption{Data structures \cite{chen_survey_2000}}
% \label{tbl:chen}
% \begin{tabular}{ll}
% Technique & Description  \\
% \hline
% Key-value pairs & Used by Schilit et al. \cite{schilit_context_aware_1994}, \\
%  & an environmental variable acts as the key while the \\
%  & actual context data is the value. \\
% Tagged encoding & Used by P.J. Brown \cite{brown_stick-e_1995}, the \\
%  & contexts are modeled as tags and corresponding \\ 
%  & fields. \\
% Object-oriented model & The systems that use this technique usually \\
%  & consider the contextual information as the states \\ 
%  & of the object and the object provides methods to \\ 
%  & access and modify these states \\
% Logic-based model & Context data can be expressed as facts in a \\
%  & rule-based system. \\
% % Others & Lightning \\
% \end{tabular} 
% \end{table}


% ----------------------------------------------------------------------

