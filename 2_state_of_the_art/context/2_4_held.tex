\subsubsection{2002: Held et al.}
% \citep{held_modeling_2002}
\label{sec:held}

In 2002~\citet{held_modeling_2002} discussed about the necessity of content 
adaptation and representation formats for context information in dynamic 
environments. A significant perspective is the justification of modelling not 
only the user but the network status and device context information as well.
The following parameters are considered as relevant context information:

\begin{itemize}
  \item Device: basic hardware features (e.g., \acs{cpu} power, memory and so 
  forth), user interface input (e.g., keyboard and voice recognition), output 
  (e.g., display and audio), and other particular specifications of the 
  device (e.g., display resolution or colour capability).
  \item User: service selection preferences, content preferences and specific
  information about the user.
  \item Network connection: bandwidth, delay, and bit error rate. 
\end{itemize}

Authors also present several requirements concerning the representation of
context information. Accordingly, they defend that a context profile should be:

\begin{itemize}
  \item Structured, to ease the management of the amount of gathered information
  and remark relevant data about the context.
  \item Interchangeable, for components to interchange context profiles.
  \item Composable/decomposable, to maintain context profiles in a distributed way.
  \item Uniform, to ease the interpretation of the information.
  \item Extensible, for supporting future new attributes.
  \item Standardized, for context profile exchanges between different entities
  in the system.
\end{itemize}