\subsubsection{1991: Jon Orwan and the Doppelgänger system}
\label{sec:orwant_doppelganger}

Doppelgänger is a user modelling system that performs inferences upon user data
and makes this information available to applications. The system allows users to
modify their models, it makes implicit generalizations about the data (which is
gathered through different channels) and provides an extensible architecture~\citep{orwant_doppelgangeruser_1991}. 

% ``Figure 1 The DOPPELGÄNGER user modelling system gathers data about users from sensors, makes inferences on
% those data, and makes the results available to applications'' cogido de http://www.cs.ucf.edu/~dcm/Teaching/COT4810-Spring2011/Presentations/FrankHines-CheapUserModelingForAdaptiveSystems.pdf

User data is gathered through a continuously operative sensor network which
senses users' everyday activities (see Listing~\ref{lst:orwant_sensor_1}). Hence,
both long-term and short-term information about the user are stored. Sensors are
integrated in users' activities. However, Doppelgänger acknowledges the imperfection
of real world data. Data streams are usually incomplete or erroneous. The main
objective of the Doppelgänger system is to recover from these imperfections through
several learning techniques:

\begin{itemize}
  \item The Beta distribution~\citep{drake1967fundamentals}, which is used to
  determine the preference strength, probability of accuracy and the confidence
  of an estimation.
  \item Linear prediction, to predict a possible next event.
  \item Markov models, which represents user's behaviour through several states and
  all possible transitions between them with the corresponding probability.
\end{itemize}


The system maintains an accuracy estimation for each sensor, which helps the
system to decide a confidence metric for the gathered data.

\newpage

\begin{minted}[linenos=true, fontsize=\footnotesize, frame=lines]{vim}
(object orwant location (place 344) (time 779562701) 
(id active-badge))
\end{minted}
\captionof{listing}{Message from a sensor to the server~\citep{orwant_heterogeneous_1994}.\label{lst:orwant_sensor_1}}


The user models are represented in SPONGE, a LISP based data structure manipulated with
C and Pearl programs \citep{orwant_heterogeneous_1994} (see Listing~\ref{lst:orwant_sensor_2}). 
Models are stored as Unix directories consisting of domain submodels and conditional 
submodels. The first one contains information about the user behaviour (i.e., 
location, preferences, and so forth); the second group contains triggering information 
for deciding actions when certain situations are met. Besides, each user model 
is conceptually represented as a point in a high dimensional space in which the 
dimensions are determined by the number of sensors in the network.


% \InsertFig{orwant}{fig:orwant}{Orwant user model \citep{orwant_heterogeneous_1994}}{}{0.70}{}
% 
% \InsertFig{orwant}{fig:orwant_sensors_apps}{The DOPPELGÄNGER user modelling system gathers data about users from sensors, makes inferences on
% those data, and makes the results available to applications. \citep{orwant_for_1996}}{}{0.70}{}

\begin{minted}[linenos=true, fontsize=\footnotesize, frame=lines]{vim}
(object orwant primary
  (object biographical_data
    (string_binding "true name" "Jon Orwant")
    (string_binding "e-mail address" orwant@media.mit.edu)
  ...)
  (object control
    (int_binding "doppelganger ID" 4))
...)
\end{minted}
\captionof{listing}{Orwant user model~\citep{orwant_heterogeneous_1994}.\label{lst:orwant_sensor_2}}

% \begin{figure}
% \centering
% \includegraphics[width=0.7\textwidth]{orwant_sensors_apps.pdf}
% \caption{The DOPPELGÄNGER user modelling system gathers data about users from sensors, makes inferences on
% those data, and makes the results available to applications~\citep{orwant_for_1996}.}
% \label{fig:orwant_sensors_apps}
% \end{figure}
% ----------------------------------------------------------------------

