
\subsection{Users Models Comparison}
\label{sec:user_model_comparison}

In this section we describe our user modelling requirements nurtured by the 
earlier works described. As mentioned before, there are several perspectives 
regarding the user modelling requirements. In this dissertation the \ac{hci} 
perspective is taken. This means, as~\citet{pohl_logic_based_1999} states, 
that the user model refers to user characteristics using a certain system. 
In Chapter~\ref{cha:ontology_model} the proposed models for user, context and device are described. These models
have been designed considering each of these entities as a set of characteristics
that define them. Hence, the adaptation platform is able to evaluate the
combination of these characteristics and perform the corresponding adaptation.
Thus, the \ac{hci} interpretation of user modelling suits the main objective 
remarked in Section~\ref{sec:hypothesis}.

% Now that the \ac{hci} viewpoint has been remarked, we emphasize the amount of 
% different domains that are addressed in the literature considering user 
% modelling. 

In the following lines the amount of different domains addressing user modelling 
are highlighted. From product design to multimedia and user interfaces adaptation, 
every approach follows the same purpose: to consider several user characteristics 
to improve the system and user's satisfaction and product or service usability. 
However, although these solutions share the same objectives, the considered 
characteristics differ. For ubiquitous and more context-aware domains, activities 
are taken into account. For example, ~\citet{razmerita_ontology_based_2003} 
discuss an ontology based architecture, which aims to be generic by collecting 
user data through two different ways (\textit{explicitly} and \textit{implicitly}).
\citet{golemati_creating_2007} also take an ontological point of view to avoid 
the problem of domain dependency (among others) by designing a more general, 
comprehensive and extensible ontology. 

\citet{gauch_ontology_based_2003} remark in the presented ontology the 
importance of time. Regarding the studied domain (web browsing) time is 
significant because it might help characterizing the user considering the spent 
time reading an article or visiting a website. Well known and popular 
recommendation systems, as YouTube, utilise this information combined with 
different explicit and implicit sources from the user interaction to make proper
recommendations~\citep{davidson_youtube_2010}.

User activities have also been considered as relevant for many authors in the 
literature. The first example is the Doppelgänger 
system~\citep{orwant_doppelgangeruser_1991} (1991), which uses activities for 
sharing relevant user information to different applications. In the same 
way,~\citet{persad_characterising_2007},~\citet{heckmann_gumogeneral_2005}
and~\citet{skillen2012ontological} model activities to take user's behaviour
and interaction into account for the proposed classifications and systems. For
~\citet{hatala_ontology_based_2005} activities are also vital within 
context-aware environments.

As occurs with context modelling (see Section~\ref{sec:context}) many different 
techniques are available for the model representation. This usually depends either 
on the developer, because of his/her experience, or in the system's technical 
characteristics. For example, if the system is able to performs inference with 
the user data an ontology based representation could be more helpful than an 
object based one.~\citet{strang_context_2004} demonstrated that ontological
modelling is more appropriate for ubiquitous computing environments.

It is also common to model physiological related characteristics of the user.
For example, the works by~\citet{gregor_designing_2002} 
and~\citet{persad_characterising_2007} consider physical, cognitive an sensory 
capabilities.~\citet{skillen2012ontological} also model several user abilities 
for performing different tasks and activities. The problem is that being aware 
of these capabilities is difficult and, in some cases, poorly practical. For 
example, measuring the sight capability of one individual requires physiological 
experience or advice. Besides, people with the same affection do not respond in 
the same way. A person who was born blind would interact differently with the 
environment than another who has been losing sight during his/her life. The 
precise same disability (e.g., tunnel vision) might affect different people 
in many different ways depending on their personal skills (e.g., adaptability,
orientation, and so forth). This might lead to an idea. What if, instead of 
modelling physiological skills (disabilities), we were able to model user's 
capabilities? The first approximation for this is found 
in~\citet{casas_user_2008} work. \citeauthor{casas_user_2008} present a user 
profile which abstracts from physiological aspects and lets users manage and 
configure their own profile. On the contrary, \citet{skillen2012ontological} 
model user capabilities as a set of abilities which allow users to perform some 
task or activities. Although this perspective covers many user capabilities, it
still needs a deeper understanding of physiological user capabilities.

On the other hand,~\citet{fischer_user_2001} states that it not only is 
difficult to model users because of the wide range of different types of people 
that exist. He also considers that each individual changes with experience and 
through time. For example, old people's capabilities decrease with ageing. This
idea is shared with~\citet{gregor_designing_2002}, whose work is centred around
the elderly. \citet{heckmann_gumogeneral_2005} not only consider that users 
might evolve (from an \ac{ai} perspective), but he also takes new context 
information from the inference process. This also opens a new point of view 
that we address in Section~\ref{sec:context_disabilities} and it is about taking 
context as a significant user's environment entity that might directly influence 
the user's capabilities. In other words, \textit{users change through experience, 
time and, in concrete situations, due to the current context characteristics}. For 
example, an individual might not suffer from any mobility disability, but in a 
crowded street would be difficult to perform several daily activities (just 
walking could be difficult).~\citet{razmerita_ontology_based_2003} address 
this issue considering an implicit user information collecting process. This, 
of course, deals with the concept of dynamism. Finally, \citet{evers_achieving_2012} 
consider that respecting user's interactive behaviour with applications needs to 
be taken into account. On the other hand, \citet{pereira_triple_2005} analyses 
the differences between emotional and perceptional user characteristics. 

% Several authors have also noticed the importance of tolerating the management of
% non-trustworthy data. Ambiguity and uncertainty mean working with data which might
% not reflect the current situation. Beynon et al. considered uncertainty as an actual
% problem to deal with \citep{beynon2000dempster}.

Table~\ref{tbl:user_comparison} summarizes the analysed approaches for user
modelling, emphasizing the modelled user characteristics and domains. 


% Besides, 
% although in a first version every used technique was remarked, in this 
% dissertation we focus on remarking just those which follow an ontology-based 
% approach. This is because many of the cited works are more theoretical or 
% surveys, or they just give some advices about important context data when 
% facing a context modelling task. Besides, \citet{strang_context_2004} 
% demonstrate that using ontologies is more appropriate for modelling 
% context-aware systems.

\begin{table}
  \caption{Related work for the user modelling approaches. Under the user
 characteristics heading \underline{A}ctivities or behaviour, \underline{C}apabilities,
 \underline{Ex}perience,  \underline{I}nterests, \underline{E}motions,
 \underline{P}ersonal, \underline{S}tress and \underline{L}ocation information
 are presented.}
 \label{tbl:user_comparison}
\footnotesize
\centering
 \begin{tabular}{l c c c c c c c c c}
  \hline 
  \textbf{Solution} & \textbf{Ontologies} & \multicolumn{8}{c}{\textbf{User characteristics}}\\
  \textbf{(Domain)} & & \textbf{A} & \textbf{C} & \textbf{Ex} & \textbf{I} & \textbf{E} & \textbf{P} & \textbf{S} & \textbf{L} \\
  \hline
  
  2002,~\citet{gregor_designing_2002}		&  		& & $\checkmark$ & $\checkmark$ & & & & & \\
  (Inclusive design)\\
  2003,~\citet{gauch_ontology_based_2003}	& $\checkmark$	& & & & $\checkmark$ & & & &\\
  (Automatic profiling)\\ 
  2003,~\citet{razmerita_ontology_based_2003}	& $\checkmark$	& $\checkmark$ & & & $\checkmark$ & & $\checkmark$ & & \\
  (\ac{kms})\\				
  2005,~\citet{hatala_ontology_based_2005} 	& $\checkmark$ 	& & & & $\checkmark$ & & $\checkmark$ & & $\checkmark$ \\
  (Tangible interfaces)\\		
  2005,~\citet{pereira_triple_2005} 		& 		& & & & & $\checkmark$ & & &  \\
  (Multimedia adaptation)\\
  2007,~\citet{persad_characterising_2007} 	&   		& $\checkmark$ & $\checkmark$ & & & & & & \\
  (Product design demands)\\
  2007,~\citet{golemati_creating_2007} 		&  $\checkmark$   	& $\checkmark$ & & $\checkmark$ & $\checkmark$ & & $\checkmark$ & & \\
  (User profiling)\\
  2007,~\citet{heckmann_gumogeneral_2005} 	& $\checkmark$   	& $\checkmark$ & & & & $\checkmark$ & $\checkmark$ & $\checkmark$ & \\
  (Ubiquitous applications)\\
  2008,~\citet{casas_user_2008} 		&  		& & $\checkmark$ & $\checkmark$ & & & & & \\
  (\ac{aui}\\
  2012,~\citet{evers_achieving_2012} 		&  		& & & & & & & $\checkmark$ & \\
  (Adaptive applications)\\
  2012,~\citet{skillen2012ontological} 		&  $\checkmark$	& $\checkmark$ & $\checkmark$ & & $\checkmark$ & & & & $\checkmark$ \\
  (Mobile environments)\\
  \hline

\end{tabular}
\end{table}