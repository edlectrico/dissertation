
\subsubsection{2008: Casas et al.}
\label{sec:casas}

Another approach is the one presented by~\citet{casas_user_2008} In this case 
the authors work under the \textit{Persona} concept which is introduced to 
distinguish between different user groups within an adaptive user interfaces
domain. Originally this concept was introduced by \citeauthor{cooper_inmates_2004} 
in 1999 by the following definition:

\begin{description}
  \item[\Defi{Personas, by \citet{cooper_inmates_2004}}] \hfill \\
  \begin{mdframed}[hidealllines=true,backgroundcolor=gray!20]
  \textit{``Personas are not real people, but they represent them through a design 
  process. They are hypothetical archetypes of real users''}. 
  \end{mdframed}
\end{description}

\citeauthor{casas_user_2008} distinguish between two categories of people:

\begin{itemize}
 \item \textit{Primary}: those who represent the main group and use primary 
 interfaces. 
 \item \textit{Secondary}: those who can use primary interfaces, but with 
 several extra needs.
\end{itemize}

By assigning random values to several characteristics (e.g., age, education,
profession, family conditions, disabilities and technological experience) 
authors are capable of covering a wide range of potential users. However, the 
most significant contribution is that, instead of being focused on users 
capabilities, they consider users needs. To that end they build a user 
profile supported by four main bases: 

\begin{itemize}
 \item The user level, which indicates the ability of the user to face the 
 system.
 \item Interface, for the interaction mechanism to be used by the user.
 \item Audio, to indicate the audio volume levels.
 \item Display, which includes usual display controls (contrast, colours, 
 brightness and so on).
\end{itemize}

This approach is focused on the solution, on the adaptation itself. It is a 
perspective which allows users to configure the interaction based on their 
capabilities. This helps applications designers because user capabilities are 
not directly taken into account in the model as medical or technical aspects. 
Hence, there is no need to be experts or have any medical knowledge about 
users disabilities. Another advantage is that each user can manage his/her own 
profile. Thus, they can configure their preferences and capabilities on 
their own. 