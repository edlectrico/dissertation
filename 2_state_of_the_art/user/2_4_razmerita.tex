\subsubsection{2003: Razmerita et al.: The OntobUM Ontology}
\label{sec:razmerita2003ontology}

Focused in the context of \ac{kms}~\citet{razmerita_ontology_based_2003} present
OntobUM, a generic ontology-based architecture for user modelling. The model is 
generated through two different ways:

\begin{itemize}
  \item \textit{Explicitly}, using a user profile editor. Therefore, the user has to 
  provide some information.
  \item \textit{Implicitly}, information maintained by several intelligent 
  services which maintain and update the information about the user considering 
  the user's behaviour with the services and provide adapted services based on 
  user's preferences.
\end{itemize}

The architecture of the presented ontology is composed of the following ontologies:

\begin{itemize}
  \item \textit{The User Ontology}, which structures the different characteristics
  and preferences of the user.
  \item \textit{The Domain Ontology}, which defines several concepts about the
  domain.
  \item \textit{The Log Ontology}, which manages the semantics of the interaction 
  between the user and the whole system.
\end{itemize}

Authors identify several users' characteristics that are relevant for a \ac{kms} 
under the Behaviour concept. Nevertheless, most of the user ontology is 
generic and it is available to be used in other application domains. 

%Dynamic user modeling? Atención al proceso de captura de datos implícita.
