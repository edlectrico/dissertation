
\subsubsection{Generic User Modelling Systems}
\label{sec:generic_users}

There is another generic approach for modelling users known as Shell Systems. 
More focused in the field of \ac{ai}, these solutions consider the user model as 
a source of information which are built on assumptions about relevant user aspects 
or behaviour~\citep{pohl_logic_based_1999}. In the following lines several 
significant researches in the field are highlighted in order to take into account
the difference between generic user modelling approach (user modelling shells) and
the approach followed in this dissertation, which is related to \ac{hci}.

\citet{heckmann_ubiquitous_2005} differentiates between the model, which is where 
the user data is collected, and the modelling system, which is the module that
manages the model. Besides, he remarks the following two definitions
from~\citet{wahlster1989user} previous work:

\begin{description}
  \item[\Defi{User Model (I), by~\citet{wahlster1989user}}] \hfill \\
  \begin{mdframed}[hidealllines=true,backgroundcolor=gray!20]
  \textit{``A user model is a knowledge source in a system which contains explicit
  assumptions on all aspects of the user that may be relevant to the behaviour
  of the system. These assumptions must be separable by the system from the
  rest of the system's knowledge''}.
  \end{mdframed}

  \item[\Defi{User Model (II), by~\citet{wahlster1989user}}] \hfill \\
  \begin{mdframed}[hidealllines=true,backgroundcolor=gray!20]
  \textit{``A user modelling component is that part of a system whose function is to
  incrementally construct a user model; to store, update and delete entries;
  to maintain the consistency of the model; and to supply other components of
  the system with assumptions about the user''}.
  \end{mdframed}
  
\end{description}


% % \begin{framed}
%   \begin{mydef}
%     {A user model is a knowledge source in a system which contains explicit
%     assumptions on all aspects of the user that may be relevant to the behaviour
%     of the system. These assumptions must be separable by the system from the
%     rest of the system's knowledge.~\citep{wahlster1989user}}
%   \end{mydef} 
% % \end{framed}
% 
% % \begin{framed}
%   \begin{mydef}
%     {A user modelling component is that part of a system whose function is to
%     incrementally construct a user model; to store, update and delete entries;
%     to maintain the consistency of the model; and to supply other components of
%     the system with assumptions about the user.~\citep{wahlster1989user}}
%   \end{mydef} 
% % \end{framed}

Assumptions are deeply studied and depicted in~\citep{pohl_logic_based_1999}. 
In this section, a quick overview of how these systems behave is performed to 
just take into account the difference between generic user modelling approach 
(user modelling shells) and the approach followed in this dissertation, which 
is related to Human-Computer Interaction.

In 1986 \ac{gums} is presented. This software allowed developers to make 
user-adaptive applications by defining several stereotypes, facts and rules to 
reason with~\citep{finin_gums_1986}. \acs{gums} supports the addition of new 
facts in runtime, manages facts inconsistencies and answers the application 
about assumptions about the user~\citep{kobsa_generic_2001}. The following 
systems are several examples of generic user systems developed in the following 
years (chronological ordered):

\begin{itemize}
  \item Doppelgänger, which uses several learning techniques for generalizing
  and extrapolating sensor data for the development of the user model
  \citep{orwant_doppelgangeruser_1991}.
  \item \ac{umt}, which supports the definition of hierarchically ordered 
  stereotypes about the user, rules for user model inferences and contradiction 
  detection~\citep{brajnik1994shell}.
  \item TAGUS, which uses first-order formulas to represent assumptions about
  the user~\citep{paiva1994tagus}.
  \item The um toolkit, which models not only assumptions but beliefs, preferences and other
  user characteristics in attribute-value pairs~\citep{kay1994toolkit}.
  \item BGP-MS, which permits user and groups of users assumptions~\citep{kobsa1994user}.
\end{itemize}

% MÁS SISTEMAS? SOLO LLEGAMOS A 1995\dots ESTOS SISTEMAS SON MUY ANTIGUOS, SI NO PONEMOS A
% PARTIR DEL 2000 MEJOR NI PONERLOS IGUAL\dots


% A significant aspect of these systems is that they are required to be usable in
% as many applications as possible. This is, domain independent. Therefore, they
% are expected to provide as many services as possible. On the contrary, in Human-Interaction
% user modelling approaches this is, in fact, an issue \citep{cita_falta_modelos_usuarios_comparar}.
% This problem is addressed in Chapter~\ref{cha:el_que_sea}.

% Domain independence
% 
% Known systems: 
% 
% GroupLens
% LikeMinds
% Personalization Server
% Frontmind
% Learn Sesame

% \InsertFig{heckmann_model}{fig:heckmann_model}{Several user model property dimensions \citep{heckmann_user_2007}}{}{0.70}{}

% ----------------------------------------------------------------------

