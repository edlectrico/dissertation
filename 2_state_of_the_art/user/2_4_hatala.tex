\subsubsection{2005: Hatala and Wakary and the Ec(h)o system}
\label{sec:hatala}

Ec(h)o is an ontology-based augmented audio reality system for museums which 
aims to maintain rich and adaptive output information. The main purpose of this 
work is to address the problem of supporting experience design and functionality 
related to museum visits through user models combined with augmented reality 
and tangible user interface system.~\citet{hatala_ontology_based_2005} find
several challenges for capturing rich information about the context. For the
presented museum scenario, social, cultural, historical and psychological factors
are significant for the user experience. In this field, the argumentation made by
\citet{dourish_what_2004}~\citep{dourish_where_2004} is remarked as relevant.
\citeauthor{dourish_what_2004} states that activities and context are directly 
and dynamically linked. This concept is called \textit{embodied interaction}.

The core of the ec(h)o's reasoning module is a dynamically updated user model
\citep{wahlster1989user}. The ruled-based model changes as the user moves 
through the museum and selects several audio objects. This models enables 
developers to consider which inputs influence user interests. In the ec(h)o 
system there are two ways of updating the model: the user movement and a 
selection of an audio object. These actions have different effects on the model 
of the user interests (i.e., influence of initial interest selection, of object 
selection on user interest and of location change). 

As it occurs with recommender systems, user's interest are vital for the concept
ontology. These concepts are weighted in the ontology as concepts which represent
the user's likes within the environment. Besides, an interaction history is 
maintained recording the way the user interacts with the museum. In addition to 
these characteristics the user type is also considered. Hence, the system is 
allowed to characterize the user experience with the environment. It classifies 
users into three different categories:

\begin{itemize}
  \item The avaricious visitor, who wants to see as much as possible in a 
  sequentially way.
  \item The selective visitor, who is more selective with the concepts he/she is
  interested in.
  \item The busy visitor, who prefers to not spend much time and get a general
  vision of the exhibition.
\end{itemize}
