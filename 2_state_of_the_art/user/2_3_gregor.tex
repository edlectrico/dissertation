\subsubsection{2002: Gregor et al.}
\label{sec:gregor}

%------------------------------------------------------------------------- 
In 2002,~\citet{gregor_designing_2002} focus their approach on a certain groups 
of users: the elderly. A three group classification is presented. In the first 
group there are the fit older people, who do not suffer from any disability. The 
second group is formed by older fragile people who have one or more 
disabilities. Finally, the last group encompasses the older and people with 
disabilities whose capabilities to function depend on other people. In this 
case, the authors identify several user capabilities:

\begin{itemize}
 \item Physical, sensory and cognitive capabilities.
 \item The ability to learn new techniques (cognitive).
 \item Memory problems (cognitive).
 \item The environment can affect several elderly capabilities.
 \item Elderly experience (as a positive fact).
\end{itemize}

On the other hand,~\citet{gregor_designing_2002} consider that, as people grow 
older, their capabilities change. This process encompasses a reduction of 
cognitive, physical and sensory functions depending on the individual. This 
diversity is a significant issue for modelling users and designing computing 
systems.

Figure~\ref{fig:gregor} shows an adaptable user interface which takes into 
account these capabilities. 

% \InsertFig{gregor}{fig:gregor}{Adaptable browsing 
% interface~\citep{gregor_designing_2002}}{}{0.70}{}

\begin{figure}
\centering
\includegraphics[width=0.70\textwidth]{gregor.png}
\caption{Adaptable browsing 
interface~\citep{gregor_designing_2002}.}
\label{fig:gregor}
\end{figure}