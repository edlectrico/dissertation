\subsubsection{2005: Fernando Pereira}
\label{sec:pereira}

Within a video adaptation and quality of experience evaluation scenario,
~\citet{pereira_triple_2005} studies a user characterization through three
different dimensions: sensory, perceptual and emotional. First of all,
\citeauthor{pereira_triple_2005} establishes the difference between sensations 
and perceptions as follows:

\begin{itemize}
  \item Sensations are monomodal, more low-level, physical and less related to
  the real world composition than perceptions. They regard the simple conscious
  experience for the corresponding physical stimulus (e.g., light variation and
  eyes reaction to this change). They are related to the first contact between a
  human and the surrounding environment.
  
  \item Perceptions are multimodal, and they are part of the cognition process
  (knowing and learning) and regard the conscious experience and identification
  of objects.
\end{itemize}

On the other hand, emotions are considered as central in a communication and 
entertainment process. Therefore, \citeauthor{pereira_triple_2005} proposes a 
triple layered \ac{spe} user model for the evaluation of the quality of 
experience in the consumption of multimedia content.



