
\subsubsection{2007: Persad et al.}
\label{sec:persad}

\citet{persad_characterising_2007}~\citep{persad_cognitive_2007} relate user
capabilities and product demands as a tool to evaluate the product design (the
definition of inclusive design is given in Section~\ref{sec:definitions}).
\citeauthor{persad_characterising_2007} remark four main components to consider 
when dealing with interaction between people and technology: the user, the 
product, the environment or context, and the set of activities or tasks 
that define the interaction. The authors try to assess an adaptation degree 
between users and the designed products using different compatibility measures. 
These measures can be assessed on different levels of human capabilities, 
including sensory, cognitive and motor. The concepts of user capability and 
product demand provide a useful framework for analysing the user-device 
compatibility. The product demand levels are considered multidimensional and 
they are set by the interface attributes of the product itself. For example, 
a product's text display will be designed with a certain text size, font, and 
colour contrast. The combination of these attributes define the visual demand 
level within the user visual capabilities. Similarly, other combinations of 
product attributes command several cognitive and motor demands. 

