
\subsubsection{2007: Persad et al.}
\label{sec:persad}

\citet{persad_characterising_2007}~\citep{persad_cognitive_2007} relate user
capabilities and product demands as a tool to evaluate the product design (the
definition of inclusive design is given in Section~\ref{sec:definitions}).
\citeauthor{persad_characterising_2007} remark four main components to consider 
when dealing with interaction between people and technology: the user, the 
product, the environment or context, and the set of activities or tasks 
that define the interaction. The authors try to assess an adaptation degree 
between users and the designed products using different compatibility measures. 
These measures can be assessed on different levels of human capabilities, 
including sensory, cognitive and motor. The concepts of user capability and 
product demand provide a useful framework for analysing the user-device 
compatibility. The product demand levels are considered multidimensional and 
they are set by the interface attributes of the product itself. For example, 
a product's text display will be designed with a certain text size, font, and 
colour contrast. The combination of these attributes define the visual demand 
level within the user visual capabilities. Similarly, other combinations of 
product attributes command several cognitive and motor demands. 

\citeauthor{persad_cognitive_2007} also reviewed functional classifications 
and experimental studies to identify the most relevant low-level skills for 
designing products within the cognitive, motor and sensory domain. This is 
highly related to the \ac{icf} functions described in Section\ref{sec:background}.

\paragraph*{Sensory capabilities}
\subparagraph*{Visual capabilities} Several sensory capabilities are known to 
deteriorate with ageing~\citep{persad_exploring_2006}. Thus, 
\citeauthor{persad_exploring_2006} stated that the following functions seem to 
account for most of visual disability:

\begin{itemize}
  \item Visual acuity.
  \item Contrast sensitivity.
  \item Colour perception.
  \item Useful field of view.
  \item Stereopsis.
\end{itemize}


\subparagraph*{Hearing capabilities} Loss of hearing capabilities may directly 
affect the speech interaction with the device. The main low-level hearing 
functions to guarantee the interaction are the following:

\begin{itemize}
  \item Pure tone detection thresholds.
  \item Speech detection and recognition discrimination thresholds.
  \item Sound localization.
\end{itemize}

% \subparagraph*{Environmental effects}
% The level of illumination, noise, weather, etc. are several environment features 
% which might affect user capabilities.

\paragraph*{Cognitive capabilities}
The product's user interface must be usable and accessible enough to guarantee 
that users easily understand the interaction. The following capabilities are 
related to the human cognitive domain:

\begin{itemize}
 \item Working memory performance.
 \item Long term memory.
 \item Mental models, planning and problem solving.
 \item Language and communication capabilities.
\end{itemize}

\paragraph*{Motor capabilities}
\subparagraph*{Upper limb capabilities}
There are many conditions that affect manipulating a product (e.g., arthritis, 
stroke, multiple sclerosis, head injury, cerebral palsy and missing or damaged 
limbs). These problems directly reduce grasp forces, range of motion and fatigue 
thresholds~\cite{persad_characterising_2007}.

The following areas are highlighted within motor capabilities:
\begin{itemize}
  \item Reach ranges for each arm.
  \item Grasping, dexterity and force exertion.
  \item Two handed actions and coordination.
\end{itemize}

\subparagraph*{Gross body movement capabilities}
Usually products require certain user mobility degree. 
\\
Besides, \citeauthor{persad_characterising_2007} provide six general categories 
for product features and their interface classification. For this classification 
their toaster case study is considered (see Table~\ref{tbl:persad_product_interface}).

\begin{table}
  \caption{Product interface classification by~\citet{persad_characterising_2007}.}
  \label{tbl:persad_product_interface}
\footnotesize
\centering
    \begin{tabular}{l l}
    \hline
    \textbf{Feature type} 	& \textbf{Examples} \\
    \hline
    Product chassis 		& Handles, gripping surface. 		\\
    Displays and indicators 	& Visual and auditory displays. 	\\
    Controls and control groups & Discrete controls (button, switch) 	\\
				& and continuous controls (slider,  	\\
				& knob, thumb, wheel, dial, joystick).	\\
				& Control groups \textit{Keypad}. 	\\
    Material/media input 	& Slots (toaster slots), powered and 	\\
    and output			& un-powered bays and trays, doors, 	\\
				& lids and covers.			\\
    Connectors for energy and data & Power and data connectors 		\\
    Software interfaces 	& Navigation menus and \ac{gui} objects. \\
    \hline
  \end{tabular}
\end{table}

