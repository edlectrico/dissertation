
% Thesis Abstract -----------------------------------------------------


%\begin{abstractslong}    %uncommenting this line, gives a different abstract heading


\begin{abstracts}        %this creates the heading for the abstract page
\selectlanguage{british}
% Put your abstract or summary here.
Since the birth of the first \acp{gui}, \acp{aui} have been used to cover
a wider range of possibilities seeking adaptive presentation of the information. 
Starting with small personalization features, more related to practical interaction, 
first customizable menus and graphical elements arose. Subsequently, the 
possibility of breaking down the interaction barriers has grown. From the user
perspective, these limitations are usually caused by physiological disabilities,
which imped users to properly interact with or consume information. In fact, 
users may suffer from specific and/or temporary disabilities due to the context 
situation. This issue brings new challenges to \ac{aui} systems. Hence, from 
these challenges new adaptive tools started to be included and integrated with 
the purpose of minimising the identified interaction barriers. This also aimed 
to allow users to feature a sufficient interaction experience.

With the arrival of mobile telephony and portable devices, a wider range of
possibilities regarding \acp{aui} have emerged. This market evolves continuously, 
bringing smaller, more powerful and wearable devices. This trend has strengthened 
the \acp{aui} within this market. Nevertheless, the user interface adaptation for 
these devices is far from the advances in static devices, which are able to perform 
complexer computations. Besides, more problems are added to the equation if the 
context is considered. The context situation, characterized by the set of singular 
features which define it and their quality might make the context dynamic. Thus, 
in each case, different adaptations or configurations might be needed. 

In this dissertation these problems are faced, aiming to reduce the boundaries
between the user and the device in several limiting situations. To solve them, 
a dynamic and mobile user adaptation system is presented, principally supported 
by a semantic model which includes a conceptualization of the user, the current
context and the device. Its design allows increasing the expressiveness of the 
user interaction contextual and interaction needs, also taking into account the 
features provided by the user's device.


\end{abstracts}

\begin{resumen} %this creates the heading for the abstract page
\selectlanguage{spanish}

La adaptación de interfaces de usuario nos acompaña desde la aparición de los
primeros sistemas operativos basados en interfaces gráficas. Comenzando con
pequeñas opciones de personalización, más ligadas a la búsqueda de la practicidad
y eficiencia en la interacción, aparecieron los primeros menús y elementos
personalizables. Más adelante, se comenzó a considerar la posibilidad de romper
ciertas barreras de interacción provocadas por las posibles discapacidades de
los usuarios. De hecho, en ciertas situaciones el usuario puede sufrir 
discapacidades relacionadas con el contexto, denominadas temporales.  Así, 
se comenzaron a integrar y a desarrollar herramientas cuyo objetivo era romper 
o saltar estas barreras, buscando en todo caso permitir al usuario final un nivel 
de interacción suficiente.

Con la llegada de la telefonía móvil y los dispositivos portátiles, cada vez
más pequeños, wearables y con mayor capacidad de cómputo, esta tendencia de 
adaptación e inclusión de usuarios se ha trasladado también a este área. Sin
embargo, la adaptación de interfaces de usuario en estos dispositivos poco
tiene que ver con el caso de dispositivos estáticos. La calidad y configuración
del contexto que rodea al usuario puede tender a ser cambiante y, en cada caso,
necesitar de una adaptación o configuración concreta.

En esta tesis se hace frente a estos problemas. Para ello se presenta un sistema
de adaptación dinámico y móvil, basado principalmente en un modelo semántico de
usuario, contexto y dispositivo que permita la caracterización de dichas
entidades en un modelo dinámico que se abstraiga de capacidades o discapacidades
concretas y se centré así en las necesidades de interacción en cada momento.
Esto permitirá una mayor expresividad de las necesidades contextuales de
interacción del usuario, teniendo además en cuenta las características y
posibilidades que su dispositivo ofrezca.


\end{resumen}


% \begin{laburpena}        %this creates the heading for the abstract page
% \selectlanguage{basque}
% Jarri zure laburpena hemen.


% ...




% \end{laburpena}

%\end{abstractlongs}


% ---------------------------------------------------------------------- 
