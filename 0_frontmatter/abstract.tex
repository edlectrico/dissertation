
% Thesis Abstract -----------------------------------------------------


%\begin{abstractslong}    %uncommenting this line, gives a different abstract heading


\begin{abstracts}        %this creates the heading for the abstract page
\selectlanguage{british}
% Put your abstract or summary here.
Since the apparition of the first \acp{gui}, \acp{aui} have been used to cover
a wider range of possibilities of how to present the information. Starting 
with small personalization features, more related to practical interaction, 
first customizable menus and graphical elements arose. Subsequently, the 
possibility of breaking down the interaction barriers has grown. From the user
perspective those interaction limitations were caused by physiological disabilities.
These disabilities have impeded users to interact properly. In fact, the user may 
suffer from specific and temporary disabilities due to the context situation. 
Hence, adaptive tools started to be developed, included and integrated with the 
purpose of minimising these interaction barriers. This also aimed to allow users 
to feature a sufficient interaction experience.

The mobile telephony arrival and portable devices has brought a wider range of
possibilities regarding \acp{aui}. Everyday, this market evolves, bringing
smaller, more powerful and wearable devices. This trend has strengthen the \acp{aui}
to this market. Nevertheless, the user interface adaptation for these devices
is far from the advances in static devices, able to make complexer computations. 
Besides, more problems are added to the equation if the context is considered.
Its situation, characterized by the set of singular features which define it 
and their quality might make the context dynamic. Thus, in each case, different 
adaptations or configurations might be needed. 

In this dissertation these problems are faced, aiming to reduce the boundaries
between the user and the device in several limiting situations. To solve them, 
a dynamic and mobile user adaptation system is presented, principally supported 
by a semantic model including user, context and device. This model allows the 
characterization of these entities in a dynamic model which is abstracted from 
physiological user capabilities or disabilities, and thus centred on the user 
needs. This allows increasing the expressiveness of the user interaction 
contextual and interaction needs, also taking into account the features provided 
by the user's device.


\end{abstracts}

\begin{resumen} %this creates the heading for the abstract page
\selectlanguage{spanish}

La adaptación de interfaces de usuario nos acompaña desde la aparición de los
primeros sistemas operativos basados en interfaces gráficas. Comenzando con
pequeñas opciones de personalización, más ligadas a la búsqueda de la practicidad
y eficiencia en la interacción, aparecieron los primeros menús y elementos
personalizables. Más adelante, se comenzó a considerar la posibilidad de romper
ciertas barreras de interacción provocadas por las posibles discapacidades de
los usuarios. De hecho, en ciertas situaciones el usuario puede sufrir 
discapacidades relacionadas con el contexto, denominadas temporales.  Así, 
se comenzaron a integrar y a desarrollar herramientas cuyo objetivo era romper 
o saltar estas barreras, buscando en todo caso permitir al usuario final un nivel 
de interacción suficiente.

Con la llegada de la telefonía móvil y los dispositivos portátiles, cada vez
más pequeños, wearables y con mayor capacidad de cómputo, esta tendencia de 
adaptación e inclusión de usuarios se ha trasladado también a este área. Sin
embargo, la adaptación de interfaces de usuario en estos dispositivos poco
tiene que ver con el caso de dispositivos estáticos. La calidad y configuración
del contexto que rodea al usuario puede tender a ser cambiante y, en cada caso,
necesitar de una adaptación o configuración concreta.

En esta tesis se hace frente a estos problemas. Para ello se presenta un sistema
de adaptación dinámico y móvil, basado principalmente en un modelo semántico de
usuario, contexto y dispositivo que permita la caracterización de dichas
entidades en un modelo dinámico que se abstraiga de capacidades o discapacidades
concretas y se centré así en las necesidades de interacción en cada momento.
Esto permitirá una mayor expresividad de las necesidades contextuales de
interacción del usuario, teniendo además en cuenta las características y
posibilidades que su dispositivo ofrezca.


\end{resumen}


\begin{laburpena}        %this creates the heading for the abstract page
\selectlanguage{basque}
% Jarri zure laburpena hemen.


...




\end{laburpena}

%\end{abstractlongs}


% ---------------------------------------------------------------------- 
