% Thesis Acknowledgements ------------------------------------------------

% \begin{acknowledgements}      
%Long version
%uncommenting this line, gives a different acknowledgements heading
\begin{acknowledgementslong} 

Tantas personas que me vienen a la cabeza y tan pocas líneas para dedicarles
el agradecimiento que se merecen. Empezando, por supuesto, por aita, ama y Lander,
que siempre han estado y estarán ahí. A los aitas por darme la oportunidad de 
estudiar, de ser siempre libre de elegir, y de darme el apoyo moral y económico 
en tantos momentos, algunos de ellos muy complicados. Y a mi hermano Lander por 
ser como es, por ese desparpajo y esa cabezonería. Pero sobre todo por darme la 
posibilidad de reir día tras día. Y la verdad, en general a toda mi familia, 
siempre preocupados, siempre sacando el tema en los momentos más inoportunos, 
pero siempre orgullosos. Gracias a mis abuelos por haber tenido una familia tan 
increíble y habernos educado así. Me considero un tipo con suerte.

Y después de agradecer a mi familia, no puedo olvidarme de mis amigos. Espero
no dejarme a ninguno en el tintero. Como de alguna forma he de comenzar, elegiré 
el clásico método cronológico. Primero, gracias a la cuadrilla de Basauri. A Iván,
también compañero de cruzadas doctorales, a Iñaki y Molly, cuya boda en Wisconsin
no me perdería por nada del mundo, y a Israel y David, por compartir además muchas
mañanas de deporte. También gracias a Mikel e Imanol, por tantas escapadas y 
tantas aventuras. Mención especial para Roberto, siempre dispuesto, siempre al 
otro lado del teléfono para lo que fuera. Mil gracias tío. Punto y aparte para
la cuadrilla macarra, tipos que he conocido en los últimos 3 años y les puedo
considerar mis hermanos. A Ieltxu, por tantos viajes, conciertos, aventuras y 
desventuras. Por tantas horas en coche juntos, en avión, y en países extranjeros.
A Dani, empezando desde aquellos años de música prohibida a la tremenda 
amistad que nos une ahora. A Jon, por empezar compartiendo un pupitre durante 
tantos años en la universidad y acabar convirtiéndose en un gran amigo. A Endika, 
eterno, el tío más íntegro que conozco. Otro de estos que siempre está al otro lado
de un click de ratón o de la línea de teléfono\dots y al otro lado de la mesa,
compartiendo un buen trago. Compañero también de aventuras y barras de bar.
A Igor, Sara, Urko, Aritz y Esti, Luis\dots Mil gracias por ser como sois. Hasta
a Javi le voy a dar las gracias. Y saliéndome de Euskadi, gracias, Marina, sabes
que eres eres más que una amiga para mi. Y gracias Mique por haberla hecho así y
por acogerme bajo tu techo. Gracias por ayudarme a evadirme tantas veces, por
esos asados argentinos, y por esa cerveza de importación. Y no me olvido de ti,
Cata. Sos impresionante. Te escribo esto mientras escucho ``Ya despiértate nena,
sube al rayo al fin''. Malditos argentinos. Os quiero.

A nivel académico, gracias Diego por haberme dado esta oportunidad, por darme la
ocasión de trabajar en este maravilloso grupo de investigación. Gracias por tu
sinceridad y cercanía, por tener siempre la puerta abierta, por el tiempo
concedido, y por tu valioso feedback en tantas ocasiones. Y gracias, Aitor Almeida,
por servirme de guía, de apoyo, por tantos capones y tanta dedicación y paciencia.
Gracias por las razones que te llevaron a co-dirigir esta tesis. Desde luego no
lo habría conseguido sin este apoyo tan sincero desde el primer día.

Gracias, compañeros del grupo MORElab. Por ayudarme en los momentos malos y por
ofrecerme tantos momentos buenos. Gracias a los senior, Aitor, Pablo y Unai, por
representar la voz de la experiencia en tantas ocasiones. Y gracias a los demás
doctorandos compañeros de batalla, especialmente a Aitor Gómez-Goiri, Iván Pretel
y Xabier Eguiluz, porque recorriendo este camino de la mano se ha hecho menos 
duro. Gracias a los demás compañeros por amenizar cada día en el laboratorio, 
y gracias a aquellos que os marchásteis y me regalásteis momentos similares.

Thanks to all the inspiring people I met at University of Ulster in Belfast.
Specially to Dr. Luke Chen and Professor Christopher Nugent, who helped and
guided me with their experience and wisdom. I would like to thank every single
researcher and teacher I had the opportunity to talk with in Belfast. I will
never forget such a welcoming, aid, feedback and support.

En definitiva, gracias a todos. A aquellos que me habéis apoyado, soportado,
aguantado, torturado, y maldecido. Todos sois un poco responsables de este 
trabajo. 


% Abuelo Pepe, abuelo Juan, si sirve de algo, esto va para vosotros.

\begin{flushright}
\textit{Eskerrik asko,}

Eduardo Castillejo

% Moth and year
\monthname \ \the\year



% Signature figure

%\begin{figure}[htbp!]
%\end{figure}
%\includegraphics{signature}%



\end{flushright}



%Closing of the acknowledgements
%Sort version
% \end{acknowledgements}
% Long version
\end{acknowledgementslong}

% ------------------------------------------------------------------------


