
\chapter{The AdaptUI Rules}
\label{cha:appendixB}

This appendix lists all the included default rules in AdaptUIOnt. The rules
are classified taking into account their participation in the adaptation process.
Thus, the pre-adaptation rules set are is detailed in Table~\ref{tbl:pre_adaptation_rules}.
Next, Table~\ref{tbl:adaptation_rules} lists the adaptation rules. Table~\ref{tbl:usability_rules}
shows the usability rules. Finally, the post-adaptation rules are detailed in 
Table~\ref{tbl:post_adaptation_rules}. 

\begin{table}
  \caption{The AdaptUIOnt pre-adaptation rules.}
 \label{tbl:pre_adaptation_rules}
\footnotesize
\centering
 \begin{tabular}{l l l}
  \hline 
  \textbf{Pre-adaptation rule} 			& \textbf{Involved ontologies} 	& \textbf{Description} 	\\
						& \textbf{and classes} 		& 			\\
  \hline
  \textit{checkBatteryLevelIsSufficient}& \textit{soupa:Battery}	& Considering Table~\ref{tbl:batteries} this rule 	\\
					& \textit{adaptui:Device}	& evaluates if the current battery 			\\
					& \textit{adaptui:DeviceAux}	& level is enough to perform any 			\\
					&				& adaptation. 						\\
  \hline
  
  \textit{checkLightLevelDarkOvercast}	& \textit{adaptui:Context}	& These rules evaluate the context   			\\
  \textit{checkLightLevelTwilight}	& \textit{adaptui:ContextAux}	& light input through sensors by  			\\
  \textit{checkLightLevelSunrise}	& \textit{umo:Light}		& using the classification shown in 			\\
  \textit{checkLightLevelOvercast}	& 				& Table~\ref{tbl:luminance}. The result of this 	\\
  \textit{checkLightLevelOfficeLightning}&				& evaluation would be a verbose  			\\
  \textit{checkLightLevelOfficeHallway}	&				& concept (e.g., \textit{sunrise}) which is   		\\
  \textit{checkLightLevelMoonlessClear}	&				& stored in the ontology using the  			\\
  \textit{checkLightLevelDayLight}	& 				& \textit{contextAuxHasLightLevel} datatype		\\
  \textit{checkLightLevelDirectSunlight}& 				&  property.						\\
  \textit{checkLightLevelFullMoon}	& \\
  \textit{checkLightLevelLivingRoom}	& \\
  \hline
  
  \textit{checkNoiseLevelBreathing}	& \textit{adaptui:Context}	& These rules evaluate the context 	 		\\
  \textit{checkNoiseLevelTruck}		& \textit{adaptui:ContextAux}	& noise input through sensors by  			\\
  \textit{checkNoiseLevelTrain}		& \textit{umo:Noise}		& using the classification  shown in 		 	\\
  \textit{checkNoiseLevelTraffic}	& 				& Table~\ref{tbl:sounds}. The result of this 	 	\\
  \textit{checkNoiseLevelOffice}	&				& evaluation would be a verbose 		 	\\
  \textit{checkNoiseLevelLibrary}	&				& concept (e.g., \textit{sunrise}) which is  		\\
  \textit{checkNoiseLevelLeavesMurmuring}&				& stored in the ontology using the 			\\
  \textit{checkNoiseLevelBreathing}	&				& \textit{contextAuxHasNoiseLevel} datatype		\\
  \textit{checkNoiseLevelBuildingWork}	&				& property.						\\
  \textit{checkNoiseLevelConversation}	&\\
  \textit{checkNoiseLevelFactory}	&\\
  \textit{checkNoiseLevelGig}		&\\
  \textit{checkNoiseLevelJackhammer}	&\\
  \hline
  
  \textit{checkUserHasAttentionRestriction}& \textit{adaptui:Activity}	& If the current activity impedes the 		 	\\
  \textit{checkUserHasNoRestriction}	& \textit{adaptui:UserAux}	& user, these rules would store the 			\\
  \textit{checkUserHasSightRestriction}	&				& corresponding boolean value in the 			\\
  \textit{checkUserHasMovementRestriction}&				& \textit{userAuxHasRestriction} datatype 		\\
  \textit{checkUserHasHandsRestriction}	&				& property.						\\
  \textit{checkUserHasHearingRestriction}&\\
%   Usability \\
%   Polishing\\
  \hline
\end{tabular}
\end{table}


\begin{table}
  \caption{The AdaptUIOnt adaptation rules.}
 \label{tbl:adaptation_rules}
\footnotesize
\centering
 \begin{tabular}{l l l}
  \hline 
  \textbf{Adaptation rule} 	& \textbf{Involved ontologies} 	& \textbf{Description} 	\\
				& \textbf{and classes} 		& 			\\
  \hline
  \textit{Adapt*}		& 				& The adaptation rules are triggered once the 	\\
				&				& checking rules and the user, context and	\\
				&				& device characteristics are stored.		\\
  \hline
\end{tabular}
\end{table}


\begin{table}
  \caption{The AdaptUIOnt usability rules.}
 \label{tbl:usability_rules}
\footnotesize
\centering
 \begin{tabular}{l l l}
  \hline 
  \textbf{Adaptation rule} 	& \textbf{Involved ontologies} 	& \textbf{Description} 	\\
				& \textbf{and classes} 		& 			\\
  \hline
  \textit{CheckEffectivityMetric*}&				& These set of rules check the cited usability metrics 		\\
				&				& shown in Table~\ref{tbl:effectiveness_metrics}.		\\
  \textit{CheckProductivityMetric*}&				& These set of rules check the cited usability metrics 		\\
				 &				& shown in Table~\ref{tbl:productivity_metrics}.		\\
  \hline
\end{tabular}
\end{table}

\begin{table}
  \caption{The AdaptUIOnt post-adaptation rules.}
 \label{tbl:post_adaptation_rules}
\footnotesize
\centering
 \begin{tabular}{l l l}
  \hline 
  \textbf{Adaptation rule} 	& \textbf{Involved ontologies} 	& \textbf{Description} 	\\
				& \textbf{and classes} 		& 			\\
  \hline
  \textit{CheckEffectivityMetric*}&				& These set of rules check the cited usability metrics 		\\
				& 				& shown in Table~\ref{tbl:effectiveness_metrics}.		\\
  \textit{CheckProductivityMetric*}&				& These set of rules check the cited usability metrics 		\\
				&				& shown in Table~\ref{tbl:productivity_metrics}.		\\
  \hline
\end{tabular}
\end{table}
