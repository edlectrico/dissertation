\subsection{Developers using AdaptUI}
\label{sec:developers}

Before introducing the results obtained from working with final users (as 
consumers of adapted user interfaces), an experiment with developers has been 
performed. This experiment aims to evaluate the provided \acp{api}, and also to 
compare the differences between performing a configuration of the user interface 
in Android and in AdaptUI. Thus, developers with experience in both Android
development and semantics are involved.

As AdaptUI provides an \ac{api} conceptually divided in two (see 
Section~\ref{sec:application_layer}), this experiment has been fragmented 
accordingly. The experiment has been presented as follows:

\begin{enumerate}
  \item First, developers are introduced to the AdaptUI framework through a brief
  explanation of AdaptUI's virtues and boundaries. Both \acp{api} are described
  and explained to them.
  
  \item Next, a Java project build in Eclipse \ac{ide} is presented. This project
  uses AdaptUI as a library, and the main class lists the code shown in
  Listing~\ref{lst:default_knowledge_experiment}. This class has an AdaptUI
  declaration and initialization and also several \textit{TODO} tasks with
  the corresponding instructions. 
  
  \inputminted[linenos=true, fontsize=\footnotesize, frame=lines]{java}{5_experiments_and_results/default_knowledge_experiment.java}
  \captionof{listing}{The default main class.\label{lst:default_knowledge_experiment}}

  \item Once the corresponding knowledge adaptations are made, the resulting
  ontology is stored and show to developers using Protégé. Figure~\ref{fig:protege}
  shows an example of a modified ontology by a participating developer.

  \begin{figure}
  \centering
  \includegraphics[width=0.85\textwidth]{protege.pdf}
  \caption{A test ontology modified by a developer. Several classes have been
  added. Also the \textit{viewBackgroundColor} for the \textit{Button} only 
  instance has been modified.}
  \label{fig:protege}
  \end{figure}
  
  \item After modifying the ontology the AdaptUIDemo Android project is launched.
  This project's main activity shows a text edit, an edit text and a button, and
  its goal is to dynamically modify these views using the changes in the ontology
  performed by the developer. In order to make this application use the new modified
  version of the ontology an \ac{adb} command is needed. This command pushes the
  required file from the laptop disk to the desired Android folder. 
  Listing~\ref{lst:adb_push} shows the mentioned command.
  
  \inputminted[linenos=true, fontsize=\footnotesize, frame=lines]{java}{5_experiments_and_results/adb_push.java}
  \captionof{listing}{The \ac{adb} push command. The first parameter refers to the
  file to be sent to the device. The second parameter points at the absolute location
  in the device. \label{lst:adb_push}}
  
  \item Next, once the developer checks that the adaptation to the views shown
  in the demo application matches the modifications performed with the knowledge 
  \ac{api}, the developer is asked to use the adaptation \ac{api} through an
  Android project. Similar to the first case, the application is already developed,
  and the developer is asked to fulfil several tasks (see 
  Listing~\ref{lst:default_adaptation_experiment}).

  \inputminted[linenos=true, fontsize=\footnotesize, frame=lines]{java}{5_experiments_and_results/default_adaptation_experiment.java}
  \captionof{listing}{The default Android version with the corresponding task to
  be completed by the developer. \label{lst:default_adaptation_experiment}}
  
  \item Finally, the developer uses the Capabilities Collector to adapt several
  views. As the last task of the Capabilities Collector is to store the required
  values in the ontology, the developer checks these changes again in the AdaptUIDemo
  application.
  
\end{enumerate}



% The following sections aim to give more details about the experiments defined
% above.
% 
% \subsubsection{Using the knowledge editor \ac{api}}
% \label{sec:knowledge_editor_api}
% 
% The knowledge editor \ac{api} provides a set of methods to modify the knowledge
% stored in the ontology. Imported as a library from a Java project, this \ac{api}
% allows to add and remove classes, individuals, object and datatype properties,
% and rules. In this experiment developers are asked to modify the ontology by
% adding new concepts to be used by the adaptation \ac{api}.
% Listing~\ref{lst:adaptui_add_knowledge} shows an example of how developers can
% modify the knowledge stored in the ontology.
% 
% \inputminted[linenos=true, fontsize=\footnotesize, frame=lines]{java}{5_experiments_and_results/adaptui_add_knowledge.java}
% \captionof{listing}{Modifying the knowledge represented in the ontology through the knowledge \ac{api}.\label{lst:adaptui_add_knowledge}}
% 
% \subsubsection{Using adaptation \ac{api}}
% \label{sec:adaptation_api_experiment}
% 
% As mentioned in Section~\ref{sec:adaptation_api}, the adaptation \ac{api} provides 
% a set of methods whose purpose is to facilitate the adaptation process for 
% developers. The purpose of this part of the experiment is to check developer's 
% feedback when dealing with the AdaptUI framework by collecting their opinions.
% To do so, the experiment presents an Android application developed with the 
% Eclipse\footnote{https://eclipse.org/} \ac{ide}. It includes a layout configuration 
% \ac{xml} file with the corresponding declaration of several user interface views
% (see Listing~\ref{lst:default_layout}), and also an \textit{onCreate} main method, 
% shown in Listing~\ref{lst:default_oncreate} (the default aspect of the application
% is illustrated through Figure~\ref{fig:default_layout}). The developers 
% participating in this experiment are guided through a brief explanation of the 
% available adaptation \ac{api} methods. Once they are aware of the possibilities 
% the framework provides, they are asked to modify the aspect of the listed views.
% 
% \inputminted[linenos=true, fontsize=\footnotesize, frame=lines]{xml}{5_experiments_and_results/default_layout.xml}
% \captionof{listing}{The default layout defining a grid layout, a text view,
% a button and an edit text.\label{lst:default_layout}}
% 
% \inputminted[linenos=true, fontsize=\footnotesize, frame=lines]{java}{5_experiments_and_results/default_oncreate.java}
% \captionof{listing}{The default \textit{onCreate} method.\label{lst:default_oncreate}}
% 
% \begin{figure}
% \centering
% \includegraphics[width=0.25\textwidth]{default_layout.png}
% \caption{The corresponding user interface considering the layout specified in
% Listing~\ref{lst:default_layout}.}
% \label{fig:default_layout}
% \end{figure}
% 
% Obviously, in order to adapt the application, developers need a copy of the
% AdaptUIOnt ontology, in which the corresponding values for the adaptation are
% semantically represented. Hence, developers are asked to install the Capabilities 
% Collector in their Android devices. Thus, they can add new knowledge to the ontology.
% 
% Listing~\ref{lst:default_oncreate} shows the presented \textit{onCreate} method, 
% which defines the views listed in Listing~\ref{lst:default_layout}. Using AdaptUI 
% developers reach the \textit{onCreate} shown in Listing~\ref{lst:adaptui_oncreate},
% which includes the corresponding calls to the provided adaptation \ac{api}.
% 
% \inputminted[linenos=true, fontsize=\footnotesize, frame=lines]{java}{5_experiments_and_results/adaptui_oncreate.java}
% \captionof{listing}{The AdaptUI \textit{onCreate} method.\label{lst:adaptui_oncreate}}
% 
% Thus, the resulting adapted user interface is illustrated by Figure~\ref{fig:adapted_layout}.
% While Figure~\ref{fig:default_layout} shows a default disposition and configuration
% of the user interface items described in Listing~\ref{lst:default_layout}, in this
% case they have been adapted according to Listing~\ref{lst:adaptui_oncreate}.
% 
% \begin{figure}
% \centering
% \includegraphics[width=0.25\textwidth]{adapted_layout.png}
% \caption{The corresponding user interface considering the layout specified in
% Listing~\ref{lst:adaptui_oncreate}.}
% \label{fig:adapted_layout}
% \end{figure}





\subsubsection{Results and conclusions}
\label{sec:api_results}