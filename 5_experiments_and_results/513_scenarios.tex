\subsection{Scenarios}
\label{sec:scenarios}

In the previous section several quantitative experiments have been presented.
The experiments have illustrated the performance differences considering the 
required time to manage the AdaptUIOnt ontology and Imhotep. Regarding the 
ontology itself, AdaptUI takes into account many context situations, user's 
capabilities and device's characteristics. This makes difficult to evaluate 
every one of these settings in real environments. In this section, several 
hypothetical scenarios are presented simulating the behaviour of the analysed 
adaptive systems in more complex situations.

The scenarios are included in the evaluation of AdaptUI compared to Imhotep.
Consequently, both adaptation processes are taken into account for the 
resulting adaptation and the presented scenario conditions. Hence, the scenarios
represent three stereotypical situations which might require a user interface 
adaptation:

\begin{itemize}
  \item First, in the first scenario the situation is characterized by several
  context parameters that limit certain user capabilities. Consequently, 
  AdaptUI would have to consider the user as an updated version of 
  himself/herself, taking into account the new set of capabilities that 
  he/she has due to the context situation.
  
  \item In the second scenario, a situation in which the user capabilities are 
  delimited by a set of tasks that form an activity is presented. The user is 
  impeded to act normally. Therefore, the AdaptUI platform would have to be 
  aware of this limitation. 
  
  \item The last scenario simulates a situation where the user suffers from a 
  certain disability, so the platform should adapt the user interface 
  accordingly.
\end{itemize}

Each scenario is presented as follows:

\begin{itemize}
  \item First, an introduction to the problem to be evaluated is described.
  
  \item Secondly, the scenario situation is presented. It is also summarized with
  a table indicating the main characteristics of the three main entities.
  
  \item Thirdly, the adaptation process by Imhotep and AdaptUI is detailed.
  
  \item Finally, each scenario presents several conclusions after a brief 
  discussion.
\end{itemize}

\subsubsection{Scenario 1: Limitations Caused by Context Conditions}
\label{sec:scenario1}

This scenario introduces several limitations that context might induce to 
certain user capabilities. As context is considered as a significant entity in 
AdaptUI, this scenario aims to solve a situation in which it impedes the user to
interact properly. On the contrary, Imhotep does not include the context 
situation for adapting the user interface. Thus, it would not be possible to 
have a coherent result regarding the proposed scenario. The following 
lines introduce the cited scenario:

John is going home after work. He lives and works in Rovaniemi, one of the
northernmost and coldest regions in Finland. As can be seen in 
Table~\ref{tbl:scenario1}, John does not suffer from any disability. He removes
his gloves and then he proceeds to send a \ac{sms} to his wife. The current 
temperature is -10~ºC.

\begin{table}
 \caption{Scenario 1 situation summary.}
 \label{tbl:scenario1}
 \footnotesize
 \centering
\begin{tabular}{l l}
  \hline 
				& \textbf{Scenario 1} 		\\
  \hline
  User \\
  \qquad - Personal data 	& John, 36 years old, Finnish 	\\
  \qquad - Activity	 	& Sending a \ac{sms}		 	\\
  \qquad - Known disabilities 	& None			 	\\
%   \hline
  Context \\
  \qquad - Location 		& Relative: Rovaniemi, Finland	\\
				& Absolute: 66.497109, 25.724977\\
  \qquad - Time			& 18:35				\\
  \qquad - Brightness		& 600 \ac{lx}			\\
  \qquad - Temperature		& -10 ºC			\\
%   \hline
  Device 			& Samsung Galaxy SIII		\\
				& Battery: 85\%			\\
  \hline
\end{tabular}
\end{table}

AdaptUI covers this situation through the following steps. First, the platform
considers through the user model that John does not suffer from any 
disabilities. Analysed by the Capabilities Collector, AdaptUI does not find 
disabilities that would limit future adaptations. 
Table~\ref{tbl:user_profile_scenario1} shows the semantic representation of the 
model, carried out by the Semantic Modeller, that fits John’s profile. 

\begin{table}
 \caption{User profile for Scenario 1.}
 \label{tbl:user_profile_scenario1}
 \footnotesize
 \centering
\begin{tabular}{l l l }
  \hline 
  \textbf{Class}  & \multicolumn{2}{c}{\textbf{Scenario 1}}		\\
		  & \textbf{Data property} & \textbf{value} 		\\
  \hline
  \textit{Display}& \textit{userDisplayBrightnessIsStatic*} & true	\\
		  & \textit{userDisplayIsApplicable}	    & true	\\
  \textit{Audio}  & \textit{userDisplayApplicableIsStatic}  & false	\\
		  & \textit{userAudioHasApplicable} 	    & true 	\\
		  & \textit{userAudioApplicableIsStatic}    & false 	\\
		  & \textit{userAudioHasVolume}  	    & 4 	\\
  \textit{Interface}& \textit{userInterfaceInput}	    & haptic	\\
		  & \textit{userInterfaceOutput} 	    & default	\\
  \textit{Experience}& \textit{userHasExperience} 	    & high	\\
  \textit{View}	  & \textit{userViewIsStatic}		    & false	\\
  \textit{Other}  & \textit{userHasLanguage}		    & Finnish	\\
		  & \textit{vibration} 			    & true 	\\
  \hline
\end{tabular}
\end{table}

Regarding John’s model, it can be seen how it is configured as an open model for
future adaptations. Besides, most of the preferences are configured as default. 
The only property that is limited from the model is the brightness, which John
has established as a static property. This means that AdaptUI understands that
this property should not be adapted no matter what the context situation is.
As the contrast and the volume are not static, the modelled integer values are
considered as preference values and not a limitation for the rules. This means
that John prefers a volume level of 4, but he allows the system to manage it if
certain conditions are met. These integer values are different depending on the
used platform. In this case, as these experiments run over the Android platform,
they refer to Android values~\citep{android_volume}. 

Table~\ref{tbl:user_profile_scenario1} shows the semantic representation of 
this model which fits John’s profile, using the AdaptUI ontology detailed in 
Chapter~\ref{cha:ontology_model}. The asterisk marks a property that is marked by the 
user in a way that limits the adaptation. In this case, the 
\textit{userDisplayBrightnessIsStatic*} is configured as \textit{true} during 
the interaction between the user and the Capabilities Collector. Thus, although 
the context light might affect the sight capability of the user, the display 
brightness will not be adapted. Combining the user model from 
Table~\ref{tbl:user_profile_scenario1} with the current context situation
depicted in Table~\ref{tbl:scenario1}, the pre-adaptation rules generate the
\textit{UserAux} profile for John, which is shown in 
Table~\ref{tbl:userAux_scenario1}. In this case, there is a temporary 
restriction for the user due to the current context conditions (freezing 
temperature) that is modelled through the \textit{hasRestriction} datatype 
property (with the value \textit{hands\_restriction}). 

\begin{table}
 \caption{UserAux class generated by the pre-adaptation rules set and resulting
 user interface first adaptation for both scenarios.}
 \label{tbl:userAux_scenario1}
 \footnotesize
 \centering
\begin{tabular}{l l}
  \hline 
  \multicolumn{2}{c}{\textbf{Scenario 1}}	\\
  \textbf{UserAux} 	& \textbf{Value}	\\
  \hline
  textit{hasDisplayApplicable} 	& true			\\
  textit{hasDisplayBrightness}	& false			\\
  textit{hasRestriction}	& hands\_restriction	\\
  \hline
\end{tabular}
\end{table}

The final adaptation for this situation is driven by the adaptation rules set.
The brightness is not adapted because the property 
\textit{userDisplayBrightnessIsStatic} from the Brightness class is configured 
as \textit{true} (see Table~\ref{tbl:user_profile_scenario1}). Besides, an 
increase of 10 pixels of the View size is needed to cover the situation in 
which the user makes more errors on the touch screen because of the freezing
temperature. An extra interaction feedback is added (vibration). The final 
adaptation is shown in Table~\ref{tbl:final_adaptation_scenario1}:

\begin{table}
 \caption{Final adaptation for the Scenario 1.}
 \label{tbl:final_adaptation_scenario1}
 \footnotesize
 \centering
\begin{tabular}{l l}
  \hline 
  \multicolumn{2}{c}{\textbf{Scenario 1}}\\	
  \textbf{Adaptation} 	& \textbf{Value}\\ 
  \hline
  \textit{hasBrightness}& -		\\
  \textit{hasColourSet}	& -		\\
  \textit{hasViewSize}	& 10		\\
  \textit{hasResponse}	& vibration	\\
  \hline
\end{tabular}
\end{table}

Finally, the system should provide the user a chance to evaluate, provide 
feedback, on whether the interaction with the presented adaptation is usable 
enough or if another adaptation should be triggered. There are also a few rules 
which take into account the device’s dynamic capabilities. The remaining 
memory, available processor and battery are monitored in order to avoid 
adaptations that could freeze the device or consume the remaining power. As is 
shown in Table~\ref{tbl:battery}, battery levels under 10\% are considered too 
low to perform an adaptation.

\begin{table}
 \caption{Battery percentage and corresponding ontology values.}
 \label{tbl:battery}
 \footnotesize
 \centering
\begin{tabular}{l l}
  \hline 
  \textbf{Battery (\%)} & \textbf{Ontology values}	\\
  \hline
   $x \leq$ 10 		& \textit{not\_sufficient}	\\
   $x$ 10 $< x \leq$ 50 	& \textit{sufficient} 		\\
   $x$ 50 $< x \leq$ 100 	& \textit{optimal}		\\
  \hline
\end{tabular}
\end{table}

\myparagraph{Discussion}
\label{sec:scenario1_discussion}

This first scenario presents a situation in which the user capabilities are
limited by certain context characteristics. More specifically, the current
temperature (-10~ºC) might make difficult for John to use his hands and properly
interact with his device. In this case, thanks to the context modelling 
availability in AdaptUIOnt and due to the configuration of the user profile, 
several sets of rules lead to the results of Table~\ref{tbl:userAux_scenario1},
in which we can see how the pre-adaptation rules evaluate several 
characteristics of the user conditions, and the results of 
Table~\ref{tbl:final_adaptation_scenario1}, in which the final adaptation is 
shown.

On the contrary, Imhotep does not take context conditions into account. The only
way to indicate a change on context would be if the user configures the Imhotep 
user profile with a concrete disability. This would lead to a series of 
configurations each time the user is within a different environment.

\myparagraph{Conclusions}
\label{sec:scenario1_conclusions}

John is perfectly healthy and he usually does not need complex adaptations. 
However, in this case the current freezing temperature makes difficult for him
to use his fingers with the usual precision. Considering these premises, any 
adaptive system would generally ignore the current situation (including Imhotep).
Nevertheless, with these weather conditions, John is partially disabled. The
weather, more concretely the temperature, might affect several physiological
capabilities of the user. Therefore, the interaction between John and the \ac{sms}
application might experience several difficulties.

Imhotep does not take into account the context situation, but we can consider 
the new set of capabilities shown in Table~\ref{tbl:user_profile_scenario1} as 
the starting user capabilities. Besides, regarding
Table~\ref{tbl:final_adaptation_scenario1} where the adaptation from AdaptUI is 
shown, Imhotep would not make an adequate adaptation. This is not just because 
the user has not a permanent disability. Imhotep considers through the user 
profile sight problems and hearing problems. Considering that the profile is 
correctly configured, an adaptation in such ways will not cover the user needs, 
which deals with mobility and precision using fingers on  the screen. 
Table~\ref{tbl:adaptui_vs_imhotep_scenario1} shows several significant 
differences between those characteristics that Imhotep is able to take into 
account in this scenario and the ones that AdaptUI considers. It is shown how 
Imhotep unawareness of context features makes it inaccurate when adapting the 
user interface to the current situation. Besides, it is not able to model an 
activity, but the consequences of it (as AdaptUI, which takes into account that 
the user is performing an activity that, due to the context conditions, cannot 
carry out properly).

\begin{table}
 \caption{AdaptUI and Imhotep comparison of the final reached adaptation for 
the Scenario 1.}
 \label{tbl:adaptui_vs_imhotep_scenario1}
 \footnotesize
 \centering
\begin{tabular}{l l c c}
\hline
\textbf{Entity} & \textbf{Capability or}&\multicolumn{2}{c}{\textbf{Solution}}\\
		& \textbf{characteristic}& \textbf{Imhotep} & \textbf{AdaptUI}\\
\hline
User		& Sight			 & 		    & 		      \\
		& Hearing 		 & 		    & 		      \\
		& Other   		 & 		    & \checkmark      \\
		& Activity		 & 		    & 		      \\

Context		& Light			 & 		    &		      \\
		& Noise			 & 		    &		      \\
		& Temperature		 & 		    & \checkmark      \\
	
Device		& Resolution		 & \checkmark	    & \checkmark      \\
		& Battery		 & 		    & \checkmark      \\
\hline
\end{tabular}
\end{table}

\subsubsection{Scenario 2: Limitations Caused by Activities}
\label{sec:scenario2}

In this case, the scenario 2 presents a situation in which the user cannot act
properly due to the activity or activities that he/she is doing. Again, Imhotep
cannot characterize activities. Nevertheless, in this scenario user disabilities
and device characteristics are going to be used in Imhotep to follow the 
corresponding user interface adaptation. The scenario 2 is described now in the 
following lines:

Karen is a 60 years old woman with several disabilities due to her ageing. She 
is colour blind, which means that she has problems distinguishing several 
colours. In addition, she suffers from a light hearing loss, which is not severe. 
She is driving to down town using a \ac{gps} mobile application. For this 
scenario, the most significant information to infer is the impossibility to use 
the hands and the impossibility to distract her, as Karen is driving. Besides, 
the user model should consider the capabilities cited in Table~\ref{tbl:scenario2}, 
as she is colour blind. Therefore, with this information a first user model is 
semantically represented by AdaptUI in Table~\ref{tbl:user_profile_scenario2}. 
The following lines analyse the adaptation process due to the situation 
summarized in Table~\ref{tbl:scenario2}.

\begin{table}
 \caption{Scenario 2 situation summary.}
 \label{tbl:scenario2}
 \footnotesize
 \centering
\begin{tabular}{l l}
  \hline 
				& \textbf{Scenario 2}		\\
  \hline
  User \\
  \qquad - Personal data 	& Karen, 60 years old, German \\
  \qquad - Activity	 	& Driving 			\\
  \qquad - Known disabilities 	& Colour blindness 		\\
				& Hearing loss 			\\
%   \hline
  Context \\
  \qquad - Location 		& Relative: Berlin, Germany  	\\
				& 				\\
  \qquad - Time			& 11:10 			\\
  \qquad - Brightness		& 1,100 \ac{lx} 		\\
  \qquad - Temperature		& 15 ºC 			\\
%   \hline
  Device 			& Samsung Galaxy Tab 	 	\\
% 				& 				\\	
  \hline
\end{tabular}
\end{table}

Karen suffers from two different disabilities: colour blindness and hearing 
loss. As AdaptUIOnt is focused on the user’s preferences rather than on the 
disabilities, the user model configures her capabilities as is shown in 
Table~\ref{tbl:user_profile_scenario2}. 

\begin{table}
 \caption{User profile for Scenario 2.}
 \label{tbl:user_profile_scenario2}
 \footnotesize
 \centering
\begin{tabular}{l l l}
  \hline 
  \textbf{Class} & \multicolumn{2}{c}{\textbf{Scenario 2}}		\\
	  & \textbf{Data property} 		   & \textbf{value}	\\
  \hline
  Display & \textit{userDisplayBrightnessIsStatic} & false		\\
	  & \textit{userDisplayIsApplicable} 	   & true		\\
  Audio   & \textit{userDisplayApplicableIsStatic}& true		\\
	  & \textit{userAudioHasApplicable} 	   & true		\\
	  & \textit{userAudioApplicableIsStatic}   & false		\\
	  & \textit{userAudioHasVolume}  	   & 4 		\\
 Interface& \textit{userInterfaceInput}		   & default		\\
	  & \textit{userInterfaceOutput} 	   & default		\\
Experience& \textit{userHasExperience} 		   & low		\\
  View	  & \textit{userViewIsStatic}		   & false		\\
  Other   & \textit{userHasLanguage}		   & German		\\
	  & \textit{vibration} 			   & true 		\\
  \hline
\end{tabular}
\end{table}

Table~\ref{tbl:user_profile_scenario2} presents several important user 
parameters regarding the final adaptation. The volume of the device is allowed
to be adapted by the platform. In addition, Karen uses the default 
input/output interaction channels. In this case, there is a temporary 
restriction for the user due to the current activity (i.e., driving) that is 
modelled through the \textit{hasRestriction} datatype property (with the value 
\textit{hands\_restriction} and \textit{attention\_restriction}). This is shown
in Table~\ref{tbl:userAux_scenario2}.

\begin{table}
 \caption{UserAux class generated by the pre-adaptation rules set and resulting
 UI first adaptation for both scenarios.}
 \label{tbl:userAux_scenario2}
 \footnotesize
 \centering
\begin{tabular}{l l}
  \hline 
  \multicolumn{2}{c}{\textbf{Scenario 2}}			\\
  \textbf{UserAux} 		& \textbf{Value} 		\\
  \hline
  \textit{hasDisplayApplicable}	& \textit{true}			\\
  \textit{hasAudioApplicable}	& \textit{true} 		\\
  \textit{hasRestriction}	& \textit{hands\_restriction}	\\
  ~				& \textit{attention\_restriction}\\
  \hline
\end{tabular}
\end{table}

The final adaptation for this situation is modelled by the adaptation rules.
As the limitations in this case come from the special situation generated by 
the activities performed by the user, AdaptUI has to change the interaction 
channels considering that the user has several restrictions. Thus, the size of 
the corresponding views is increased (not because Karen's colour blindness, but
because of the activity of driving), the interaction channel allows voice 
interaction, and the volume is also increased due to the possible noise while
driving a car or traffic. These adaptations are shown in 
Table~\ref{tbl:final_adaptation_scenario2}:

\begin{table}
 \caption{Final adaptation for the Scenario 2.}
 \label{tbl:final_adaptation_scenario2}
 \footnotesize
 \centering
\begin{tabular}{l l}
  \hline 
    \multicolumn{2}{c}{\textbf{Scenario 2}}		\\
    \textbf{Adaptation} 	& \textbf{Value} 	\\
    \hline
    \textit{hasColourSet}	& Colour blindness 	\\
    \textit{hasViewSize}	& 20 			\\
    \textit{hasInput}		& Voice and haptic	\\
    \textit{hasOutput}		& Visual and audio	\\
    \textit{hasVolume}		& 7 (max) 		\\
  \hline
\end{tabular}
\end{table}

As said before, Imhotep does not consider activities in its user and device 
profile. However, we can assume a possible user configuration profile under the 
presented circumstances in Table~\ref{tbl:scenario2}. Thus, a possible profile
is illustrated through Listing~\ref{lst:scenario2_imhotep}.


\inputminted[linenos=true, fontsize=\footnotesize, frame=lines]{json}{5_experiments_and_results/scenario2_imhotep.json}
\captionof{listing}{The variables file, in which device characteristics and user 
capabilities are described for the scenario 2.\label{lst:scenario2_imhotep}}


Regarding Listing~\ref{lst:scenario2_imhotep} the Imhotep framework will avoid
combining red/green colours, but no magnification of the screen or the size
of the text would be added. Voice control will also be provided, and volume
levels will be increased by indicating a hearing disability.


\myparagraph{Discussion}
\label{sec:scenario2_discussion}

In this scenario a situation in which the user capabilities are limited by the
current activity is presented. Karen is driving in this scenario, and this 
activity makes her unable to use the usual interaction channel with the device. 
This activity normally requires a full attention to the road and the 
use of both hands. Therefore, we find two different limitations that 
this scenario combine. Considering this situation AdaptUI models several 
restrictions through the pre-adaptation rules. These restrictions and their 
representation in the AdaptUIOnt ontology are shown in 
Table~\ref{tbl:userAux_scenario2}. The final adaptation for this situation is 
shown in Table~\ref{tbl:final_adaptation_scenario2}. In this case the main 
adaptations are centred in the interaction channel, as the user is driving and 
suffers from several restrictions due to the current activity. Hence, audio 
control and bigger views are presented.

As happens in Scenario 1, Imhotep does not cover the circumstances described 
for this scenario through Table~\ref{tbl:scenario2}. However, it is possible to
somehow represent this through the user profile variables. 
Listing~\ref{lst:scenario2_imhotep} shows an example of a modified user profile
representing several limitations that, although they might not be true, they
represent a practical situation in which the user is limited by external agents 
(which would be the activity of driving). Thus, Imhotep considers that the user 
is blind and suffers from hearing loss. Consequently, the resulting user interface
also increases the volume levels, uses \ac{tts} for reading the displayed
information and changes the colour set.

\myparagraph{Conclusions}
\label{sec:scenario2_conclusions}

Karen suffers from two disabilities which, combined with the current activity
makes the interaction with her mobile device troublesome. In this scenario, 
Karen is driving her car and intends to use a \ac{gps} guiding application to go to
a certain location. AdaptUI covers this situation adapting the interaction 
channel and through a few modifications in the displayed views. On the contrary, 
to be able to represent a similar situation with Imhotep, the user variables 
profile has to be modified.
% As this 
% interaction might be difficult, the Adaptation Polisher module explained in
% Section~\ref{sec:adaptation_polisher} takes great importance in this situation.

The main differences between both solutions is shown when Karen interacts with
the application and the results are not satisfactory. This means that, although
the application's user interface has been adapted, it might not be close to what
the user needs. In AdaptUI the Adaptation Polisher dynamically monitors this 
interaction and performs several changes under the post-adaptation rules set. On
the contrary, Imhotep would require a whole new specification of the user 
profile, with the corresponding consequences.

Table~\ref{tbl:adaptui_vs_imhotep_scenario2_user} and 
Table~\ref{tbl:adaptui_vs_imhotep_scenario2_context_device} show the main 
difference of these two solutions in the modelling of the user, the context and
the device. Through them we can see how both solutions model the user 
disabilities, but only AdaptUI is aware of the activity of the user and the 
restrictions it implies.


\begin{table}
 \caption{AdaptUI and Imhotep comparison of the final reached adaptation for the
 Scenario 2 regarding the user.}
 \label{tbl:adaptui_vs_imhotep_scenario2_user}
 \footnotesize
 \centering
\begin{tabular}{l l l l l }
  \hline 
  \textbf{Solution} & \multicolumn{4}{c}{\textbf{User capabilities}}\\
      & \textbf{Sight} & \textbf{Hearing} & \textbf{Other} & \textbf{Activity}\\
  \hline
  Imhotep & \checkmark & \checkmark 	  & \checkmark 		   & ~ 	\\
  AdaptUI & \checkmark & \checkmark       & colour blind & ``driving''\\
  \hline
\end{tabular}
\end{table}

\begin{table}
 \caption{AdaptUI and Imhotep comparison of the final reached adaptation for the
 Scenario 2 regarding the context and the device.}
 \label{tbl:adaptui_vs_imhotep_scenario2_context_device}
 \footnotesize
 \centering
\begin{tabular}{l l l l l l l l}
\hline 
\textbf{Solution} & \multicolumn{3}{c}{\textbf{Context characteristics}} 
& \multicolumn{2}{c}{\textbf{Device characteristics}}\\
& \textbf{Light} & \textbf{Noise} & \textbf{Temperature} & \textbf{Resolution} 
& 
\textbf{Battery} \\
  \hline
	Imhotep		& ~ & ~ & ~ & \checkmark & ~\\
	AdaptUI		& ~ & ~ & \checkmark & \checkmark & \checkmark \\
  \hline
\end{tabular}
\end{table}


\subsubsection{Scenario 3: Limitations Caused by Disabilities}
\label{sec:scenario3}

The last scenario introduces a situation in which the user suffers from a 
disability that impedes a proper interaction with the device. The scenario and 
its main characteristics are described in the following lines.

Patrick is a 25 years old man who lives in London. He lost his eyesight when he
was 10. He uses accessibility tools when he interacts with computers and home
devices. His PC is equipped with an utility which reads from the screen so he
can navigate and use it. The problem is that a similar feature in his mobile
device is often not enough depending on the context situation. Mostly traffic
and street noise usually make difficult for Patrick to hear the messages read by
the device.

Table~\ref{tbl:scenario3} shows the configuration of the first model for this
scenario. As can be seen in this table, Patrick needs from some adaptation to
interact with his mobile device. Common accessibility tools cover this issue by
reading the text in the screen. However, they do not consider context. This
entails several obstacles during the interaction, for example when there are
problems to hear the text read by the mobile device.

\begin{table}
 \caption{Scenario 3 situation summary.}
 \label{tbl:scenario3}
 \footnotesize
 \centering
\begin{tabular}{l l}
  \hline 
				& \textbf{Scenario 3}		\\
  \hline
  User \\
  \qquad - Personal data 	& Patrick, 25 years old, British\\
  \qquad - Activity	 	& 				\\
  \qquad - Known disabilities 	& Sight disability		\\
%   \hline
  Context \\
  \qquad - Location 		& Relative: London, United Kingdom\\
  \qquad - Time			& 12:00			\\
  \qquad - Brightness		& 20,000 \ac{lx}		\\
  \qquad - Temperature		& 14 ºC			\\
%   \hline
  Device 			& Samsung Galaxy Ace		\\
				&  Battery: 55\%		\\	
  \hline
\end{tabular}
\end{table}

To reach the final adaptation the following steps are performed by the AdaptUI
platform. As Patrick configures his profile indicating that the preferred input
interaction should be \textit{``voice\_control''} (as can be seen in 
Table~\ref{tbl:userAux_scenario3}, the system becomes aware of the fact that 
Patrick suffers from a sight disability). Besides, the profile is configured 
indicating that the Display is not applicable, and that its configuration 
should not be adapted (because in this case it is not necessary, as Patrick is 
blind). Therefore, the adaptation should result into a promotion of an 
alternative channel, so the user would be able to interact alternatively and 
perform the same tasks. The volume value configured by Patrick is defined as 4. 
This is because the user considers that this value is enough to clearly hear 
the device's voice. Nevertheless, Patrick knows that this should not be an 
static value, as when he is in the street sometimes that value does not allow 
him to hear properly. As a consequence, the \textit{userAudioApplicableIsStatic}
property is modelled as false.

\begin{table}
 \caption{User profile for Scenario 3.}
 \label{tbl:user_profile_scenario3}
 \footnotesize
 \centering
\begin{tabular}{l l l}
  \hline 
  \textbf{Class}		& \multicolumn{2}{c}{\textbf{Scenario 3}}\\
		& \textbf{Data property} & \textbf{value}\\
  \hline
  Display 	& \textit{userDisplayBrightnessIsStatic}& true	\\
		& \textit{userDisplayIsApplicable} 	& false	\\
  Audio 	& \textit{userDisplayApplicableIsStatic}& true	\\
		& \textit{userAudioHasApplicable} 	& true	\\
		& \textit{userAudioApplicableIsStatic} 	& false	\\
		& \textit{userAudioHasVolume}  		& 4 	\\
  Interface 	& \textit{userInterfaceInput}		& voice\_control\\
		& \textit{userInterfaceOutput} 		& audio \\
  Experience	& \textit{userHasExperience} 		& default\\
  View		& \textit{userViewIsStatic}		& true	\\
  Other 	& \textit{userHasLanguage}		& English\\
		& \textit{vibration} 			& true 	\\
  \hline
\end{tabular}
\end{table}

Taking into account the input shown in Table~\ref{tbl:scenario3} and
Table~\ref{tbl:user_profile_scenario3}, AdaptUI begins the process filling
the auxiliary classes and properties as is shown in
Table~\ref{tbl:userAux_scenario3}


\begin{table}
 \caption{UserAux class generated by the pre-adaptation rules set and resulting
 \ac{ui} first adaptation for both scenarios.}
 \label{tbl:userAux_scenario3}
 \footnotesize
 \centering
\begin{tabular}{l l}
  \hline 
	\multicolumn{2}{c}{\textbf{Scenario 3}}	\\
	\textbf{UserAux} 	& \textbf{Value}\\
  \hline
  \textit{hasDisplayApplicable} & false		\\
  \textit{hasDisplayBrightness}	& false		\\
  \textit{hasAudioApplicable}	& true		\\
%   \textit{hasRestriction}	& true???	\\
  \hline
\end{tabular}
\end{table}

In the final adaptation is shown how due to Patrick's sight disability the 
AdaptUI platform adapts the interface channel with the user. Thus, the voice
becomes the best choice for interacting with the device.

\begin{table}
 \caption{Final adaptation for the scenarios 3.}
 \label{tbl:final_adaptation_scenario3}
 \footnotesize
 \centering
\begin{tabular}{l l}
  \hline 
	\multicolumn{2}{c}{\textbf{Scenario 3}}	\\
	\textbf{Adaptation} & \textbf{Value}\\
  \hline
  \textit{hasBrightness}& false		\\
  \textit{hasInput}	& voice 	\\
  \textit{hasResponse}	& vibration	\\
  \textit{hasVolume}	& 7 (max) 	\\
  \hline
\end{tabular}
\end{table}

Imhotep models user visual disabilities in its profile under the
\textit{problems.sight} variable. The problem comes when the user has to fill 
the profile. In AdaptUI the Capabilities Collector allows blind users to 
configure the interaction channels thanks to voice control. In any case, 
considering that the user profile is configured, it is shown in 
Listing~\ref{lst:scenario3_imhotep}.

% As said before, Imhotep does not consider activities in its user and device 
% profile. However, we can assume a possible user configuration profile under 
% the 
% presented circumstances in Table~\ref{tbl:scenario2}. Thus, a possible profile
% is illustrated through Listing~\ref{lst:scenario2_imhotep}.

\inputminted[linenos=true, fontsize=\footnotesize, frame=lines]{json}{5_experiments_and_results/scenario3_imhotep.json}
\captionof{listing}{The variables file, in which device characteristics and user 
capabilities are described for the scenario 3.\label{lst:scenario3_imhotep}}

Regarding the profile it is very difficult for Imhotep to make the 
proper decisions. The input file lacks of practical information about the user
disability. In fact, the Imhotep's adapted user interface is the same as in the
scenario 2.


\myparagraph{Discussion}
\label{sec:scenario3_discussion}

In this scenario we present a situation in which the adaptation should be lead
by a concrete user disability. Here, context or activities are not as
important as in the other scenarios. Thus, the efforts of the AdaptUI platform
are focused on avoiding any visual interface. 

In the case of AdaptUI this is carried out based on the user model filled 
during the interaction with the Capabilities Collector. This module allows 
blind users to interact with it using their voice. On the contrary, in the case 
of Imhotep this is not possible. Filling the user profile (represented by the 
variables file) is impossible for a blind user.

Nevertheless, considering that the user is able to fill it, Imhotep cannot 
react to the specific disability of the user. It is true that it will present 
an alternative user interface, with audio instructions and a different colour 
configuration set, but it will not take the specific problem of the user into 
account. On the contrary, the AdaptUI platform does not consider any specific
disability. As it is centred in the preferences of the user, the resulting 
adapted user interface will be more practical.

\myparagraph{Conclusions}
\label{sec:scenario3_conclusions}

As is shown in Table~\ref{tbl:user_profile_scenario3} Patrick has a sight
disability. Imhotep takes sight and hearing disabilities into account for the
adaptation process. Therefore, depending on the network range and on the server
side programmed adaptations different \ac{ui} configuration will be available. 
The main setback is that Patrick has to indicate the concrete physiological 
problem. For example, he would have to specify a figure detailing the dioptres 
he has. Then, the Imhotep server would act consequently. According to this, if 
the concrete figure has not been considered in the adaptation process, the 
resulting \ac{ui} might not be adequate. Another issue is that, the network 
dependency would make the adaptation process too long. AdaptUI covers this problem 
as it runs fully in the mobile device.

The main differences when dealing with the user, context and device models in
this scenario are shown in Table~\ref{tbl:adaptui_vs_imhotep_scenario3_user} and 
Table~\ref{tbl:adaptui_vs_imhotep_scenario3_context_device}. 


\begin{table}
 \caption{AdaptUI and Imhotep comparison of the final reached adaptation for the
 Scenario 3 regarding the user.}
 \label{tbl:adaptui_vs_imhotep_scenario3_user}
 \footnotesize
 \centering
\begin{tabular}{l l l l l}
  \hline 
  \textbf{Solution}& \multicolumn{4}{c}{\textbf{User capabilities}} \\
  & \textbf{Sight} & \textbf{Hearing} & \textbf{Other} & \textbf{Activity}\\
  \hline
  Imhotep	& \checkmark & ~ & ~ & ~ \\
  AdaptUI	& \checkmark & ~ & \checkmark & \checkmark\\
  \hline
\end{tabular}
\end{table}

\begin{table}
 \caption{AdaptUI and Imhotep comparison of the final reached adaptation for the
 Scenario 3 regarding the context and the device.}
 \label{tbl:adaptui_vs_imhotep_scenario3_context_device}
 \footnotesize
 \centering
\begin{tabular}{l l l l l l l l}
  \hline 
  \textbf{Solution} &  \multicolumn{3}{c}{\textbf{Context characteristics}} & 
  \multicolumn{2}{c}{\textbf{Device characteristics}}\\
  & \textbf{Light} & \textbf{Noise} & \textbf{Temperature} & 
  \textbf{Resolution} & \textbf{Battery} \\
  \hline
  Imhotep	& ~ & ~ & ~ & \checkmark & ~\\
  AdaptUI	& ~ & \checkmark & ~ & \checkmark & \checkmark \\
  \hline
\end{tabular}
\end{table}