\section{Discussion}
\label{sec:evaluation_discussion}

Along this chapter several experiments have been carried out to test the AdaptUI
platform. First, we have compared the possible consequences of porting semantics
and reasoning to a mobile Android based platform. Through the experiments shown 
in Section~\ref{sec:technical_evaluation} it has been demonstrated how managing 
semantics in mobile phones is possible and affordable. The only condition we 
find that should be considered when dealing with mobile reasoning is that the 
ABox and \ac{swrl} axioms set should be limited or small enough to avoid the 
overloading of the whole system. Table~\ref{tbl:eval_default_ont},
Table~\ref{tbl:eval_abox} and Table~\ref{tbl:eval_swrl} show these differences 
remarking performance of reasoning for the corresponding execution platform.

After this first part of the experiments, a comparison between other 
alternatives has been presented. As it is explained in 
Section~\ref{sec:imhotep_vs_adaptui}, Imhotep is a framework which aims to ease 
the development of adaptive user interfaces. The main problems or limitations 
of the Imhotep framework are: 

\begin{itemize}
  \item It cannot be deployed in a mobile phone. The design of the platform
  requires the existence of an external server which will manage the adaptation.
  
  \item To collect the user profile and capabilities a client application in
  the user's mobile phone is needed. This application sends user and device
  profile to the adaptation server.
  
  \item Besides, the knowledge about the user capabilities is physiological,
  which we have defended along this dissertation to be non practical and 
difficult
  to manage.
  
  \item It does not consider the current context, limiting the adaptations
  to the user's and devices capabilities.
  
  \item The adaptations are static. Once the server compiles the corresponding
  sources and the user interfaces are adapted, if the user is not comfortable
  with them the whole process has to be repeated.
  
  \item Finally, there is the problem of the network dependency, which 
  approximately adds 14 seconds to the whole process for compiling the 
  corresponding sources. Although Imhotep's performance when using the cache is 
  good enough (see Figure~\ref{fig:imhotep_comparison}), the response decreases 
  significantly  when this cache is not present.
\end{itemize}


AdaptUI faces these problems inherited from Imhotep and solve them as follows:

\begin{itemize}
  \item Regarding the problem of using an external server, the AdaptUI platform
  runs 100\% in the mobile device. It is true that nowadays these devices
  capabilities are far from the ones present when Imhotep was conceived. This
  allowed us to design a platform compatible with Android and network or 
  external services independent. Through the ontology and several optimized 
  adaptation engines the adaptation is carried out without needing external 
  processing assistance.
  
  \item Although AdaptUI also requires a software module to collect user
  capabilities, the perspective differs. In AdaptUI, the Capabilities Collector
  module aims to collect user capabilities regarding their preferences, not
  the physiological capabilities themselves. Besides, this module is fully
  integrated in the AdaptUI platform.
  
  \item As said before, AdaptUI does not consider physiological capabilities of
  the user. Instead of that, AdaptUI looks for several preferences of the user,
  modelled through the AdaptUIOnt ontology. Thus, adaptations and corrections
  on the corresponding adaptations are easier and practical.
  
  \item Regarding the context issue, AdaptUI considers the environment as one 
  of the three main entities in the user interfaces adaptation domain. Besides,
  the disabilities that users might suffer are understood as a set of conditions
  that somehow limit the user. These conditions do not have to be specifically
  physiological. To us, context might also generate several situations in which
  the user might sense temporary disabled.
  
  \item The adaptations are dynamic, not just considering that they are 
  modifiable in running time. They also change through the experience monitored 
  by the Adaptation Polisher. This module aims to always monitor the user 
  interaction with the adapted interfaces to provide alternatives if the 
  usability decreases.
  
  \item AdaptUI do not depend on any network connection. It runs 100\% in the
  mobile device, which allows a fully independence from external connectivity
  issues.
\end{itemize}

Regarding the performance of both platforms, Table~\ref{tbl:imhotep_vs_adaptui}
shows how after loading the ontology the adaptation process response in AdaptUI
is acceptable, not exceeding the 2 seconds. On the contrary, if the cache is not
used, Imhotep has to deal with a performance over 14 seconds due to the network
dependency.


After these comparisons, several qualitative experiments are presented through
different scenarios (see Section~\ref{sec:scenarios}). These experiments are
based on several complex situations that put AdaptUI under the microscope.
The presented scenarios have been categorized into 3 different groups, each 
group representing a concrete situation in which an adaptation of the user 
interface would be needed. In this case, these categories are:

\begin{enumerate}[label=\alph{*}]
  \item Limitations caused by the context current conditions.
  \item Limitations caused by the set of activities the user might be doing.
  \item Limitations caused by the disabilities the user might suffer from.
\end{enumerate}

These scenarios present a series of characteristics that distinguish them. After
analysing each specific situation, the adaptation process is detailed. Besides,
before presenting several conclusions, this adaptation process performed by 
AdaptUI is compared with Imhotep. Thus, an overview of the behaviour and 
performance of both systems is depicted.

Finally, the users' feedback is collected. To do so, an experiment with AdaptUI 
has been developed. Users have to use an application which translates the 
interaction to the AdaptUIOnt ontology. Then, several context changes are 
triggered so the users witness how the user interface adapts dynamically. After 
this experiment, a questionnaire is presented, so their experiences with the 
AdaptUI platform can be collected. Along with several extra questions to ease 
the analysis of the results, the \ac{sus} questionnaire is used. The main idea 
extracted from these results is that users find the platform useful. Although 
there are several differences considering one group of users or another (e.g., 
based on their age, technical experience or disability), the obtained results 
are promising. Besides, the requested developers also consider that the 
framework would benefit the design of adaptive user interfaces, integrating them 
with their personal applications. Hence, we conclude that, although much work 
can be performed, the results of the evaluation performed and described in this 
chapter are promising.