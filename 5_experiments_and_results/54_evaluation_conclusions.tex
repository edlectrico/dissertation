\section{Conclusions}
\label{sec:evaluation_conclusions}

During this dissertation we have introduced AdaptUI as a platform for dynamic
user interface adaptation. As new technical contributions have been developed 
and detailed (e.g., a mobile reasoning engine), several experiments regarding 
this area have been presented. Comparing the presented system with others gives
us the perspective to criticise AdaptUI and extract several conclusions. In this
section these conclusions are depicted.

On the one hand, the performed technical experiments revealed that mobile 
reasoning engines are useful, practical and affordable regarding performance. It
is true that, in this case the \textit{Pellet4Android} suffers depending on the 
hardware of the device. Besides, the more axioms we add to the corresponding 
ontology (specially to the \ac{swrl} axioms set), the bigger the delay in the 
performance of the system added. However, we believe that the AdaptUIOnt ontology 
is light enough to be easily managed by any Android based device. Furthermore, 
the smartphones market is increasing the possibilities of these devices by 
improving their hardware capabilities. This means that in the near future 
complex processing will be totally affordable by these devices. 

The comparison with Imhotep exposes several benefits of AdaptUI. The first one, 
is the lack of dependency on external services or servers. This fact does not 
only have an affect on the performance, it avoids possible network failure 
problems. Another fundamental aspect that should be emphasized due to this 
comparison is the way users are modelled. While in Imhotep specific user 
disabilities are needed, AdaptUI just requires a simple interaction with the 
Capabilities Collector module. The interaction of the user is translated to the 
AdaptUIOnt ontology, and the rest of the adaptation process is delegated to the 
AdaptUI platform. Besides, AdaptUI relies on the Adaptation Polisher to perform
slight modifications on the fly of the adapted user interface, in case the 
interaction with the user is not satisfactory. Regarding the time performances, 
although using cached applications in Imhotep, showed a faster response, the 
truth is that new and real scenarios revealed a much bigger accuracy in AdaptUI. 
This has been shown through the scenarios presented in Section~\ref{sec:scenarios}.

On the other hand 30 users have tested the platform and have been asked about 
its usability. Users seem to react satisfactorily to AdaptUI. The obtained 
results through the questionnaire show how depending on the users ageing and 
technical knowledge they feel more or less comfortable with it. These results
also expose how most developers would use it as a tool for their applications. 
On the contrary, users who declare to suffer sometimes from context disabilities 
are willing to use it integrated with their Android device.

% 
% As we aimed to evaluate not only the technical specifications of AdaptUI wee 
% studied how we could include the user feedback in the whole evaluation equation.
% In order to do this, a demonstrator has been developed under a controlled 
% environment to collect this feedback through a usability questionnaire.
% 
% After analysing the results obtained during the experiments presented in this 
% chapter, the following conclusions are noted:
% 
