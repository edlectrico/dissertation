\subsection{The knowledge \ac{api}}
\label{sec:knowledge_api}

Ontologies are formal specification of concepts that represent knowledge within
a specific domain. This knowledge is provided through a vocabulary which describes
the types and relationships of the concepts represented in that specific domain.

The knowledge conceptualization is mainly described through classes, attributes,
relations, individuals and axioms: 

\begin{itemize}
  \item Classes represent concepts within the domain. In other words, classes
  describe the type or kind of the members of the class.
  
  \item Each class can have a set of properties or characteristics which describe
  it. These properties are represented through attributes. These attributes
  are also called datatype properties.
  
  \item Relations detail the relationships that the concepts of the ontology
  might share. They are referred as object properties.
  \item Instances are the representation of the concepts of the
  ontology.
  
  \item Finally, axioms, including rules, are assertions that together comprise
  the overall theory that the ontology describes in the current domain.
\end{itemize}

These concepts together build ontologies to represent the knowledge of a domain.
In AdaptUI,the knowledge of a domain is not considered static. Besides, the 
solutions provided in the literature usually do not apply well when changing 
the domain. Therefore, AdaptUI aims to ease the adaptation of the domain 
knowledge through the customization of these concepts. Through a set of methods 
within the AdaptUI \ac{api}, developers are allowed to insert, edit and delete 
classes, attributes, instances and rules. Table~\ref{tbl:api_knowledge} shows 
the available methods for developers to modify the knowledge contained in the 
AdaptUIOnt ontology.

AdaptUI is designed to support several user capabilities, context status and
device characteristics. Nevertheless, regarding the rules set it is impossible
to assume every possible situation and react accordingly. Thus, a set of methods
to create, edit and delete \ac{swrl} rules has been provided. This allows 
developers to adapt the AdaptUI platform to new and unexpected situations in the 
domain where the platform might not behave properly.

\begin{center}
\footnotesize
\begin{longtable}{l l}
  \caption{Knowledge related AdaptUI \ac{api} methods.}\\
  \label{tbl:api_knowledge} \\
  \hline 
  \textbf{Method}		& \textbf{Description}\\
  \hline
  insertClass(namespace, className)	& Inserts a new class.		\\
  removeClass(namespace, className)	& Erases an existing class, its	\\
					& individuals, attributes and	\\
					& relationships with other 	\\
					& classes.			\\
  editClass(namespace, newClassName)	& Changes the name of the class.\\
					& Internally it deletes the 	\\
					& class passed as parameter 	\\
					& and creates a new one.	\\
  \hline 
  insertObjectProperty	(namespace, objectPropName,& Inserts a new object\\
  Map$<$namespace, classname$>$)	& property connecting two classes.\\
  removeObjectProperty(namespace, objectPropName)& Erases an existing datatype\\
 					& property.			\\
  editObjectProperty(namespace, objectPropName)& Changes the name of the\\
					& datatype property. Internally \\
					& it deletes the property 	\\
					& passed as parameter and 	\\
					& creates a new one.		\\
  \hline 
  addDatatypeProperty(namespace, dataTypeName)	& Inserts a new datatype\\
					& property assigning a value to \\
 					& a class.			\\
					& If it does not exist, it is	\\
					& created.			\\
  removeDatatypeProperty(namespace, dataTypeName)& Erases an existing datatype\\
 					& property.			\\
  editDatatypeProperty(namespace, dataTypeName)	& Changes the name of the \\
					& datatype property. Internally \\
					& it deletes the property passed\\
					& as parameter and creates 	\\
 					& a new one.			\\
  \hline 
  insertIndividual(name, namespace, className)& Inserts a new individual of a \\
					& concrete class.		\\
  removeIndividual(name)	 	& Removes an individual.	\\

  \hline 
  insertRule(name, antecedent, consequent)& Inserts a new rule.		\\
  removeRule(name)	 		& Removes a rule if the rule 	\\
					& has a name associated to it.	\\
  \hline
\end{longtable}
\end{center}