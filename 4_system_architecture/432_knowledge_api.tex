\subsection{The knowledge \ac{api}}
\label{sec:knowledge_api}

Ontologies are formal specification of concepts that represent knowledge within
a specific domain. This knowledge is provided through a vocabulary which describes
the types and relationships of the concepts represented in that specific domain.

The knowledge conceptualization is mainly described through classes, attributes,
relations, individuals and axioms: 

\begin{itemize}
  \item Classes represent concepts within the domain. In other words, classes
  describe the type or kind of the members of the class.
  
  \item Each class can have a set of properties or characteristics which describe
  it. These properties are represented through attributes. These attributes
  are also called datatype properties.
  
  \item Relations detail the relationships that the concepts of the ontology
  might share. They are referred as object properties.
  \item Instances are the representation of the concepts of the
  ontology.
  
  \item Finally, axioms, including rules, are assertions that together comprise
  the overall theory that the ontology describes in the current domain.
\end{itemize}

These concepts together build ontologies to represent the knowledge of a domain.
In AdaptUI,the knowledge of a domain is not considered static. Besides, the 
solutions provided in the literature usually do not apply well when changing 
the domain. Therefore, AdaptUI aims to ease the adaptation of the domain 
knowledge through the customization of these concepts. Through a set of methods 
within the AdaptUI \ac{api}, developers are allowed to insert, edit and delete 
classes, attributes, instances and rules. Table~\ref{tbl:api_knowledge} shows 
the available methods for developers to modify the knowledge contained in the 
AdaptUIOnt ontology.

AdaptUI is designed to support several user capabilities, context status and
device characteristics. Nevertheless, regarding the rules set it is impossible
to assume every possible situation and react accordingly. Thus, a set of methods
to create, edit and delete \ac{swrl} rules has been provided. This allows 
developers to adapt the AdaptUI platform to new and unexpected situations in the 
domain where the platform might not behave properly.

\begin{center}
\footnotesize

\begin{longtable}{l l l l}
  \caption{Knowledge related AdaptUI \ac{api} methods}\\
  \label{tbl:api_knowledge} \\
% \footnotesize
% \centering
%  \begin{tabular}{l l l l}
  \hline 
  \textbf{\ac{api} group}& \textbf{Method}& \textbf{Description} 	& \textbf{Input}\\
  \hline
  Classes	& insertClass		& Inserts a new class.		& The namespace and the \\
		& 			& 				& class name.		\\
		& removeClass		& Erases an existing class, its	& The namespace and the \\
		& 			& individuals, attributes and	& class name.		\\
		& 			& relationships with other 	& 			\\
		& 			& classes.			& 			\\
		& editClass		& Changes the name of the class.& The namespace and the \\
		&			& Internally it deletes the 	& new class name.	\\
		&			& class passed as parameter 	&			\\
		&			& and creates a new one.	&			\\
\hline 
  Object 	& insertObjectProperty	& Inserts a new object property	& The namespace and the	\\
  properties	&			& connecting two classes.	&  object property name.\\
		&			&				& It also needs the 	\\
		& 			& 				& namespaces and names of\\
		&			&				& the classes that will be\\
		& 			& 				& connected. The two set\\ 
		&			&				& of classes are passed as\\
		& 			& 				& a Map$<$String namespace,\\
		& 			& 				& String classname$>$.	\\
		& removeObjectProperty	& Erases an existing datatype	& The namespace and the \\
		& 			& property.			& object property name.	\\
		& editObjectProperty	& Changes the name of the 	& The namespace and the \\
		&			& datatype property. Internally & new class name.	\\
		&			& it deletes the property 	& 			\\
		&			& passed as parameter and 	&			\\
		& 			& creates a new one.		& 			\\
  \hline 
  Datatype 	& addDatatypeProperty	& Inserts a new datatype 	& The namespace and the \\
  properties	&			& property assigning a value to & datatype property name.\\
		& 			& a class.			& 			\\
		&			& If it does not exist, it is	& It also needs the namespace\\
		&			& created.			& and name of the class \\
		& 			& 				& that will have this 	\\
		& 			& 				& property and the value\\
		& 			& 				& type.			\\
		& removeDatatypeProperty& Erases an existing datatype	& The namespace and the \\
		& 			& property.			& datatype property name.\\
		& editDatatypeProperty	& Changes the name of the 	& The namespace and the \\
		&			& datatype property. Internally & new class name.	\\
		&			& it deletes the property passed& 			\\
		&			& as parameter and creates 	&			\\
		& 			& a new one.			& 			\\
  \hline 
  Individuals 	& insertIndividual	& Inserts a new individual of a & The name of the 	\\
		&			& concrete class.		& individual and the 	\\
		& 			& 				& namespace and name 	\\
		& 			& 				& of the class.		\\
		& removeIndividual 	& Removes an individual.	& The name of the 	\\
		&			&				& individual to be 	\\
		& 			& 				& deleted.		\\
  \hline 
  Rules 	& insertRule		& Inserts a new rule.		& The name of the 	\\
		&			& 				& rule, the antecedent 	\\
		& 			& 				& and the consequent.	\\
		& removeRule 		& Removes a rule if the rule 	& The name of the rule 	\\
		&			& has a name associated to it.	& to be deleted.	\\
  \hline
% \end{tabular}
\end{longtable}

\end{center}