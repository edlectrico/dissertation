\subsection{The adaptation \ac{api}}
\label{sec:adaptation_api}

The adaptation \ac{api} aims to provide developers a set of methods to include
user interface adaptation in their applications. Table~\ref{tbl:api_adaptation} 
details the most useful methods available in AdaptUI.

\begin{center}
\footnotesize
\begin{longtable}{l l}
  \caption{Adaptation related AdaptUI \ac{api} methods.}\\
  \label{tbl:api_adaptation} \\
  \hline 
  \textbf{Method}				& \textbf{Description}					\\
  \hline
  adaptLoadedViews()				& It returns a Map object which keys are the property	\\
						& names, and its values the stored values in the ontology.\\
						& It requires the use of the constructor with parameters.\\
  adaptViewBackgroundColor(namespace, className)& It adapts the view's background colour with the 	\\
						& corresponding value.					\\
  adaptViewTextColor(namespace, className)	& It adapts the view's text colour with the corresponding\\
						& value.						\\
  adaptViewTextSize(namespace, className)	& It adapts the view's text size with the corresponding	\\
						& value.						\\
  adaptBrightness(namespace, className)		& It adapts the display's brightness with the corresponding\\
						& value.						\\
  adaptVolume(namespace, className)		& It adapts the volume with the corresponding value.	\\
  \hline
\end{longtable}
\end{center}

Listing~\ref{lst:api_adaptation} shows an example of an Android activity 
developed using AdaptUI. 

\inputminted[linenos=true, fontsize=\footnotesize, frame=lines]{java}{4_system_architecture/api_adaptation.java}
\captionof{listing}{The AbstractActivity class ontology related methods.\label{lst:api_adaptation}}