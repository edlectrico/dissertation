\subsection{The Adaptation \ac{api}}
\label{sec:adaptation_api}

The adaptation \ac{api} aims to provide developers a set of methods to include
user interface adaptation in their applications. As the adaptation is lead by
Android views, the goal of the adaptation \ac{api} is to be easy enough for
any Android developer to use it. Thus, with a simple view initialization
it would be possible to change the view's aspect by calling several simple 
AdaptUI methods. Table~\ref{tbl:api_adaptation} details the most useful methods 
available in the AdaptUI's adaptation \ac{api}.

\begin{center}
\footnotesize
\begin{longtable}{l l}
  \caption{Adaptation related AdaptUI \ac{api} methods.}\\
  \label{tbl:api_adaptation} \\
  \hline 
  \textbf{Method}				& \textbf{Description}			\\
  \hline
  adaptLoadedViews()				& It returns a Map object which keys are\\
						& the property names, and its values the\\
						& stored values in the ontology.	\\
						& It requires the use of the constructor\\
						& with parameters.			\\
  adaptViewBackgroundColor(namespace, className)& It adapts the view's background colour\\
						& with the corresponding value.		\\
  adaptViewBackgroundColor(fullClassName)	& 					\\
  adaptViewTextColor(namespace, className)	& It adapts the view's text colour with \\
						& the corresponding value.		\\
  adaptViewTextColor(fullClassName)		&					\\
  adaptViewTextSize(namespace, className)	& It adapts the view's text size with 	\\
						& the corresponding value.		\\
  adaptViewTextsize(fullClassName)		&					\\
  adaptBrightness(namespace, className)		& It adapts the display's brightness 	\\
						& with the corresponding value.		\\
  adaptBrightness(fullClassName)		& 					\\
  adaptVolume(namespace, className)		& It adapts the volume with the 	\\
  adaptVolume(fullClassName)			& corresponding value.			\\
  \hline
\end{longtable}
\end{center}

Listing~\ref{lst:api_adaptation} shows an example of an Android activity using 
AdaptUI. This example presents a simple button initialized with the values 
stored in the AdaptUIOnt ontology.

\inputminted[linenos=true, fontsize=\footnotesize, frame=lines]{java}{4_system_architecture/api_adaptation.java}
\captionof{listing}{The AbstractActivity class ontology related methods.\label{lst:api_adaptation}}