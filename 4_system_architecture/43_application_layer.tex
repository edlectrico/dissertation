\section{The Application Layer}
\label{sec:application_layer}

The last layer of the AdaptUI architecture is the Application Layer. This layer
aims to ease the use of the AdaptUI platform to developers. To do this, several
tools have been provided. These tools have been designed not only to integrate
AdaptUI within developers' applications (to adapt their user interfaces) but also
to leave in their hands the decision of how to manage the knowledge of the AdaptUI
platform. Thus, through the AdaptUI provided \ac{api}, developers are allowed to:

\begin{enumerate}[label=\alph*)]
 \item Initialize a Pellet reasoner and connect the application to it, loading
 the AdaptUIOnt ontology and its rules.
 
 \item Launch queries about the knowledge stored in the ontology in order to
 adapt the corresponding user interface elements.
 
 \item Generate, change and adapt the knowledge contained in AdaptUI. Classes,
 properties and relationships described in the AdaptUIOnt ontology are fully
 customizable by developers to cover the domain of their adaptation problem.
 
 \item Customize the provided set of rules. The AdaptUI \ac{api} provides a set of
 methods to edit different rules.
\end{enumerate}

The AdaptUI's \ac{api} can be internally divided into two separated \acp{api}. 
The first one, aiming to solve developer's adaptation issues. 
Listing~\ref{lst:api_adaptation} shows an example of an Android activity 
developed using AdaptUI. The second one, with the purpose of adapting the 
knowledge of the domain (see an example in Listing~\ref{lst:api_knowledge}).

\inputminted[linenos=true, fontsize=\footnotesize, frame=lines]{java}{4_system_architecture/api_adaptation.java}
\captionof{listing}{The AbstractActivity class ontology related methods.\label{lst:api_adaptation}}

\inputminted[linenos=true, fontsize=\footnotesize, frame=lines]{java}{4_system_architecture/api_knowledge.java}
\captionof{listing}{Modifying the ontology knowledge with the AdaptUI's \ac{api}.\label{lst:api_knowledge}}

Table~\ref{tbl:api} contains the most significant methods from AdaptUI, regarding
the loading of the ontology and initialization.

\begin{center}
\footnotesize
\begin{longtable}{l l}
  \caption{AdaptUI \ac{api} methods.}\\
  \label{tbl:api} \\
  \hline 
  \textbf{Method}				& \textbf{Description}\\
  \hline
  AdaptUI()					& Constructor with no parameters			\\
  AdaptUI(namespace, views)			& This method requires the main ontology namespace	\\
						& and a Collection with the views to adapt.		\\
  loadOntologyFromFile(path, fileLocation)	& It loads an ontology file from a specific Android	\\
						& path.							\\
  loadOntologyFromUri(uri)			& Loads the ontology from a specified \ac{uri}.		\\
  getExternalDirectory()			& Returns the root storage folder in the current	\\
						& Android device.					\\
  mapOntology(namespace, fileName)		& It loads the stored ontology to avoid Internet connection\\
						& when managing the namespace.				\\
  getOntologyManager()				& It returns the \textit{OntologyManager} instance, which\\
						& gives access to further operations (i.e., 		\\
						& \textit{getIndividualOfClass}.			\\
  \hline
\end{longtable}
\end{center}

\subsection{The adaptation \ac{api}}
\label{sec:adaptation_api}

The adaptation \ac{api} aims to provide developers a set of methods to include
user interface adaptation in their applications. Table~\ref{tbl:api_adaptation} 
details the most useful methods available in AdaptUI.

\begin{center}
\footnotesize
\begin{longtable}{l l}
  \caption{Adaptation related AdaptUI \ac{api} methods.}\\
  \label{tbl:api_adaptation} \\
  \hline 
  \textbf{Method}				& \textbf{Description}					\\
  \hline
  adaptLoadedViews()				& It returns a Map object which keys are the property	\\
						& names, and its values the stored values in the ontology.\\
						& It requires the use of the constructor with parameters.\\
  adaptViewBackgroundColor(namespace, className)& It adapts the view's background colour with the 	\\
						& corresponding value.					\\
  adaptViewTextColor(namespace, className)	& It adapts the view's text colour with the corresponding\\
						& value.						\\
  adaptViewTextSize(namespace, className)	& It adapts the view's text size with the corresponding	\\
						& value.						\\
  adaptBrightness(namespace, className)		& It adapts the display's brightness with the corresponding\\
						& value.						\\
  adaptVolume(namespace, className)		& It adapts the volume with the corresponding value.	\\
  \hline
\end{longtable}
\end{center}

Listing~\ref{lst:api_adaptation} shows an example of an Android activity 
developed using AdaptUI. 

\inputminted[linenos=true, fontsize=\footnotesize, frame=lines]{java}{4_system_architecture/api_adaptation.java}
\captionof{listing}{The AbstractActivity class ontology related methods.\label{lst:api_adaptation}}
\subsection{The knowledge \ac{api}}
\label{sec:knowledge_api}

Ontologies are formal specification of concepts that represent knowledge within
a specific domain. This knowledge is provided through a vocabulary which describes
the types and relationships of the concepts represented in that specific domain.

The knowledge conceptualization is mainly described through classes, attributes,
relations, individuals and axioms: 

\begin{itemize}
  \item Classes represent concepts within the domain. In other words, classes
  describe the type or kind of the members of the class.
  
  \item Each class can have a set of properties or characteristics which describe
  it. These properties are represented through attributes. These attributes
  are also called datatype properties.
  
  \item Relations detail the relationships that the concepts of the ontology
  might share. They are referred as object properties.
  \item Instances are the representation of the concepts of the
  ontology.
  
  \item Finally, axioms, including rules, are assertions that together comprise
  the overall theory that the ontology describes in the current domain.
\end{itemize}

These concepts together build ontologies to represent the knowledge of a domain.
In AdaptUI,the knowledge of a domain is not considered static. Besides, the 
solutions provided in the literature usually do not apply well when changing 
the domain. Therefore, AdaptUI aims to ease the adaptation of the domain 
knowledge through the customization of these concepts. Through a set of methods 
within the AdaptUI \ac{api}, developers are allowed to insert, edit and delete 
classes, attributes, instances and rules. Table~\ref{tbl:api_knowledge} shows 
the available methods for developers to modify the knowledge contained in the 
AdaptUIOnt ontology.

AdaptUI is designed to support several user capabilities, context status and
device characteristics. Nevertheless, regarding the rules set it is impossible
to assume every possible situation and react accordingly. Thus, a set of methods
to create, edit and delete \ac{swrl} rules has been provided. This allows 
developers to adapt the AdaptUI platform to new and unexpected situations in the 
domain where the platform might not behave properly.

\begin{center}
\footnotesize
\begin{longtable}{l l}
  \caption{Knowledge related AdaptUI \ac{api} methods.}\\
  \label{tbl:api_knowledge} \\
  \hline 
  \textbf{Method}		& \textbf{Description}\\
  \hline
  createClass(namespace, className)	& Inserts a new class.		\\
  removeClass(namespace, className)	& Erases an existing class, its	\\
					& individuals, attributes and	\\
					& relationships with other 	\\
					& classes.			\\
%   editClass(namespace, newClassName)	& Changes the name of the class.\\
% 					& Internally it deletes the 	\\
% 					& class passed as parameter 	\\
% 					& and creates a new one.	\\
  \hline 
  createObjectProperty	(namespace, objectPropName,& Inserts a new object\\
  Map$<$namespace, classname$>$)	& property connecting two classes.\\
  removeObjectProperty(namespace, objectPropName)& Erases an existing datatype\\
 					& property.			\\
%   editObjectProperty(namespace, objectPropName)& Changes the name of the\\
% 					& datatype property. Internally \\
% 					& it deletes the property 	\\
% 					& passed as parameter and 	\\
% 					& creates a new one.		\\
  \hline 
  createDatatypeProperty(namespace, dataTypeName)	& Inserts a new datatype\\
					& property assigning a value to \\
 					& a class.			\\
					& If it does not exist, it is	\\
					& created.			\\
  removeDatatypeProperty(namespace, dataTypeName)& Erases an existing datatype\\
 					& property.			\\
%   editDatatypeProperty(namespace, dataTypeName)	& Changes the name of the \\
% 					& datatype property. Internally \\
% 					& it deletes the property passed\\
% 					& as parameter and creates 	\\
%  					& a new one.			\\
  \hline 
  createIndividual(name, namespace, className)& Inserts a new individual of a \\
					& concrete class.		\\
  removeIndividual(name)	 	& Removes an individual.	\\

  \hline 
  createSWRLRule(name, ant\_classes, 	& Inserts a new rule, including	\\
  ant\_dataprop, ant\_objectprop, cons\_classes,& the set of antecedent and\\
  cons\_dataprop, cons\_objprop)	& consequent classes and properties.\\
  removeRule(name)	 		& Removes a rule if the rule 	\\
					& has a name associated to it.	\\
  \hline
\end{longtable}
\end{center}

Listing~\ref{lst:api_knowledge} shows an example of how AdaptUI manages the
knowledge modification through this \ac{api}. 


\inputminted[linenos=true, fontsize=\footnotesize, frame=lines]{java}{4_system_architecture/api_knowledge.java}
\captionof{listing}{Modifying the ontology knowledge with the AdaptUI's \ac{api}.\label{lst:api_knowledge}}